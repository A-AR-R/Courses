\documentclass[11pt]{article}
\input{\string~/.macros.tex}
\newcommand{\oa}{\overline{a}}
\newcommand{\ob}{\overline{b}}
\newcommand{\oc}{\overline{c}}
\newcommand{\oi}{\overline{i}}

\newcommand{\integermodn}[1][n]{\Z/#1\Z}
\newcommand{\integermodnmul}[1][n]{(\Z/#1\Z)^{\times}}
\newcommand{\order}[1]{\left|#1\right|}
\newcommand{\modb}[1]{\left(mod \;\; #1 \right)}
\newcommand{\aut}[1]{Aut\left(#1\right)}
\newcommand{\actson}{\ensuremath{\curvearrowright}}

\newcommand{\heading}[1]{(#1)}
\newcommand{\bheading}[1]{\textbf{(#1)}}

% arg1=pdfurl arg2=pagenum arg3=text
\usepackage{url}
\usepackage{hyperref}
\hypersetup{colorlinks=true, linktoc=all, linkcolor=blue}
\newcommand{\linkbook}[3][../../abstract_algebra_dummit_and_foote.pdf]{
    \noindent\href[page=#2]{#1}{\urlstyle{rm}{#3}}
}

\usepackage[a4paper, total={6in, 8in}, margin=0.5in]{geometry}


\begin{document}

\begin{center}
    {\Huge Preliminaries}
\end{center}

\tableofcontents
\newpage


\section{\linkbook{14}{Basics}}


\begin{definition*}
\textbf{(Functions)} Let $f:A\to B$
\begin{enumerate}
    \item \bheading{injection} $a_1\neq a_2 \Rightarrow f(a_1) \neq f(a_2)$
    \item \bheading{surjection} image of $f$ is all of $B$, i.e. $\forall b\in B \;\; \exists a\in A \;\; f(a)=b$
    \item \bheading{left inverse} a function $g:B\to A$ such that $g\circ f:A\to A$ is the identity map on $A$
    \item \bheading{right inverse} a function $h:B\to A$ such that $f\circ h:B\to B$ is the identity map on $B$
\end{enumerate}
\end{definition*} 
    
\begin{proposition*} Let $f:A\to B$
\begin{enumerate}
    \item $f$ is injective if and only if $f$ has a left inverse
    \item $f$ is surjective if and only if $f$ has a right inverse
    \item $f$ is bijective if exists $g:B\to A$ such that $f\circ g$ is identity map on $B$ and $g\circ f$ is identity map on $A$ ($g$ is the two-sided inverse)
    \item If $A,B$ are finite sets and $|A|=|B|$, then $f$ is bijective iff $f$ is injective iff it is surjective
\end{enumerate}
\end{proposition*}
  
\begin{definition*}
    \textbf{(Permutation, Restriction, Extension)}
    \begin{enumerate}
        \item \bheading{permutation} of set $A$ is a bijection from $A$ to itself
        \item \bheading{restriction} If $A\subset B$ and $f:B\to C$, $f|_{A}$ is restriction of $f$ to $A$.
        \item \bheading{extension} If $ A\subset B$ and $g:A\to C$ and there is a function $f:B\to C$ such that $f|_{A}=g$, then $f$ is an extension of $g$ to $B$
    \end{enumerate}
\end{definition*}

\begin{definition*}
    \textbf{(Equivalence Relation \& Partition)}
    \begin{enumerate}
        \item \bheading{binary relation} on a set $A$ is a subset $R$ of $A\times A$ and we write $a\sim b$ if $(a,b)\in R$
        \item \bheading{relation} $\sim$ on $A$ is an equivalence relation if it is 
        \begin{itemize}
            \item (reflexive) $a\sim a$ for all $a\in A$
            \item (symmetric) $a\sim b$ implies $b\sim a$, for all $a,b\in A$
            \item (transitive) $a\sim b$ and $b\sim c$ implies $a\sim c$ for all $a,b,c\in A$
        \end{itemize}
        \item \bheading{equivalence class} Given $\sim$ on $A$, the equivalence class of $a \in A$ is $\pc{x\in A \mid x\sim a}$. If $C$ is any equivalence class, any element of $C$ is a representative to class $C$
        \item \bheading{partition} of $A$ is any collection $\pc{A_i \mid i\in I}$ of nonempty subsets of $A$, for some indexing set $I$ such that 
        \begin{itemize}
            \item $A = \cup_{i\in I} A_i$
            \item $A_i \cap A_j = \emptyset$ for all $i,j\in I$ with $i\neq j$
        \end{itemize}
    \end{enumerate}
\end{definition*}

\begin{proposition*} \textbf{(Equivalence relation and partition are the same)} Let $A$ be nonempty set
    \begin{enumerate}
        \item If $\sim$ is an equivalence relation on $A$ then the set of equivalence classes of $\sim$ forms a partition of $A$
        \item If $\pc{A_i\mid i\in I}$ is a partition of $A$ then there is an equivalence relation on $A$ whose equivalence classes are precisely the sets $A_i$, $i\in I$ 
    \end{enumerate}
\end{proposition*}


\newpage
\section{\linkbook{17}{Properties of Integers}}

\begin{definition*}
    \textbf{(Properties of $\Z$)}
    \begin{enumerate}
        \item \bheading{well ordering of $\Z$} If $A \subset \Z^+$, exists $m\in A$ such that $m\leq a$ for all $a\in A$ ($m$ is minimal element of $A$)
        \item \bheading{divides} If $a,b\in \Z$ and $a\neq 0$, $a \mid b$ if there is an element $c\in \Z$, such that $b=ac$. Otherwise, $a\nmid b$ 
        \item \bheading{g.c.d.} If $a,b\in \Z - \pc{0}$, there is unique $d \in \Z^+$, the greatest common divisor $\p{a,b}$ of $a,b$ satisfying
        \begin{enumerate}
            \item $d$ is a common divisor of $a,b$ ($d\mid a$ and $d\mid b$ )
            \item $d$ is greatest such divisor (If $e \mid a$ and $e\mid b$, then $e\mid d$ )
        \end{enumerate}
        Intuitively, an a-by-b rectangle can be covered with square tiles of side-length c only if c is a common divisor of a and b. gcd of $a$ and $b$ is the largest of such $c$
        \item \bheading{relative prime} If $\p{a,b}=1$, then $a,b$ are relative prime
        \item \bheading{l.c.m} If $a,b\in \Z - \pc{0}$. there is unique $l\in \Z^+$, the least common multiple of $a,b$ satisfying 
        \begin{enumerate}
            \item $l$ is a common multiple of $a$ and $n$ ($a\mid l$ and $b\mid l$)
            \item $l$ is least of such multiple (If $a\mid m$ and $b\mid m$, then $l\mid m$)
        \end{enumerate} 
        \item \bheading{Relation between g.c.d. and l.c.m} Let $a,b\in \Z - \pc{0}$, let $d = \p{a,b}$ and $l = l.c.m.(a,b)$, then $dl=ab$
        \item \bheading{The Division Algorithm} If $a,b\in \Z - \pc{0}$ there exist unique $q,r\in \Z$ such that $a=qb+r$ and $0\leq r < |b|$, where $q$ is the quotient and $r$ is the reminder. 
        \item \bheading{Euclidean Algorithm} is a procedure that generates g.c.d. of two integers by iterating the division algorithm. Idea is g.c.d. of $a,b$ where $a>b$ is same as g.c.d. of $b,a-b$. Or equivalently. 
        \begin{align*}
            a &= q_0 b + r_0 \\
            b &= q_1 r_0 + r_1 \\
            r_0 &= q_2 r_1 + r_2 \\
            \vdots \\
            r_{n-2} &= q_n r_{n-1} + r_n \\
            r_{n-1} &= q_{n+1} r_n
        \end{align*}
        where $r_n = \p{a,b}$ is the last nonzero reminder
        \item \bheading{Consequence of Euclidean Algorithm} If $a,b\in \Z -\pc{0}$, then exists $x,y\in\Z$ such that 
        \[
            \p{a,b} = ax + by    
        \]
        by reversing steps of Euclidean algorithm
        \item \bheading{prime} $p\in \Z^+$ is called a prime if $p>1$ and the only positie divisors of $p$ are 1 and $p$. An integer greater than 1 which is not prime is composite. For any prime number $p$ where $p\mid ab$ for some $a,b\in\Z$, then either $p\mid a$ or $p \mid b$
        \item \bheading{foundamental theorem of arithemtic} If $n\in \Z$ and $n>1$, then $n$ can be factored uniquely into products of primes, i.e. exists distinct $p_1,\cdots,p_s$ and $\alpha_1,\cdots, \alpha_s$ such that 
        \[
            n = p_1^{\alpha_1} p_2^{\alpha_2} \cdots p_s^{\alpha_s}
        \]
        Additionally suppose $a = p_1^{\alpha_1} \cdots p_s^{\alpha_s}$ and $b = p_1^{\beta_1} \cdots p_s^{\beta_s}$ where $\alpha_i,\beta_i$ can be 0. then 
        \[
            \p{a,b} = p_1^{\min(\alpha_1,\beta_1)} p_2^{\min(\alpha_2,\beta_2)} \cdots p_s^{\min(\alpha_s,\beta_s)}
        \]
        and l.c.m. is obtained by taking maximum of $\alpha_i,\beta_i$ instead of minimum
        \begin{itemize}
            \item $57970 = 2\cdot 5 \cdot 11 \cdot 17 \cdot 31$ and $10353 = 3\cdot 7 \cdot 17 \cdot 19$, then $\p{57970, 10353} = 17$
        \end{itemize}
        \item \bheading{Euler $\varphi$-function} for $n\in \Z^+$, let $\varphi(n)$ be number of positive integers $a\leq n$ with $a$ relative prime to $n$, i.e. $\p{a,n} = 1$.
        \begin{itemize}
            \item $\varphi(12)=4$ (1,5,6,7)
        \end{itemize}
    \end{enumerate}
\end{definition*}


\newpage
\section{\linkbook{21}{$\integermodn$: The integers modulo n}}

\begin{definition*}
    \bheading{Integer Modulo n}
    \begin{enumerate}
        \item \bheading{modulo relation} Define $a\sim b$ \underline{iff} $n \mid (b-a)$. $\sim$ satisfies axioms for a relation
        \item \bheading{congruence} $a$ is congruent to $b \mod n$ \underline{iff} $a \equiv b \modb{n}$ \underline{iff} $a\sim b$
        \item \bheading{congruence/residue class of $a \mod n$} is the equivalence class by congruent modulo $n$, consisting of integers which differ from $a$ by an integral multiple of $n$, i.e. 
        \[
            \overline{a} = \pc{a+kn \mid k\in \Z}        
        \]
        There are $n$ distanct equivalence classes $\mod n$, i.e. $\pc{\overline{0}, \overline{1}, \cdots, \overline{n-1}}$. Specifically. $\oi$ are integers which leave a reminder of $i$ when divided by $n$
        \item \bheading{integer modulo $n$ group} $\integermodn = (\pc{\overline{0}, \overline{1}, \cdots, \overline{n-1}}, \sim)$
        \item \bheading{reducing $a \mod n$} is the process of finding the equivalence class mod $n$ of some integer $a$. Specifically, this is refering to finding the smallest nonnegative integer congruent to $a\mod n$
        \item \bheading{modular arithmetic} Let $\oa,\ob \in \integermodn$, define sum and product by $\oa + \ob = \overline{a+b}$ and $\oa \cdot \ob = \overline{ab}$.
        \item \bheading{theorem} Modular Arithmetic on $\integermodn$ is well defined; the sum/product of the residue classes does not depend on the choice of representatives chosen. Specifically, if $a_1,a_2,b_1,b_2\in \Z$ with $\overline{a_1} = \overline{b_1}$ and $\overline{a_2} = \overline{b_2}$ then $\overline{a_1 + a_2} = \overline{b_1 + b_2}$ and $\overline{a_1 a_2} = \overline{b_1 b_2}$.
    \end{enumerate}
    \begin{itemize}
        \item $\p{\integermodn}^{\times} \subset \integermodn$ are residue classes which have a multiplicative inverse
        \[
            \p{\integermodn}^{\times} = \pc{
                \oa \in \integermodn \mid
                \exists \oc \in \integermodn \;\; \oa \cdot \oc = \overline{1}
            } = \pc{
                \oa \in \integermodn \mid \p{a,n} = 1
            }
        \]
        \item \heading{example} $\integermodnmul[9] = \pc{\overline{1}, \overline{2}, \overline{4}, \overline{5}, \overline{7}, \overline{8}}$ ($\p{3,9}\neq 1$ and $\p{6,9}\neq 1$), with inverses $\pc{\overline{1}, \overline{5}, \overline{7}, \overline{2}, \overline{4}, \overline{8}}$ 
        \item \heading{method} for computing inverse of $\overline{a} \subset \integermodnmul$. The condition for inverse is $\overline{a a^{-1}} = \overline{1}$ or $aa^{-1} \equiv 1 \modb{n}$. Since $\overline{a}$ is in $\integermodnmul$, $\p{a,n}=1$ holds, then exists $x,y\in \Z^+$ such that $ax + ny = 1$, i.e. $ax \equiv 1 \modb{n}$ the desired condition for inverses. Therefore, $\overline{x}$ is the multiplicative inverse of $\overline{a}$. So to find inverse for $\overline{a}$, we simply use Euclidean algorithm to compute the coefficient $x$
        \item \heading{example} For $\integermodnmul[60]$ and $a=17$. Apply Euclidean algorithm, 
        \begin{align*}
            60 &= (3) 17 + 9 \\
            17 &= (1) 9  + 8 \\
            9  &= (1) 8  + 1
        \end{align*}
        $\p{a,n}=1$ so $\oa \in \integermodnmul[60]$ and $(-7)17 + (1)60 = 1$. So $\overline{-7} = \overline{53}$ is multiplicative inverse of $\overline{17}$
    \end{itemize}
\end{definition*} 



 


\end{document}
