\documentclass[11pt]{article}
\input{\string~/.macros.tex}
\newcommand{\oa}{\overline{a}}
\newcommand{\ob}{\overline{b}}
\newcommand{\oc}{\overline{c}}
\newcommand{\oi}{\overline{i}}

\newcommand{\integermodn}[1][n]{\Z/#1\Z}
\newcommand{\integermodnmul}[1][n]{(\Z/#1\Z)^{\times}}
\newcommand{\order}[1]{\left|#1\right|}
\newcommand{\modb}[1]{\left(mod \;\; #1 \right)}
\newcommand{\aut}[1]{Aut\left(#1\right)}
\newcommand{\actson}{\ensuremath{\curvearrowright}}

\newcommand{\heading}[1]{(#1)}
\newcommand{\bheading}[1]{\textbf{(#1)}}

% arg1=pdfurl arg2=pagenum arg3=text
\usepackage{url}
\usepackage{hyperref}
\hypersetup{colorlinks=true, linktoc=all, linkcolor=blue}
\newcommand{\linkbook}[3][../../abstract_algebra_dummit_and_foote.pdf]{
    \noindent\href[page=#2]{#1}{\urlstyle{rm}{#3}}
}

\usepackage[a4paper, total={6in, 8in}, margin=0.5in]{geometry}


\begin{document}
\begin{center}
    {\Huge Chapter 2 Subgroups}
\end{center}
\tableofcontents
\newpage


\section{\linkbook{59}{Definition and Examples}}


\begin{definition*}
    \bheading{Subgroup}
    \begin{enumerate}
        \item \bheading{subgroup} Let $G$ be a group. The subset $H$ of $G$ is a subgroup of $G$, denoted as $H\leq G$ if
        \begin{enumerate}
            \item $H$ is nonempty
            \item $H$ is closed under products and inverses, i.e. $x,y\in G$ implies $x^{-1},xy\in H$
        \end{enumerate}
        If $H\leq G$ and $H\neq G$, then $H < G$. $H\leq G$ implies operation on $H$ is the operation on $G$ restricted to $H$. So any equation in $H$ can also be viewed as equation in $G$
        \item \bheading{The Subgroup Criterion} $H \subset G$ is a subgroup if and only if
        \begin{enumerate}
            \item $H\neq \emptyset$
            \item for all $x,y\in H$, $xy^{-1} \in H$
        \end{enumerate}
        Furthermore, if $H$ is finite, then suffice to check $H$ is nonempty and closed under multiplication
    \end{enumerate}
    \begin{itemize}
        \item \heading{examples}
        \begin{itemize}
            \item $G\leq G$ and $\pc{1}\leq G$ (latter is called the trivial subgroup)
            \item $\Z \leq \Q \leq \R$ under operation of addition 
            \item $\pc{1,r,r^2,\cdots,r^{n-1}} \leq D_{2n}$
            \item $2\Z \leq \Z$
            \item $(\Q^{\times}, \times) \not\leq (\R, +)$ (operation are different)
            \item $\Z^+ \leq \Z$ and $(\Z^+)^{\times} \not\leq \Q^{\times}$ (not closed under inverses and does not contain identity)
            \item $D_6 \not\leq D_8$ ($D_6 \not\subset D_8$)
        \end{itemize}
        \item \bheading{theorem} subgroup is a transitive relation, i.e. $K\leq H, H\leq G$, then $K\leq G$ 
    \end{itemize}
\end{definition*}
 


\section{\linkbook{62}{Centralizers and Normalizers, Stabilizers and Kernels}}



\end{document}
