\documentclass[11pt]{article}
\input{\string~/.macros.tex}
\newcommand{\oa}{\overline{a}}
\newcommand{\ob}{\overline{b}}
\newcommand{\oc}{\overline{c}}
\newcommand{\oi}{\overline{i}}

\newcommand{\integermodn}[1][n]{\Z/#1\Z}
\newcommand{\integermodnmul}[1][n]{(\Z/#1\Z)^{\times}}
\newcommand{\order}[1]{\left|#1\right|}
\newcommand{\modb}[1]{\left(mod \;\; #1 \right)}
\newcommand{\aut}[1]{Aut\left(#1\right)}
\newcommand{\actson}{\ensuremath{\curvearrowright}}

\newcommand{\heading}[1]{(#1)}
\newcommand{\bheading}[1]{\textbf{(#1)}}

% arg1=pdfurl arg2=pagenum arg3=text
\usepackage{url}
\usepackage{hyperref}
\hypersetup{colorlinks=true, linktoc=all, linkcolor=blue}
\newcommand{\linkbook}[3][../../abstract_algebra_dummit_and_foote.pdf]{
    \noindent\href[page=#2]{#1}{\urlstyle{rm}{#3}}
}

\usepackage[a4paper, total={6in, 8in}, margin=0.5in]{geometry}


\begin{document}
\begin{center}
    {\Huge Chapter 2 Subgroups}
\end{center}
\tableofcontents
\newpage


\section{\linkbook{59}{Definition and Examples}}


\begin{definition*}
    \bheading{Subgroup}
    \begin{enumerate}
        \item \bheading{subgroup} Let $G$ be a group. The subset $H$ of $G$ is a subgroup of $G$, denoted as $H\leq G$ if
        \begin{enumerate}
            \item $H$ is nonempty
            \item $H$ is closed under products and inverses, i.e. $x,y\in G$ implies $x^{-1},xy\in H$
        \end{enumerate}
        If $H\leq G$ and $H\neq G$, then $H < G$. $H\leq G$ implies operation on $H$ is the operation on $G$ restricted to $H$. So any equation in $H$ can also be viewed as equation in $G$
        \item \bheading{The Subgroup Criterion} $H \subset G$ is a subgroup if and only if
        \begin{enumerate}
            \item $H\neq \emptyset$
            \item for all $x,y\in H$, $xy^{-1} \in H$
        \end{enumerate}
        Furthermore, if $H$ is finite, then suffice to check $H$ is nonempty and closed under multiplication
    \end{enumerate}
    \begin{itemize}
        \item \heading{examples}
        \begin{itemize}
            \item $G\leq G$ and $\pc{1}\leq G$ (latter is called the trivial subgroup)
            \item $\Z \leq \Q \leq \R$ under operation of addition 
            \item $\pc{1,r,r^2,\cdots,r^{n-1}} \leq D_{2n}$
            \item $2\Z \leq \Z$
            \item $(\Q^{\times}, \times) \not\leq (\R, +)$ (operation are different)
            \item $\Z^+ \leq \Z$ and $(\Z^+)^{\times} \not\leq \Q^{\times}$ (not closed under inverses and does not contain identity)
            \item $D_6 \not\leq D_8$ ($D_6 \not\subset D_8$)
        \end{itemize}
        \item \bheading{theorem} subgroup is a transitive relation, i.e. $K\leq H, H\leq G$, then $K\leq G$ 
    \end{itemize}
\end{definition*}
 


\section{\linkbook{62}{Centralizers and Normalizers, Stabilizers and Kernels}}


\begin{definition*}
    \bheading{Centralizers and Normalizers} Let $G$ be a group and $A\subset G$ be nonempty
    \begin{enumerate}
        \item \bheading{centralizer} The centralizer of $A$ in $G$ is a subset of $G$ which commute with every element of $A$
        \[
            C_G(A) = \pc{
                g\in G \mid gag^{-1} = a \text{ for all }  a\in A
            }    
        \]
        \begin{itemize}
            \item  $ga=ag \iff gag^{-1}=a$
        \end{itemize}
        \item \bheading{center} The center of $G$ is a subset of $G$ which commutes with all the elements of $G$
        \[
            Z(G) = C_G(G) = \pc{
                g\in G \mid gx = xg \text{ for all } x\in G
            }    
        \]
        \item \bheading{normalizer} The normalizer of $A$ in $G$ are subsets of $G$ that \underline{fixes} $A$ by conjugation
        \[
            N_G(A) = \pc{
                g\in G \mid gAg^{-1} = A
            }
        \]
        where $gAg^{-1} = \pc{gag^{-1} \mid a\in A}$
    \end{enumerate}
    \begin{itemize}
        \item \heading{convention} For $A=\pc{a}$, write $C_G(a)$ instead of $C_G(\pc{a})$. Note $a^n \in C_G(a)$ for all $n\in\Z^+$
        \item \bheading{theorem} $Z(G) \subgroup C_G(A) \subgroup N_G(A) \subgroup G$
        \begin{proof}
            proofs for $C_G(A)$ and $N_G(A)$ are subgroups of $G$ are similar. For now want to show $N_G(A)\subgroup G$. Note $1 \in N_G(A)$ so $N_G(A) \neq \emptyset$. Let $g_1,g_2\in N_G(A)$, then $g_1Ag_1^{-1} = A$ and $g_2Ag_2^{-1} = A$. therefore 
            \[
                g_1g_2^{-1} A (g_1 g_2^{-1})^{-1} = g_1g_2^{-1} (g_2 A g_2^{-1}) g_2 g_1^{-1} = g_1 A g_1^{-1} = A  
            \]
            hence $g_1g_2^{-1} \in N_G(A)$. So $N_G(A)\subgroup G$.
        \end{proof}
        \item \heading{examples}
        \begin{itemize}
            \item If $G$ is abelian 
            \begin{itemize}
                \item $Z(G)=G$
                \item $C_G(A) = N_G(A) = G$ for any subset $A$ ($gag^{-1} = gg^{-1}a = a$ for all $a\in A$, $g\in G$)
            \end{itemize}
            \item $C_{Q_8}(i) = \pc{\pm 1, \pm i}$
            \item Let $G=D_8$ and $A = \pc{1,r,r^2,r^3} \subgroup G$ be subgroup of rorations
            \begin{itemize}
                \item $C_{D_8}(A) = A$
                \item $N_{D_8}(A) = D_8$
                \item $Z(D_8) = \pc{1,r^2}$
            \end{itemize}
            \begin{proof}
                \textbf{(1)} Since all powers of $r$ commutes with each other, $A\subgroup C_{D_8}(A)$. since $sr^i = r^{-i}s \neq r^is$, $s$ does not commute with any rotation, so $s\not\in C_{D_8}(A)$. In fact, any $sr^i \not\in C_{D_8}(A)$ where $i\in \pc{0,1,2,3}$. If assume for contradiction, $s = (sr^i)(r^{-i}) \in C_{D_8}(A)$, a contradiction. Hence $C_{D_8} = A$. \textbf{(2)} Note, $A \subgroup N_{D_8}(A)$ by fact that centeralizer is contained in normalizer. Now consider 
                \[
                    sAs^{-1} = \pc{
                        s1s^{-1}, srs^{-1}, sr^2s^{-1}, sr^3s^{-1}
                    } = \pc{
                        1, r^3,r^2,r
                    } = A
                \]
                so that $s\in N_{D_8}$. Since $r,s\in N_{D_8}(A)$ and $N_{D_8}$ is closed under multiplication (its a subgroup!), $s^i r^j \in N_{D_8}$ for all $i,j$. $D_8 \subgroup N_{D_8}$, hence $N_{D_8}(A) = D_8$ \textbf{(3)} Note, $Z(D_8) \subgroup A$ by fact that center is contained in the centralizer. Note $sr = r^{-1}s = r^3 s \neq rs$ and $sr^3 = r^{-3}s = rs \neq r^3s$ hence $r,r^3\not\in Z(D_8)$ (but $sr^2 = r^{-2}s = r^2s$). Therefore $Z(D_8) \subgroup \pc{1,r^2}$. The reverse inclusion holds by $1$ (and $r^2$) commutes with $r$ and $s$. Since $r,s$ generates $D_8$, every element of $D_8$ commutes with $1$ (and $r^2$) hence $\pc{1,r^2} \subgroup Z(D_8)$ and so equality holds.
            \end{proof}
            \item Let $G=S_3$ and $A=\pc{1, \p{1\;2}}$,
            \begin{itemize}
                \item $C_{S_3}(A) = A$
                \item $N_{S_3}(A) = A$
                \item $Z(S_3) = \pc{1}$
            \end{itemize}
            \begin{proof}
                \textbf{(1)} Both $1$ and $\p{1\;2}$ commutes with all of $A$ hence $A \subgroup C_{S_3}(A)$ (for $a,g=(1\;2)$, $gag^{-1} = (1\;2)(1\;2)(2\;1) = (1\;2)$, commutativity with $a=1$ or $g=1$ is trivial). To show $C_{S_3}(A) \subgroup A$, enough to show that both $(2\;3)$ and $(1\;3)$ do not commute with all elements of $A$, specifically $(1\;2)$ (by fact that transpositions generates $S_3$). $(2\;3)(1\;2) = (1\;3\;2) \neq (1\;2\;3) = (1\;2)(2\;3)$ similarly for $(1\;3)$. Alternatively, by Lagrange theorem, $\order{C_{S_3}(A)} \mid \order{S_3}=6$ and $2 = \order{A} \mid \order{C_{S_3}(A)}$. Possible values for $\order{C_{S_3}(A)}$ are 2 or 6. If latter is true, then $C_{S_3}(A) = S_3$ but this is a contradiction since $(2\;3)$ does not commute with $(1\;2)$. So $\order{C_{S_3}(A)} = 2$ hence $C_{S_3}(A)=A$. \textbf{(2)} Note $N_{S_3}(A)=A$ because $\sigma \in N_{S_3}(A)$ if and only if 
                \[
                    \sigma A\sigma^{-1}
                    = \pc{\sigma 1 \sigma^{-1}, \sigma (1\;2) \sigma^{-1}}
                    = \pc{1, (1\;2)}
                    = A
                \]
                if and only if $\sigma (1\;2) \sigma^{-1} = (1\;2)$, i.e. $\sigma\in C_{S_3}(A) =A$. \textbf{(3)} $Z(S_3) \subgroup C_{S_3}(A)=A$ and $(1\;2)\not\in Z(S_3)$
            \end{proof}
        \end{itemize}
    \end{itemize}
\end{definition*}

\begin{definition*}
    \bheading{Stabilizers and Kernels of Group Actions}
    \begin{enumerate}
        \item \bheading{stabilizer} If $G$ is a group acting on a set $S$ and $s\in S$ is a fixed element, the stabilizer of $s$ in $G$ is 
        \[
            G_s = \pc{g\in G \mid g\cdot s = s}    
        \]
        \item \bheading{kernel} of action of $G$ on $S$ is defined as 
        \[
            \ker{\varphi} = \pc{
                g\in G \mid g\cdot s = s \text{ for all } s\in S
            }
        \]
        \item \bheading{centralizers and normalizers as kernels of some group action}
        \begin{enumerate}
            \item \heading{normalizer} Let $G\actson \sP(G)$ by conjugation, i.e. for any $g\in G$ and $B\subset G$
            \[
                g: B \to gBg^{-1} 
                \quad \text{ where } \quad
                gBg^{-1} = \pc{gbg^{-1} \mid b\in B}
            \]
            This is a group action. \\ 
            \underline{The normalizer of $G$ on $A$ is the stabilizer of $A$ when $G$ acts on $\sP(G)$ by conjugation} \\
            , i.e. $N_G(A) = G_s$ where $s=A\subset \sP(G)$. Therefore $N_G(A) \subgroup G$
            \item \heading{centralizer} Let $N_G(A) \actson A$ by conjugation, i.e. for any $g\in N_G(A)$ and $a\in A$
            \[
                g: a\to gag^{-1}    
            \]
            which maps $A$ to $A$ by definition of $N_G(A)$ fixing $A$ and so gives an action on $A$. \\  \underline{The centralizer of $G$ on $A$ is simply the kernel of $N_G(A)$ acting on $A$ by conjugation}. 
            \[
                \ker{(G\actson S)} = \pc{
                    g\in G \mid g\cdot s = s \text{ for all } s\in S
                }
                = \pc{
                    g\in G \mid gsg^{-1} = s \text{ for all } s\in S
                }
                = C_G(S)
            \]
            \\ Since $C_G(A)\subgroup N_G(A)$ and $N_G(A)\subgroup G$, we have $C_G(A)\subgroup G$
            \item \heading{center} \underline{The center of $G$ is the kernel of $G$ acting on $S=G$ by conjugation}
        \end{enumerate} 
    \end{enumerate}
    \begin{itemize}
        \item \bheading{theorem} $\ker{(G\actson S)} \subgroup G$
        \item \bheading{theorem} $G_s \subgroup G$ ($1\in G_s$ and $(xy^{-1}) \cdot s = (xy^{-1}) \cdot (y\cdot s) = x\cdot s = s$ for any $x,y\in G_s$)
        \item \heading{examples}
        \begin{itemize}
            \item Let $G=D_8$ and $A =\pc{1,2,3,4}$ the vertices of a square. Then the stabilizer of any vertex $a\in A$ is the subgroup $\pc{1,t} \subgroup D_8$, where $t$ is the reflection about line of symmetry passing through $a$ and center of the square. The kernel of the action is just the identity
            \item Let $G=D_8$ and $A = \pc{\pc{1,3}, \pc{2,4}}$ be the two unordered pairs of opposite vertices. The kernel of the action of $G$ on $A$ is the subgroup $\pc{1,s,r^2,sr^2}$ and for any $a\in A$, the stabilizer of $a$ in $D_8$ is equal to the kernel of the action
        \end{itemize}
    \end{itemize}
\end{definition*}



\section{\linkbook{67}{Cyclic Groups and Cyclic Subgroups}}

\begin{definition*}
    \bheading{cyclic group} A group $H$ is cyclic if $H$ can be generated by a single element, i.e. there is some element $x\in H$ such that $H = \pc{x^n \mid n\in\Z}$ in multiplicative notation (or that $H=\pc{nx \mid n\in \Z}$ in additive notation). We write $H=\pa{x}$ and say $H$ is \textbf{generated by} $x$ and $x$ is a \textbf{generator}. For any $n\in\Z^+$, let $Z_n$ be the \underline{cyclic group of order $n$} (written multiplicatively)
    \begin{itemize}
        \item \heading{fact} A cyclic group may have more than one generator ($H=\pa{x}$ implies $H=\pa{x^{-1}}$)
        \item \heading{fact} not all powers of the generator are distinct, i.e. possibly $x^n = x^m$ where $n\neq m$
        \item \heading{fact} cyclic group is abelian (law of exponent)
        \item \heading{examples}
        \begin{itemize}
            \item all rotations of a regular n-gon $H=\pa{r} = \pc{1,r,r^2,\cdots,r^{n-1}}$ is a cyclic subgroup of $D_{2n}$ 
            \begin{itemize}
                \item $\order{H} = \order{r} = n$
                \item we can reduce arbitrary powers of a generator in a finite cyclic group to the least residual power, i.e. $r^t = r^{nq+k} = (r^n)^q r^k = 1^q r^k = r^k$ for some $0\leq k < n$
            \end{itemize}
            \item $H=\Z = \pa{1}$ is a cyclic group, since any element in $H$ can be written as $n\cdot 1$.
            \begin{itemize}
                \item $\order{H} = \order{1} = \infty$
            \end{itemize}
        \end{itemize}
        \item \bheading{proposition} order of a cyclic group is the order of its generator, i.e. if $H=\pa{x}$, then $\order{H} = \order{x}$
        \begin{enumerate}
            \item \heading{$\order{H} = n < \infty$}: $x^n=1$ and $1,x,x^2,\cdots,x_{n-1}$ are distinct
            \item \heading{$\order{H} = \infty$}: $x^n \neq 1$ for all $n\neq 0$ and $x^a \neq x^b$ for all $a\neq b\in \Z$
        \end{enumerate}
        \item \bheading{proposition} Let $G$ be a group. Let $m,n\in\Z$, then
        \begin{enumerate}
            \item $x^m=1$, $x^n=1$ implies $x^{\p{m,n}}=1$ (by Euclidean Algo, $x^{(m,n)} = x^{mr+ns} = (x^m)^r + (x^n)^s = 1^r1^s=1$)
            \item $x^m=1$ implies $\order{x} \mid m$ (let $n=\order{x}$, by previous, $x^{(n, m)}=1$, $0<d\leq n$ implies $n = d\mid m$ by gcd)
        \end{enumerate}
        We can say something about the power $m$ when we have we know $x^m=1$
        \item \bheading{theorem} two cyclic group of same order are isomorphic (both finite and infinite case)
        \begin{itemize}
            \item \heading{examples}
            \begin{itemize}
                \item $(Z_n, \times) \cong (\integermodn, +)$
                \item $(\pa{x}, \times) \cong (\Z, +)$
            \end{itemize}
        \end{itemize}
        \begin{proof}
            $ $ \\
            \textbf{(finite case)} if $\pa{x}, \pa{y}$ are cyclic group of order $n\in\Z^+$, show $\varphi$ is an isomorphism
            \begin{align*}
                \varphi: \pa{x} &\to \pa{y} \\
                              x^k &\mapsto y^k
            \end{align*}
            \begin{itemize}
                \item \heading{well defined} let $x^r = x^s$ for some $r,s\in\Z$ and show $\varphi(x^r)=\varphi(x^s)$. \\
                $x^{r-s}=1$, hence $n \mid r-s$ and write $r-s = tn$, then $\varphi(x^r) = \varphi(x^{tn+s}) = y^{tn+s} = (y^n)^t y^s = y^s = \varphi(x^s)$
                \item \heading{homomorphism} $\varphi(x^ax^b) = \varphi(x^{a+b}) = y^{a+b} = y^ay^b = \varphi(x^a)\varphi(x^b)$
                \item \heading{bijection} $\varphi$ surjective since any $y^k$ is image of $x^k$ under $\varphi$. As $\order{\pa{x}} = \order{\pa{y}} = n$, $\varphi$ is bijective
            \end{itemize}
            \textbf{(infinite case)} If $\pa{x}$ is an infinite cyclic group, show $\varphi$ is an isomorphism 
            \begin{align*}
                \varphi: \Z &\to \pa{x} \\
                k   & \mapsto x^k 
            \end{align*}
            \begin{itemize}
                \item (well-defined) no ambiguity on $\Z$
                \item (homomorphism) by law of exponent
                \item (bijection) $\varphi$ surjective by definition of cyclic group. $\varphi$ injective by previous proposition, i.e. $x^a\neq x^b$ for all distinct $a, b\in \Z$.
            \end{itemize}
        \end{proof}
        \item \bheading{proposition} Let $G$ be a group, $x\in G$, and $a\in \Z - \pc{0}$
        \begin{enumerate}
            \item If $\order{x}=\infty$, then $\order{x^a}=\infty$
            \item If $\order{x}=n<\infty$, then $\order{x^a} = \frac{n}{(n,a)}$
            \item (special case to 2) if $\order{x}=n<\infty$ and $a \mid n$ where $a\in\Z^+$, then $\order{x^a} = \frac{n}{a}$ ($\p{n,a}=a$)
        \end{enumerate}
        Intuitively, we can say something about the order of $x^a$ when we know the order of $x$
        \begin{proof}
            \textbf{(1)} by contradiction, assume $\order{x^a}=1$, then $1 = (x^a)^m = x^{am}$, similarly, $x^{-am} = (x^{am})^{-1} = 1^{-1} = 1$. Either $am$ or $-am$ is positive, so some positive power of $x$ is the identity, contradicting $\order{x} =\infty$. \textbf{(2)} Let $y=x^a$ and $\p{n,a}=d$ and $n=db$ and $a=dc$. Note $\p{b,c}=1$. Let $\order{y}=k$ we show $b$ and $k$ divides each other hence proving equality
            \begin{itemize}
                \item \heading{$k\mid b$} $y^b = x^{ab} = x^{dcb} = x^{nc} = 1^c = 1$. by previous proposition on $\pa{y}$, $k \mid b$
                \item \heading{$b\mid k$} $x^{ak}=y^k=1$. by previous proposition on $\pa{x}$, $n\mid ak \Rightarrow db \mid dck \Rightarrow b\mid ck$, so $\p{b,c}=1 \Rightarrow b\mid k$.
            \end{itemize}
        \end{proof}
        \item \bheading{proposition} 
    \end{itemize}
\end{definition*}
 

\end{document}
