\documentclass[11pt]{article}
\input{\string~/.macros.tex}
\newcommand{\oa}{\overline{a}}
\newcommand{\ob}{\overline{b}}
\newcommand{\oc}{\overline{c}}
\newcommand{\oi}{\overline{i}}

\newcommand{\integermodn}[1][n]{\Z/#1\Z}
\newcommand{\integermodnmul}[1][n]{(\Z/#1\Z)^{\times}}
\newcommand{\order}[1]{\left|#1\right|}
\newcommand{\modb}[1]{\left(mod \;\; #1 \right)}
\newcommand{\aut}[1]{Aut\left(#1\right)}
\newcommand{\actson}{\ensuremath{\curvearrowright}}

\newcommand{\heading}[1]{(#1)}
\newcommand{\bheading}[1]{\textbf{(#1)}}

% arg1=pdfurl arg2=pagenum arg3=text
\usepackage{url}
\usepackage{hyperref}
\hypersetup{colorlinks=true, linktoc=all, linkcolor=blue}
\newcommand{\linkbook}[3][../../abstract_algebra_dummit_and_foote.pdf]{
    \noindent\href[page=#2]{#1}{\urlstyle{rm}{#3}}
}

\usepackage[a4paper, total={6in, 8in}, margin=0.5in]{geometry}


\begin{document}
\begin{center}
    {\Huge Chapter 1 Introduction to Groups}
\end{center}
\tableofcontents
\newpage


\section{\linkbook{29}{Basic Axioms and Examples}}


\begin{definition*} \bheading{Binary Operation}
    \begin{enumerate}
        \item \bheading{binary operation} $\star$ on a set $G$ is a function $\star:G\to G$. write $a\star b$ instead of $\star(a,b)$
        \item \bheading{associative $\star$} A binary operation on $G$ is associative if for all $a,b,c\in G$ $a\star (b\star c) = (a\star b) \star c$
        \item \bheading{commutative $\star$} A binary operation on $G$ is commutative if for all $a,b\in G$, $a\star b = b\star a$
        \item \bheading{closed under $\star$} $\star$ is a binary operation on $G$ and $H\subset H$, if $\star|_H$ is a binary operation on $H$, i.e. for all $a,b\in H$, $a\star b \in H$, then $H$ is closed under $\star$. Associativity/Commutativity of $\star$ is inherited on $H$
    \end{enumerate}
    \begin{itemize}
        \item \heading{examples}
        \begin{enumerate}
            \item $+$ on $\Z,\Q,\R,\C$ is a commutative binary operation
            \item $\times$ on $\Z,\Q,\R,\C$ is a commutative binary operation
            \item $-$ is not commutative on $\Z$ ($a-b \neq b-a$ usually)
            \item $-$ is not commutative on $\Z^+$ ($1,2\in\Z^+$, but $1-2 = -1 \not\in \Z^+$)
        \end{enumerate}
    \end{itemize}
\end{definition*}


\begin{definition*}
    \bheading{Group}
    \begin{enumerate}
        \item \bheading{group} A group is an ordered pair $(G,\star)$ where $G$ is a set and $\star$ is a binary operation on $G$ satisfying 
        \begin{enumerate}
            \item (associative) $\forall a,b,c\in G$, $(a\star b) \star c = a\star (b \star c)$
            \item (identity) $\exists e\in G \;\; \forall a\in G \;\;  a\star e = e \star a = a$ ($e$ is an identity of $G$, alternatively denoted by 1)
            \item (inverse) $\forall a\in G \;\; \exists a^{-1}\in G$, $a\star a^{-1}  = a^{-1} \star a = e$ ($a^{-1}$ is an inverse of $a$)
        \end{enumerate}
        \item \bheading{abelian group} A group if abelian/commutative if $a\star b = b\star a$ for all $a,b\in G$ 
        \item \bheading{finite group} $G$ is a finite group if $G$ is a finite set
        \item \bheading{direct product} If $(A, \star)$ and $(B, \circ)$ are groups, a new group $A\times B$ called direct product are defined as 
        \[
            A\times B = \pc{\p{a,b} \mid a\in A \;\; b\in B}
        \]
        with binary operation defined component-wise 
        \[
            (a_1, b_1)(a_2, b_2) = (a_1 \star a_2, b_1 \circ b_2)    
        \]
    \end{enumerate}
    \begin{itemize}
        \item \heading{examples}
        \begin{itemize}
            \item $\Z,\Q,\R,\C$ are groups under $+$ ($e=0$, $a^{-1} = -a$, associativity by axioms of $+$)
            \item $\Q - \pc{0}, \R-\pc{0}, \C-\pc{0}, \Q^+, \R^+$ are gorups under $\times$ ($e = 1$, $a^{-1} = 1/a$, associativity by $\times$))
            \item $(\Z - \pc{0}, \times )$ is not a group ($2^{-1} = 1/2\not\in \Z - \pc{0}$)
            \item $(V,+)$ is an abelian group, where $V$ is a vector space (commutativity by axioms of a vector space)
            \item $(\integermodn, +)$ is an abelian group ($e = \overline{1}$, $a^{-1} = \overline{-a}$)
            \item $(\integermodnmul, \times)$ is abelian group ($e = \overline{1}$, $a^{-1}$ exists by definition of $\integermodnmul$)
        \end{itemize}
        \item \bheading{theorem} direct product of two groups is a group
        \item \bheading{proposition}
        \begin{enumerate}
            \item (identity unique) identity of $G$ is unique
            \item (inverse unique) inverse $a^{-1}$ of any $a$ in $G$ is unique
            \item $(a^{-1})^{-1} = a$ for all $a$ in $G$
            \item $(a\star b)^{-1} = b^{-1} \star a^{-1}$
            \item (generalized associativity law) value of $a_1 \star a_2\star \cdots \star a_n$ independent of how its bracketed
        \end{enumerate}
        \item \heading{notation}
        \begin{itemize}
            \item ($\times$) denote $x^n = xx \cdots x$ by $x^n$ and $x^{-n} = x^{-1}x^{-1} \cdots x^{-1}$ and $x^0 = 1$ the identity
            \item ($+$) denote $na = a+a+ \cdots + a$ and $-na = -a-a-\cdots -a$ and $0a = 0$ the identity 
        \end{itemize}
        \item \bheading{proposition} Let $a,b,u,v\in G$
        \begin{enumerate}
            \item (left cancellation law holds) if $au = av$, then $u=v$ 
            \item (right cancellation law holds) if $ub = vb$, then $u=v$
        \end{enumerate}
    \end{itemize}
\end{definition*}

\begin{definition*}
    \bheading{order for an element $x\in G$} is the smallest positive integer $n\in\Z^+$ such that $x^n=1$, denoted by $\order{x}$. If no positive power of $x$ is the identity, the order of $x$ is defined to be infinity
    \begin{itemize}
        \item \heading{examples}
        \begin{itemize}
            \item if $|x|=1$, then $x=1$ the identity
            \item In $(\Z,\Q,\R,\C, +)$, every nonzero elements has infinite order
            \item In $(\R-\pc{0}, \Q-\pc{0}, \times)$, $|-1|=2$ and all other nonidentity elements have infinite order
            \item In $\integermodn[9]$, $\order{\overline{5}} = 9$ since $9$ is the smallest integer multiple of 5 that is congruent to $0 \modb{9}$
            \item In $\integermodnmul[7]$, $\order{\overline{3}} = 6$ since $3^6$ is smallest positive power of $3$ that is congruent to $1 \modb{7}$
        \end{itemize}
    \end{itemize}
\end{definition*}

\begin{definition*}
    \bheading{multiplication/group table} Let $G= \pc{g_1, g_2, \cdots, g_n}$ be a finite group where $g_1 = 1$. The multiplication or group table of $G$ is a $n\times n$ matrix $A$ where $A_{ij} = g_i g_j$. 
    \begin{itemize}
        \item \heading{fact} For finite groups, the group table contains all information about the group
    \end{itemize}
\end{definition*}



\section{\linkbook{36}{Dihedral Groups}}

\begin{definition*}
    \bheading{Dihedral Groups}
    \begin{enumerate}
        \item \bheading{symmetry of $n$-gon} is any rigid motion of the n-gon. We can describe symmetry by choosing a labelling of vertices $\pc{1,2,\cdots,n}$ and let the corresponding permutation $\sigma$ over the set as symmetry $s$
        \item \bheading{order of $D_{2n}$} is $2n$. (lower bound: vertex 1 can be sent to any vertex $i$, and vertex 2 can be sent to either $i-1$ or $i+1$. Knowing position of $1,2$ determines position of all other vertices; upper bound: by reasoning that any element of $D_{2n}$ can be written as $r^is^j$ where $0\leq i \leq n-1$ and $0\leq j \leq 1$)
        \item \bheading{dihedral group $D_{2n}$} Fix a regular $n$-gon at origin and label vertices through from 1 to n in a clockwise manner. Let $r$ be rotation clockwise about the origin through $2\pi/n$ radian and let $s$ be reflection about line of symmetry through vertex 1 and the origin.
        \[
            D_{2n} = \pc{
                r,s \mid r^n = s^2 = 1 \;, \;\; sr^k = r^{-k}s
            } = 
            \pc{
                1,r,r^2,\cdots,r^{n-1}, s,rs,r^2s,\cdots, r^{n-1}s
            }
        \]
        \begin{enumerate}
            \item $\order{r} = n$ and $\order{s} = 2$
            \item $s \neq r^i$ for any $i$ and $sr^i \neq sr^j$ for all $i\neq j$
            \item $r^ks = sr^{-k}$ for all $0\leq i \leq n$
        \end{enumerate}
        \item \bheading{interpreting presentation for $D_{2n}$} $r^n = 1$ means any power of $r$ can be reduced so that the power lie between 0 and $n-1$. Similarly, any power of $s$ can be reduced so that the power is either 0 or 1. $sr^k = r^{-k}s$ means every element in the group can be written as $r^i s^j$ for some $i,j$
    \end{enumerate}
    \begin{itemize}
        \item \heading{fact} $D_{2n}$ for $n\geq 3$ is non-abelian
    \end{itemize}
\end{definition*}



\begin{definition*}
    \bheading{generators and relations}
    \begin{enumerate}
        \item \bheading{generators of $G$} is the set $S\subset G$ where every element of $G$ can be written as a (finite) product of elements of $S$ and their inverses. Denote $G = \pa{S}$ and say $G$ is generated by $S$ and $S$ generates $G$ 
        \item \bheading{relations in $G$} any equation in a general group $G$ that the generator satisfies
        \item \bheading{presentation of $G$} If $G=\pa{S}$ and $R_1,R_2,\cdots,R_m$ are relations in $G$ such that any relation among $S$ can be deduced from these, the generators and relations are called presentations
        \[
            G = \pa{S \mid R_1, R_2, \cdots, R_m}    
        \]
    \end{enumerate}
    \begin{itemize}
        \item \heading{example} $\Z = \pa{1}$
        \item \heading{example} $D_{2n} = \pa{r,s}$
    \end{itemize}
\end{definition*}


\section{\linkbook{42}{Symmetric Groups}}

\begin{definition*}
    \bheading{Symmetric Group}
    \begin{enumerate}
        \item \bheading{symmetric group $S_{\Omega}$ on set $\Omega$} Let $\Omega$ be nonempty set, $S_{\Omega} = \pc{\sigma: \Omega \to \Omega \mid \sigma \text{ is a bijection}}$, the set of all permutations of $\Omega$. $(S_{\omega}, \circ)$ is the symmetric group on $\Omega$.
        \item \bheading{symmetric group of degree n} If $\Omega = \pc{1,2,\cdots,n}$, $S_n$ is the symmetric group of degree n
        \item \bheading{$\order{S_n} = n!$} (by counting number of possible permutations using the constraint that $\sigma$ is injective) 
        \item \bheading{cycle} a string of integers representing elements of $S_n$, which cyclically permutes them. $(a_1 \; a_2 \; \cdots \; a_m)$ is the permutation sending $a_i$ to $a_{i+1}$. $1\leq i \leq m-1$ and sends $a_m$ to $a_1$
        \item \bheading{length of cycle} is the number of integers which appear in it
        \item \bheading{$t$-cycle} is a cycle with length $t$
        \item \bheading{disjoint cycle} A cycle is disjoint if they have no numbers in common
        \item \bheading{$k$ cycles} Any $\sigma\in S_n$, we can represent $\sigma$ with $k$ cycles of the form 
        \[
            (a_1 \; a_2\; \cdots\; a_{m_1})(a_{m_1+1}\; a_{m_1+2}\;\cdots \; a_{m_2}) \cdots (a_{m_{k-1}+1}\; a_{m_{k-1}+2} \;\cdots \; a_{m_k})
        \]
        \item \bheading{cycle-decomposition of $\sigma$} is the product of $k$-cycles that representing $\sigma$
    \end{enumerate}
    \begin{itemize}
        \item \heading{convention} 1-cycle not written during cycle-decomposition. This convention ensures that cycle decomposition of $\tau\in S_n$ is exactly the same as cycle decomposition of permutation in $S_{m}$ where $m>n$, which acts as $\tau$ on $\pc{1,2,\cdots,n}$ and fixes elements in $\pc{n+1,n+2,\cdots,m}$
        \item \heading{computing inverse} Let $\sigma\in S_n$, cycle decomposition of $\sigma^{-1}$ can be obtained by writing numbers in each cycle of the cycle decomposition of $\sigma$ in reverse order
        \item \heading{computing product} by following elements in successive permutations
        \item \heading{example} $S_n$ is non-abelian for $n\geq 3$ (counterexample: $(12) \circ(13) = (1\;3\;2)$ but $(13)\circ(12) = (1\;2\;3)$)
        \item \bheading{proposition} disjoint cycle commutes
        \item \bheading{proposition} cycle-decomposition uniquely expresses a permutation as a product of disjoint cycles
        \item \bheading{proposition} The order of a permutation is the l.c.m. of the lengths of cycles in its cycle decomposition
    \end{itemize}
\end{definition*}


\section{\linkbook{47}{Matrix Groups}}

\begin{definition*}
    \bheading{Field and Matrix Group}
    \begin{enumerate}
        \item \bheading{field} A field is a set $F$ with two binary operations $+$ and $\cdot$ such that $(F,+)$ is an abelian group and $(F-\pc{0}, \cdot)$ is also an abelian group, and follows distributive law 
        \[
            a\cdot (b+c) = (a\cdot b) + (a\cdot c)    
        \]
        Denote $F^{\times} = F - \pc{0}$. 
        \item \bheading{general linear group} For each $n\in\Z^+$, let $GL_n(F)$ be the set of all $n\times n$ matrices whose entries come from $F$ and whose determinant is nonzero
        \[
            GL_n(F) = \pc{
                A \mid A \text{ is $n\times n$ matrix with entries from $F$ and $\det{A} \neq 0$}
            }
        \]
        with matrix multiplication as the binary operation. $GL_n(F)$ is a group under matrix multiplication, called \textbf{general linear group of degree n}: since its closed under matrix multiplication, and satisfies inverse/identity axioms
    \end{enumerate}
    \begin{itemize}
        \item \heading{example} $\Q,\R$, and $\BF_p = \integermodn[p]$ where $p$ is prime are fields
        \item \heading{fact} $GL_n(F)$ for $n\geq 2$ is nonabelian (matrix multiplication does not commute)
        \item \bheading{theorem} If $F$ is a field and $|F| < \infty$, then $\order{F} = p^m$ for some prime $p$ and integer $m$
        \item \bheading{theorem} If $\order{F} = q < \infty$, then $\order{GL_n(F)} = (q^n - 1)(q^n - q)(q^n-q^2) \cdots (q^n - q^{n-1})$ 
    \end{itemize}
\end{definition*}



\section{\linkbook{49}{Quaternion Group}}


\begin{definition*}
    \bheading{quaternion group} The quaternion group $Q_8$ is defined by 
    \[
        Q_8 = \pc{
            1, -1, i, -i, j, -j, k, -k
        }    
    \]
    with product $\cdot$ defined as 
    \begin{align*}
        &1 \cdot a = a \cdot 1 = a \quad \forall a\in Q_8 \\ 
        &(-1) \cdot (-1) = 1 \\ 
        &(-1) \cdot a = a \cdot (-1) = -a \quad \forall a\in Q_8 \\ 
        &i \cdot i = j \cdot j = k \cdot k = -1 \\
        &i\cdot j = k \quad j \cdot k = i \quad k \cdot i = j \\ 
        &j\cdot i = -k \quad k \cdot j = -i \quad i \cdot k = -j \\ 
    \end{align*}
    \begin{itemize}
        \item \heading{fact} $Q_8$ is non-abelian
        \item \heading{fact} order of elements in $Q_8$
        \begin{center}
            \begin{tabular}{c|c}
                element & order \\
                1 & 1 \\
                -1 & 2 \\ 
                i, -i, j, -j, k,-k & 4 \\
            \end{tabular}
        \end{center}
    \end{itemize}
\end{definition*}


\section{\linkbook{49}{Homomorphisms and Isomorphisms}}


\begin{definition*}
    \bheading{homomorphisms} Let $(G,\star)$ and $(H,\diamond)$ be groups. A map $\varphi:G\to H$ such that 
    \[
        \varphi(x\star y) = \varphi(x) \diamond \varphi(y)
        \quad \quad
        \forall x,y\in G    
    \]
    is called a \textbf{homomorphism}. Intuitively, $\varphi$ respects the group structures of its domain and codomain
    \begin{itemize}
        \item \bheading{theorem} If $\phi: G\to h$ is a homomorphism, then 
        \begin{enumerate}
            \item $\varphi(e_G) = e_H$
            \item $\varphi(x^{-1}) = \varphi(x)^{-1}$
            \item $\varphi(x^n) = \varphi(x)^n$ for all $n\in\Z$
        \end{enumerate}
    \end{itemize}
\end{definition*}

\begin{definition*}
    \bheading{Isomorphisms}
    \begin{enumerate}
        \item \bheading{isomorphisms} The map $\varphi:G\to H$ is called an \textbf{isomorphism} and $G$ and $H$ are said to be \textbf{isomorphic} or of the same \textbf{isomorphic type}, write $G\cong H$, if 
        \begin{enumerate}
            \item $\varphi$ is a homomorphism
            \item $\varphi$ is a bijection
        \end{enumerate}
        $G$ and $H$ are the same group, except that elements/operations are written differently.
        \item \bheading{isomorphism classes} Let $\sG$ be nonempty collection of groups. Then $\cong$ is an equivalence relation on $\sG$. the equivalence classes are called isomorphism classes
        \item \bheading{classification theorems} determine what properties of a structure specify its isomorphic types, i.e.
        \[
            \text{any \underline{non-abelian group of order 6} is isomorphic to $S_3$}    
        \]
        from which we know $D_6 \cong S_3$ and $GL_2(\BF_2) \cong S_3$
    \end{enumerate}
    \begin{itemize}
        \item \bheading{theorem} $G$ and $H$ share properties which rely on group structures (i.e. commutativity)
        \item \bheading{theorem} Isomorphic type of a symmetric group depends on cardinality only $S_{\triangle} \cong S_{\Omega} \iff \order{\triangle} = \order{\Omega}$
        \item \bheading{theorem} If $\varphi: G\to H$ is an isomorphism, then 
        \begin{enumerate}
            \item $\order{G} = \order{H}$
            \item $G$ is abelian iff $H$ is abelian
            \item for all $x\in G$, $\order{x} = \order{\varphi (x)}$
        \end{enumerate}
        \item \heading{examples}
        \begin{itemize}
            \item $G\cong G$ by the identity map or conjugation $g \mapsto xgx^{-1}$ for some $x\in G$
            \item $(\R,+) \cong (\R^+, \times)$ by the exponential map $exp: \R \to \R^+; x \mapsto e^x$
            \item $S_3 \cong D_6$ by the example classification theorem
            \item $GL_n(F) \cong F^{\times}$ by $\det: GL_n(F) \to F^{\times}$, i.e. $\det{AB} = \det{A}\det{B}$
            \item $S_3 \ncong \integermodn[6]$ ($S_3$ is non-abelian; $\integermodn[6]$ is abelian)
            \item $(\R,+) \ncong (\R^{\times}, \times)$ ($-1\in \R$ has order 2; $\R^{\times}$ has no element of order 2)
        \end{itemize}
    \end{itemize}
\end{definition*}


\begin{definition*}
    \bheading{homomorphism/isomorphism and presentations} Let $G$ be a finit group of order $n$ with a presentation. Let $S=\pc{s_1,\cdots,s_m}$ be the generators let $H$ be another group and $R=\pc{r_1,\cdots,r_m}$ be elements of $H$. If any relation satisfied in $G$ by $s_i$ is also satisfied in $H$ when each $s_i$ is replaced with $r_i$. Then there exists unique homomorphism $\varphi:G\to H; s_i \mapsto r_i$
    \begin{enumerate}
        \item if $H$ is generated by $\pc{r_1,\cdots,r_m}$, then $\varphi$ is surjective (any $r_{i_1} r_{i_2} \cdots \in H$ is $\varphi(s_{i_1} s_{i_2} \cdots)$)
        \item if in addition, $\order{H}=\order{G}$, then surjective $\varphi$ is necessarily injective, $\varphi$ is a bijection and $G\cong H$ 
    \end{enumerate}
    \heading{examples}
    \begin{itemize}
        \item Let $D_{2n} = \pc{r,s \mid r^n = s^2 = 1 \;\; sr = r^{-1}s}$. Let $D_{2k} = \pc{r_1,s_1 \mid r_1^n = s_1^2 = 1 \;\; s_1r_1 = r_1^{-1}s_1}$ where $k\mid n$, specifically $n=km$. Then 
        \[
            \varphi: D_{2n} \to D_{2k} 
            \quad \text{by} \quad
            \varphi(r) = r_1
            \quad
            \varphi(s) = s_1    
        \]
        is a homomorphism by the previous theorem as $r_1,s_1$ satisfies the relation of $D_{2n}$, specifically
        \[
            r_1^n = (r_1^k)^m = 1^m = 1    
        \] 
        Since $r_1,s_1$ generates $D_{2k}$, $\varphi$ is surjective. However for any $k<n$, $\order{D_{2n}} \neq \order{D_{2k}}$ so $D_{2n} \ncong D_{2k}$
        \item $D_6 \cong S_3$
        \begin{proof}
            Let $G=D_6$ and $H=S_3$. $a=(1\;2\;3), b=(1\;2)\in H$ satisfies $a^3=b^2=1$ and $ba=ab^{-1}$. Hence exists unique homomorphism $\varphi$ by $\varphi(r) \mapsto a$ and $\varphi(s) \mapsto b$. Note $a,b$ generates $S_3$. Therefore $\varphi$ surjective. Since $\order{D_6} = \order{S_3}$, $\varphi$ is an isomorphism 
        \end{proof}
    \end{itemize}
\end{definition*}

\begin{definition*}
    \bheading{automorphisms} Let $G$ be a group and define
    \[
        \aut{G} = \pc{
            \varphi: G\to G \mid \varphi \text{ is an isomorphism }
        }
    \]
    Then $(\aut{G}, \circ)$ is a group under function composition, called \textbf{automorphism group of $G$}. Any element of $\aut{G}$ is an \textbf{automorphism} of $G$
\end{definition*}



\section{\linkbook{54}{Group Actions}}


\begin{definition*}
    \bheading{Group Action}
    \begin{enumerate}
        \item \bheading{group action} A group action of a \underline{group} $G$ on a \underline{set} $A$ is a map satisfying
        \begin{align*}
            G\actson A: G\times A &\to A \\
                        (g,a)     &\mapsto g\cdot a
        \end{align*}
        \begin{enumerate}
            \item $g_1 \cdot (g_2 \cdot a) = (g_1 g_2) \cdot a$ for all $g_1,g_2\in G$ and $a\in A$ 
            \item $1\cdot a = a$ for all $a\in A$
        \end{enumerate}
        \item \bheading{permutation representation}
        For each fixed $g\in G$, define a map $\sigma_g$ by 
        \begin{align*}
            \sigma_g: A &\to A \\
            a&\mapsto g\cdot a
        \end{align*}
        \underline{is a permutation}, i.e. $\sigma_g \in S_A$. The permutation representation of $G\actson A$
        \begin{align*}
            \varphi: G &\to S_A \\
                     g &\mapsto \sigma_g
        \end{align*}
        \underline{is a homomorphism}.
        Intuitively, a group action of $G$ on a set $A$ means every element $g\in G$ acts as a permutation on $A$ in a manner consistent with the group operation in $G$.
        \item \bheading{faithful} If $G$ acts on $B$, then the action is faithful if
        \begin{itemize}
            \item distinct elements of $G$ induce distinct permutations of $B$
            \item permutation representation is injective
        \end{itemize}
        \item \bheading{kernel} The kernel of action of $G$ on $B$ is defined by to 
        \[
            \pc{
                g\in G \mid gb = b \text{ for all }b\in B
            }
        \]
        i.e. elements of $G$ that fix all elements of $B$
        \item \bheading{trivial action} $g\cdot a=a$ is the trivial action and the permutation representation $\varphi$ is the trivial homomorphism. $\ker{\varphi} = G$ and action not faithful when $\order{G}>1$
        \item \bheading{left regular action} $G\actson G$ by left multiplication (translation) $g \cdot a = ga$ ($g\cdot a = g + a$) for all $g,a\in G$. Then this action is called left regular action and is faithful by cancellation laws.
    \end{enumerate}
    \begin{itemize}
        \item \bheading{theorem} The actions of a group $G$ on a set $A$ and the homomorphisms from $G$ into $S_A$ are in bijective correspondence, i.e. the same thing. 
        \begin{proof}
            \textbf{($\rightarrow$)} from construction of permutation representation \textbf{($\leftarrow$)} Let $\varphi:G\to S_A$ be any homomorphism, then the map 
            \begin{align*}
                g: G\times A &\to A \\ 
                        (g,a)&\mapsto g\cdot a = \varphi(g)(a)
            \end{align*}
            satisfies properties of a group action $G\actson A$
        \end{proof}
        \item \heading{examples}
        \begin{itemize}
            \item $F^{\times} \actson V$ where $F^{\times}$ is a field and $V$ is a vector space. For example, action for $\R \actson \R^n$ specified by 
            \[
                \alpha \cdot (x_1, x_2,\cdots,x_n)
                = (\alpha x_1, \alpha x_2, \cdots, \alpha x_n)    
            \]
            \item $S_A \actson A$ by $\sigma\cdot a = \sigma(a)$. The permutation representation $\varphi: S_A\to S_A$ is the identity map
            \item $D_{2n} \actson \pc{1,2,\cdots,n}$ where $\pc{1,2,\cdots,n}$ is a labelling of vertices of a regular n-gon. The action is faithful or the associated permutation representation $\varphi:D_{2n}\to S_3$ is injective since distinct symmetries of a regular n-gon induce distinct permutations of the vertices. Since $\order{D_{2n}} = \order{S_3}$, $\varphi$ is surjective and so $\varphi$ is an isomorphism and $D_6 \cong S_3$. Geometrically, this means any permutation of vertices of a triangle is a symmetry; however this is not true for any n-gon with $n\geq 4$
        \end{itemize}
    \end{itemize}
\end{definition*}



\end{document} 
