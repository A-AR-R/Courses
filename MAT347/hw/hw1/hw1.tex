\documentclass[11pt]{article}
\input{\string~/.macros}
\newcommand{\oa}{\overline{a}}
\newcommand{\ob}{\overline{b}}
\newcommand{\oc}{\overline{c}}
\newcommand{\oi}{\overline{i}}

\newcommand{\integermodn}[1][n]{\Z/#1\Z}
\newcommand{\integermodnmul}[1][n]{(\Z/#1\Z)^{\times}}
\newcommand{\order}[1]{\left|#1\right|}
\newcommand{\modb}[1]{\left(mod \;\; #1 \right)}
\newcommand{\aut}[1]{Aut\left(#1\right)}
\newcommand{\actson}{\ensuremath{\curvearrowright}}

\newcommand{\heading}[1]{(#1)}
\newcommand{\bheading}[1]{\textbf{(#1)}}

% arg1=pdfurl arg2=pagenum arg3=text
\usepackage{url}
\usepackage{hyperref}
\hypersetup{colorlinks=true, linktoc=all, linkcolor=blue}
\newcommand{\linkbook}[3][../../abstract_algebra_dummit_and_foote.pdf]{
    \noindent\href[page=#2]{#1}{\urlstyle{rm}{#3}}
}

\usepackage[a4paper, total={6in, 8in}, margin=0.5in]{geometry}


\begin{document}

\begin{center}
    {\Huge HW1}
\end{center}


\section*{\linkbook{25}{0.3 12}} Let $n\in \Z$, $n>1$ and let $a\in\Z$ with $1\leq a \leq n$. Prove if $a$ and $n$ are not relatively prime then there is an interger $b$ such that $ab \equiv 0 \modb{n}$ and deduce that there cannot be an integer $c$ such that $ac \equiv 1 \modb{n}$


\section*{\linkbook{25}{0.3 13}} Let $n\in \Z$, $n>1$ and let $a\in\Z$ with $1\leq a \leq n$. Prove if $a$ and $n$ are relatively prime then there is an interger $c$ such that $ac \equiv 1 \modb{n}$ (use the fact that g.c.d. of two integers is a $\Z$-linear combination of the integers)

\begin{proof}
    By Euclidean algorithm, $\exists x,y\in\Z$ s.t. $ax + ny = \p{a, n} = 1$, so $1 - ax = yn$, hence $ax \equiv 1 \modb{n}$
\end{proof}


\section*{\linkbook{46}{1.3 15}} Prove that the order of an element in $S_n$ equals the least common multiple of the lengths of the cycles in its cycle decomposition

\begin{proof}
    Let $\sigma = \gamma_1 \gamma_2 \cdots \gamma_k$ be cycle decomposition to product of $k$ disjoint cycles. Suppose $\order{\sigma}=n$. Since disjoint cycles commutes,
    \[
        \sigma^n = \gamma_1^n \gamma_2^n \cdots \gamma_k^n = 1
        \quad \iff \quad
        \gamma_1^n = \gamma_2^n = \cdots = \gamma_k^n = 1
    \]
    Therefore $n$ is a common multiple of length of $\gamma_1,\cdots, \gamma_k$. By definition of order of $\sigma$, $n$ is the smallest common multiple, i.e. the l.c.m. of the length of the cycles
\end{proof}




\end{document}
