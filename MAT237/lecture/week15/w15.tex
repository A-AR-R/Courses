\documentclass[11pt]{article}
\input{/Users/markwang/.preamble}
\begin{document}



\section*{5.6 Surface Integrals}

\subsection*{Surface Area}

\begin{defn*}
  \textbf{Surface Area} Given a $C^1$ surface $S\in\R^3$, we approximate $S$ with infinitesimal elements parallelograms and assign it to an \textbf{element of area}
  \[
    dA = ||\,dx \times dy\,||
  \]
  Let $G:R\subseteq\R^2 \to \R^3$ be parameterization of $S$, and fix some $(u_0, v_0)\in\R^2$ and apply infinitesimal translation $du$ and $dv$ to $(u_0, v_0)$ to get corresponding vectors
  \[
    G(u, v + dv) - G(u, v) = \frac{\partial G}{\partial v}dv \quad \quad G(u + du, v) - G(u,v) = \frac{\partial G}{\partial u}du
  \]
  Note that magnitude of cross product of two vectors is the area of the parallelogram, so then
  \[
    dA = \left|\left| \frac{\partial G}{\partial u} \times \frac{\partial G}{\partial v} \right|\right| dudv = \left|\left| \, \vec{n} \, \right|\right|dudv
  \]
  where $\vec{n}$ is the normal to the tangent plane. The surface area of surface $S$ is given by just integrating over the area element
  \[
    A(S) = \iint_S dA = \iint_R \left|\left| \frac{\partial G}{\partial u} \times \frac{\partial G}{\partial v} \right|\right| dudv
  \]
  Let $G: (u,v)\mapsto (x,y,z)$ so that $\frac{\partial G}{\partial u} = (x_u, y_u, z_u)$ and $\frac{\partial G}{\partial v} = (x_v, y_v, z_v)$ so then
  \[
    \left|\left| \frac{\partial G}{\partial u} \times \frac{\partial G}{\partial v} \right|\right| = \left|\left| y_uz_v - z_uy_v, z_ux_v - x_uz_v, x_uy_v - y_ux_v  \right|\right|
  \]
  If we have instead given $S$ as graph of $C^1$ function we can re-parameterize $z = f(x,y)$ with $G(u,v) = (u,v,f(u,v))$ in which case
  \[
    \frac{\partial G}{\partial u} = (1, 0, f_u) \quad \frac{\partial G}{\partial v} = (0, 1, f_v) \quad
    \frac{\partial G}{\partial u} \times \frac{\partial G}{\partial v} = (-\frac{\partial f}{\partial u}, -\frac{\partial f}{\partial v}, 1)
  \]
  \[
    \left|\left| \frac{\partial G}{\partial u} \times \frac{\partial G}{\partial v} \right|\right| = \sqrt{ \left( \frac{\partial f}{\partial u} \right)^2 + \left( \frac{\partial f}{\partial v} \right)^2 + 1}
  \]
\end{defn*}

\subsection*{Surface Integrals over Vector Fields}

\begin{defn*}
  \textbf{Orientation of surface} An orientation of a surface $S$ is a consistent choice of normal vector to the surface. A paramterization determines an orientation of the surface
  \[
    \frac{\partial G}{\partial u} \times \frac{\partial G}{\partial v} = \hat{n}dA
  \]
  where $\hat{n}$ is a unit normal vector and indicates positive orientation. Hence we can reverse orientation by exchanging roles of $u$ and $v$. For $S$ which bounds a 3-manifold, $S$ has \textbf{Stoke's Orientation} if the normal vector of $S$ points outwards with respect to the space it bounds. Although not all surface has orientation, such as the Mobius strip.
\end{defn*}


\begin{defn*}
  \textbf{Surface Integral} Given a vector field $F:\R^3 \to \R^3$ and a surface $S$, we want to compute the flux of the vector field through the surface, where flux represents the amount of force/fluid passing through $S$. The vector field travelilng in direction $\hat{n}$ is given by $F \cdot \hat{n}$ and so the surface integral is given by
  \[
    \iint_S F\cdot \hat{n} dA
  \]
  where $F\cdot \hat{n}$ is the vector field projected onto the normal of the surface. Given parameterization $G:R\subseteq\R^2 \to \R^3$ then
  \[
    flux = \iint_S F\cdot \hat{n} dA = \iint_R F(G(u,v))\cdot \left[ \frac{\partial G}{\partial u}\times \frac{\partial G}{\partial v} \right] dudv
  \]
  note that
  \[
    \hat{n}dA = \frac{\vec{n}}{||\vec{n}||} ||\vec{n}||dudv = \left[ \frac{\partial G}{\partial u} \times \frac{\partial G}{\partial v}\right] dudv
  \]
\end{defn*}


\section*{5.7 Divergence Theorem}

\begin{theorem*}
  \textbf{Divergence Theorem} The outward flux of a vector field through a closed surface is equal to the volume integral of the divergence over the region inside the surface. Let $R\subseteq \R^3$ be a regular region with piecewise smooth boundary $\partial R$. If $F: \R^3 \to \R^3$ is a $C^1$ vector field and $\partial R$ is positively oriented with respect to $R$ then
  \[
    \iint_{\partial R} F \cdot \hat{n} dA = \iiint _R div F dV
  \]
  \begin{rem}
    A result that relates the flux of vector field through a surface to the behavior of the vector field inside the surface.
  \end{rem}
\end{theorem*}


\begin{theorem*}
  \textbf{Stokes' Theorem} Let $S$ be a smooth surface with piecewise smooth (geometric, not topological) boundary $\partial S$, endowed with Stokes' Orientation. If $F:\R^3 \to\R^3$ is a $C^1$ vector field in a neighborhood of $S$, then
  \[
    \int_{\partial S} F\cdot dx = \iint_S (curl F)\cdot \hat{n}dA
  \]
  Note $\partial S$ has stoke's orientation if $n\times t$ points into $S$, where $n$ is the orientation of $S$ and $t$ is the tangent vector of a parameterization of $\partial S$. Or that $t$ points counterclockwise when the surface normal $n$ points toward the viewer.

  \begin{enumerate}
    \item If $S$ is just a region (surface) in xy-plane, then $\hat{n} = (0,0,1)$ and so
    \[
      \int_{\partial S} F\cdot dx = \iint_S (curl F)\cdot \hat{n}dA = \iint_S \left[ \frac{\partial F_2}{\partial x_1} - \frac{\partial F_1}{\partial x_2} \right]dA
    \]
    Hence, Stokes' Theorem in the xy-plane is just Green's theorem
    \item If $\partial S = \emptyset$ ($S$ is closed surface) then
    \[
      \iint_S \nabla \times F \cdot dx = 0
    \]
    \item If $S_1$ and $S_2$ are 2 surfaces with a common boundary $C$, where $C$ is oriented in such a way that $S_1$ and $S_2$ has Stoke's Orientatino then
    \[
      \iint_{S_1} \nabla\times F \cdot dx = \iint_{S_2} \nabla\times F \cdot dx
    \]
  \end{enumerate}


  \end{rem}
\end{theorem*}



\end{document}
