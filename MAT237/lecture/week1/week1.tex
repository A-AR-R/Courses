\documentclass[11pt]{article}
\input{/Users/markwang/.preamble}
\begin{document}


\section{Set and Notation}


\begin{defn}
  \label{union}
  Let S be a set and choose 2 sets $A, B \subseteq S$. The \textbf{union} of A and B is
  \[
    A\cup B = \left\{ x\in S | \quad x\in A \lor x\in B \right\}
  \]
  \[
    \bigcup_{i\in \I} A_i = \left\{ x\in S | \quad \exists i\in I, x\in A_i\right\}
  \]
\end{defn}

\begin{defn}
  \label{intersection}
  Let S be a set and choose 2 sets $A, B \subseteq S$. The \textbf{intersection} of A and B is
  \[
    A\cap B = \left\{ x\in S| \quad x\in A \land x\in B \right\}
  \]
  \[
    \bigcap_{i\in\I} A_i = \left\{ x\in S | \quad \forall i \in I, x\in A_i \right\}
  \]
\end{defn}

\begin{defn}
  \label{setcomplement}
  If $A\subseteq S$, then the \textbf{complement} of A with respect to S is all elements which are not in A, that is
  \[
    A^c = \{ x\in S: x\not\in A \}
  \]
\end{defn}

\begin{defn}
  \label{function}
  Given 2 sets A, B, a \textbf{function} $f: A\to B$ is a map which assigns to every point in A a unique point of B, that is
  \[
    f: a \mapsto f(a), \text{ where } a\in A, f(a)\in B
  \]
\end{defn}

\begin{defn}
  \label{image and preimage}
  Let $f: A\to B$ be a function.
  \begin{enumerate}
    \item If $U \subseteq A$, then we define the \textbf{image} of $U$ to be
    \[
      f(U) = \left\{ y\in B: \exists x\in U, f(x) = y\right\} = \left\{ f(x): x\in U\right\}
    \]

    \item If $V\subseteq B$ we define the \textbf{pre-image} of $V$ to be
    \[
      f^{-1}(V) = \left\{ x\in A: f(x) \in V\right\}
    \]
    \begin{rem}
      Note $U,V$ are sets, not variable.
    \end{rem}
  \end{enumerate}
\end{defn}

\begin{defn}
  \label{jectivity}
  Let $f: A \to B$ be a function. We say that
  \begin{enumerate}
    \item f is \textbf{injective} if whenever $f(x) = f(y)$ then $x=y$
    \item f is \textbf{surjective} if for every $y\in B$ there exists $x\in A$ such that $f(x) = y$
    \item f is \textbf{bijective} if f is both injective and surjective
  \end{enumerate}

  \begin{rem}
    Testing injectivity by using the horizontal line test in $\R^2$: An injective function is one whose graph that never intersect any horizontal line twice. Test surjectivity by ensuring that every horizontal line in the domain is crossed at least once by the graph.
  \end{rem}
\end{defn}


\end{document}
