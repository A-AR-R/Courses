\documentclass[11pt]{article}
\input{/Users/markwang/.preamble}
\begin{document}


\section{Open, closed and everything in between}

\begin{defn}
  \label{open ball}
  Let $x\in \R^n$ and $r > 0$ a real number. We define the \textbf{open ball} of radius $r$ at the point $x$ as

  \[
    B_r(x) := \{ y \in \R^n: d(x, y) < r\} = \{ y \in \R^n: \| x-y\| < r\}
  \]
  \begin{rem}
    Here $d(x,y)$ means distance, i.e. $d(x,y) = \| x-y \|$. The ball is therefore a collection (set) of points which are a distance at most $r$ from $x$. In $\R^1$, $B_r(x) = \{ y \in \R^n: | x-y| < r\} = (x-r, x+r)$. In $\R^2$, $B_r(x) = \{ (x,y) \in \R^n: x^2 + y^2 < r^2\}$
  \end{rem}
\end{defn}


\begin{defn}
  \label{boundary of open ball}
  The \textbf{boundary of an open ball} is,
  \[
    \partial B = \{ y\in \mathbb{R}^n: d(y, x) = r\}
  \]
\end{defn}

\begin{defn}
  \label{closed ball}
  The \textbf{closed ball}, denoted $\overline{B}$, is defined as

  \begin{align*}
    \overline{B} &= B\cup \partial B \\
    &= \{ y\in \mathbb{R}^n: d(y, x) \leq r\}
  \end{align*}
\end{defn}


\begin{defn}
  \label{bounded set}
  A set $S\subseteq \R^n$ is \textbf{bounded} if there exitsts a large enough ball $B\subseteq \R^n$ such that $\exists r > 0, S\subseteq B_r(0)$
\end{defn}


\begin{defn}
  \label{interior and boundary point}
  Let $S\subseteq \R^n$,
  \begin{enumerate}
    \item We say that  $x\in\R^n$ is an \textbf{interior point} of $S$ if $\exists$ an $r>0$ such that $B_r(x)\subseteq S$; that is, $x$ is an interior point if we can enclose it in an open ball which is completely contained in $S$
    \label{interior point}
    \item We say that  $x\in\R^n$ is a \textbf{boundary point} of $S$ if for every $r>0$, $B_r(x)\cap S\neq \emptyset$ and $B_r(x)\cap S^c \neq \emptyset$; that is, $x$ is a boundary point if no matter how small the ball we place around $x$, that ball \textit{meets} or \textit{lives} both inside and outside of $S$.
    \label{boundary point}
  \end{enumerate}

\end{defn}


\begin{defn}
  $ $\\
  \label{interior and boundary of set}
  If $S\subseteq \R^n$
  \begin{enumerate}
    \item The \textbf{interior} of $S$, denoted $intS$, is the set of all interior points of $S$ \\
    \item The \textbf{boundary} of $S$, denoted $\partial S$, is the set of all boundary points of $S$
  \end{enumerate}

  There are some properties,
  \begin{align}
    & \text{If } x\not\in \partial S \text{ , then either } x\in intS \lor x\in intS^c \\
    & \partial (S) = \partial (S^c) \tag{by \hyperref[interior and boundary point]{definition\ref{interior and boundary of set}}} \\
    & x = intS \sqcup intS^c \sqcup \partial S \tag{by \# 1} \\
    & \partial S \cap intS = \emptyset \tag{by \hyperref[interior and boundary point]{definition\ref{interior and boundary of set}}}\\
    & S\subseteq \partial S \sqcup intS \tag{by \# 1} \\
    & intS \text{ is an open set } \tag{by \hyperref[open and closed set]{definition\ref{open and closed set}}}
  \end{align}
  \begin{note}
    $A\sqcup B$ means $A\cup B \land A\cap B = \emptyset$
  \end{note}

  \begin{rem}
    As an example, $(-1, 1]$, has interior points $(-1, 1)$ and boundary points $1, -1$
  \end{rem}
\end{defn}



\begin{defn}
  \label{open and closed set}
  $ $ \\
  \begin{enumerate}
    \item A set $S\subseteq \R^n$ is said to be \textbf{open} if every point of $S$ is an interior point; that is, $S$ is open if for every $x\in S$, there exists an $r > 0$ such that $B_r(x)\subseteq S$.
    \[
      \forall x\in S, \exists r>0, B_r(x)\subseteq S
    \]
    \item The set $S$ is \textbf{closed} if $S^c$ is open.

    \begin{rem}
      The contrapositive is also true: If set $S$ is not closed then $S^c$ is not open.
    \end{rem}
    \begin{proposition}
      A set $S\subseteq \R^n$ is closed if and only if $\partial S\subseteq S$
    \end{proposition}
  \end{enumerate}
  \begin{rem}
    $ $\\
    An open ball is open. \\
    A closed ball $B$ is closed, which is equivalent to proving $B^c$ is open, or that every point in $B^c$ is an interior point. \\
    The entire space and $\emptyset$ are both open and closed. \\
    If $U$ is open, then $U^c$ is closed
  \end{rem}
\end{defn}



\begin{rem}
  To solve that a set is open, we prove all points in the set are interior points or there exists a ball centered at every point that contains completely in the set. We set an arbitrary element within a set and construct a ball centered at this point with radius satisfying the contraints of the set, so that the ball is contained in the set. Then we choose an arbitrary point within the ball and prove that it always become contained in the set. In this process, we can use the fact that the metric of this arbitrary point and the center of the open ball to be less than the radius chosen. Triangular inequality becomes handy here.
\end{rem}


\begin{defn}
  Let $S\subseteq \R^n$. The closure of the set $S$ is the set $\overline S = S\cup \partial S$ Closure follows properties,
  \begin{enumerate}
    \item The closure is always a closed set
    \item $S$ is closed if and only if $S = \overline S$
    \item $S$ is the smallest closed set containing $S$
  \end{enumerate}
  \begin{rem}
    The closure of open interval is closed interval. The closure of the open ball is
    \[
      B_r(x)=\{y\in \R^n :\| x-y\|\leq r\}
    \]
    Closure simply means that the entirety of the boundary is delegated to any set $S$. Therefore $S$ is always closed by \hyperref[open and closed set]{proposition in~\ref{open and closed set}}
  \end{rem}
\end{defn}

\end{document}
