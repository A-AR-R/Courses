\documentclass[11pt]{article}
\input{/Users/markwang/.preamble}
\begin{document}


\section*{5 Vector Field}

\subsection*{Vector Derivatives}


\begin{defn*}
  \textbf{Vector Field} is aa function $F: \R^2 \to \R^2$ which prescribes to every point $x\in \R^n$ an arrow, $F(x)$.
\end{defn*}

\begin{defn*}
  The nabla operator is
  \[
    \nabla = \left(\frac{\partial}{\partial x_1}, \frac{\partial}{\partial x_2}, \cdots, \frac{\partial}{\partial x_n}\right)
  \]
  Then
  \begin{enumerate}
    \item \textbf{Gradient} Let $f:\R^n\to\R$ be $C^1$ function. The gradient of $f$ is
    \[
      grad\, f = \nabla f = \left(\frac{\partial f}{\partial x_1}, \frac{\partial f}{\partial x_2}, \cdots, \frac{\partial f}{\partial x_n}\right)
    \]
    which measures how quickly $f$ is changing in each of the given coordinate axes. And $\nabla f$ gives the direction of steepest ascent.
    \item \textbf{Divergence} If $F:\R^n \to \R^n$ is a $C^1$-vector field, then the divergence of $F$ is
    \[
      div\, F = \nabla \cdot F = \frac{\partial F_1}{\partial x_1} + \cdots + \frac{\partial F_n}{\partial x_n}
    \]
    which measures the infinitesimal flux (how quickly field is spreading out) of the vector field, i.e. the amount of the field which is passing through an infinitesimal surface area. If $div\, F = 0$, we say that $F$ is incompressible.
    \item \textbf{Curl} If $F:\R^3 \to \R^3$ is a $C^1$ vector field in $\R^3$ then the curl of $F$ is
    \[
      curl\, F = \nabla \times F = \left( \frac{\partial F_3}{\partial x_2} - \frac{\partial F_2}{\partial x_3}, \frac{\partial F_1}{\partial x_3} - \frac{\partial F_3}{\partial x_1}, \frac{\partial F_2}{\partial x_1} - \frac{\partial F_1}{\partial x_2} \right)
    \]
    which measures the infinitesimal circulation (how quickly field is spinning around) of the vector field. Note it is only valid in $\R^3$. If $grad\, F = 0$, we say that $F$ is irrotational.
    \item \textbf{Laplacian} If $f:\R^m \to \R$ is a $C^1$ function, then the Laplacian of $f$ is
    \[
      \nabla^2 f = \nabla \cdot \nabla f = \triangle f = \frac{\partial^2 f}{\partial x_1^2} + \cdots + \frac{\partial^2 f}{\partial x_n^2}
    \]
    which measures the infinitesimal rate of change of the function $f$ in outward rays along spheres. Laplacian of a $f$ is the divergence of the gradient of $f$.
  \end{enumerate}
\end{defn*}

\begin{defn*}
  \textbf{Properties} Let $f, g: \R^n \to \R$ and $F, G" \R^n \to \R^n$ all be $C^1$. then the following property holds
  \begin{enumerate}
    \item $\nabla (fg) = f\nabla g + g \nabla f$
    \item $\nabla (f\cdot g) = (F\cdot \nabla)G + F\times (\nabla \times G) + (G\cdot \nabla)F + G\times (\nabla \times F) $
    \item $\nabla \times (fG) = f(\nabla \times G) + (\nabla f)\times G $
    \item $curl \enspace (grad \enspace f) = 0$ or $\nabla \times (\nabla f) = 0$
    \item $div \enspace (curl \enspace F) = 0$ or $\nabla \cdot (\nabla \times F) = 0$
    \item \textbf{de rham cohomology}
    \[
      f \xrightarrow{gradient} F \xrightarrow{curl} F \xrightarrow{divergence} f
    \]
  \end{enumerate}

\end{defn*}


\subsection*{Arc Length}

\begin{defn*}
  \textbf{Arc Length} Given a $C^1$ curve $C \in \R^n$, we approximate $C$ with infinitesimal straight line components and assign it to an \textbf{element of arc}
  \[
    ds = ||dx|| = \sqrt{dx_1^2 + \cdots + dx_n^2}
  \]
  To facilitate computation, we introduce a parameterization of $C$ with $x =  g:[a,b] \to \R^n$ such that
  \[
    dx = g'(t)dt = \left( \frac{dg_1}{dt},\cdots, \frac{dg_2}{dt}\right)dt
  \]
  Then
  \[
    ds = ||dx|| = \sqrt{ \left( \frac{dg_1}{dt} \right)^2,\cdots, \left(\frac{dg_2}{dt} \right)^2 }dt
  \]
  By integrating from $a$ to $b$
  \[
    Arclength(C) = \int_C ds = \int_a^b ||g'(t)||dt = \int_a^b \sqrt{ \left( \frac{dg_1}{dt} \right)^2,\cdots, \left(\frac{dg_2}{dt} \right)^2 }dt
  \]
  Here we can interpret $g'(t)$ as velocity and $|g'(t)|$ as speed. By integrating speed over time, we det distance travelled.
  \begin{rem}
    The arc length formula computes total distance travelled, not necessarily the graph of the curve. So have to restrict domain of parameterization to find proper distance.
  \end{rem}
\end{defn*}



\begin{proposition*}
  \textbf{Invariant of arc length under re-parameterization} If $g:[a,b]\to \R^n$ is a $C^1$ function and $\phi:[c,d]\to [a,b]$ is a re-parameterization of $g(t)$ so that $g\circ \phi: [c,d]\to\R^n$ then
  \[
    \int_{[c,d]} \left| \frac{d}{dt} (g\circ \phi)(t) \right| dt = \int_{[a,b]} |g'(t)|dt
  \]
  \begin{proof} By change of variable and chain rule we have
    \begin{align*}
      \int_{[a,b]} ||g'(t)||dt &= \int_{\phi^{-1}([a,b])} ||g'(\phi(t))||\,|\det D\phi(t)|dt\\
      &=\int_{[c,d]} ||g'(\phi(t))||\, |\phi'(t)|dt\\
      &= \int_{[c,d]} || \frac{d}{dt} (g\circ \phi)(t)||dt
    \end{align*}
  \end{proof}
\end{proposition*}


\subsection*{Line Integrals}

\begin{defn*}
  \textbf{Line Integral} A line integral is an integral where the function to be integrated is evaluated along a curve. The value of the line integral is the sum of values of the field at all points on the curve, weighted by some scalar function on the curve.
  The function to be integrated may be a scalar field or a vector field.
\end{defn*}



\begin{enumerate}
  \item \textbf{Scalar Field} Let $f:U\subseteq \R^n \to \R$ be continuous function and $C\subseteq U$ a smooth curve. Let $g: [a,b] \to \R^n$ be a parameterization of $C$. The line integral of a scalar function is given by
  \[
    \int_C f ds = \int_{[a,b]} f(g(t)) | g'(t) | dt
  \]
  Note that $\int_{U} f dA$ is the total volume between graph of $f(x)$ and the $\R^n$ plane; whereas $\int_C f ds$ represents the area which lies between the curve $C$ and the graph of $f(x)$.
  \item \textbf{Vector Field} Let $F: \R^n \to \R^n$ be a vector field, and $C\subseteq \R^n$ be some smooth curve. Parameterize this curve by $g:[a,b] \to\R^n$. We think of vector field $F$ acting on curve at each point $t_0 \in [a,b]$ and the line integral is hence the total amount of work done by vector field on the curve, i.e. $F\cdot dx$. (note that $x = g(t)$ then $dx = g'(t)dt$)
  \[
    \int_C F \cdot dx = \int_C (F_1dx_1 + \cdots + F_ndx_n) = \int_a^b (F_1\frac{dg_1}{dt}dt + \cdots + F_n\frac{dg_n}{dt}dt) = \int_a^b F(g(t))\cdot g'(t)dt
  \]
  whose magnitude is invariant under re-parameterization but its sign is subject to change. Element of curve is given by $ds = | g'(t) | dt$ we have
  \[
    \int_a^b F(g(t)) \cdot \frac{g'(t)}{|g'(t)|}|g'(t)|dt = \int_a^b F(g(t)) \cdot \hat{T}(t)ds
  \]
  where $\hat{T}(t) = \frac{g'(t)}{|g'(t)|}$ is the unit speed vector. Note that $F(g(t)) \cdot \hat{T}(t)$ is the projection of $F$ in the direction of $\hat{T}$, which is precisely the component of $F$ doing work in direction of $\hat{T}$
  \begin{rem}
    Parameterization of curve has to be $C^1$ and regular (i.e. $\gamma'(t) \neq 0$ for all $t\in [a,b]$)
  \end{rem}
\end{enumerate}

\end{document}
