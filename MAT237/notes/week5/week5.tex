\documentclass[11pt]{article}
\input{"../../preamble"}
\begin{document}


\section{Compactness}

\begin{defn}
  \label{least upper bound}
  Let $S$ be a non-empty set of real numbers.
  \begin{enumerate}
    \item A real number $x$ is called an upper bound for $S$ if $x \geq s$ for all $s \in S$.
    \item A real number $x$ is the \textbf{least upper bound} (or supremum) for $S$ if $x$ is an upper bound for $S$ and $x \leq y$ for every upper bound $y$ of $S$.
  \end{enumerate}

  \begin{theorem}
    \label{complete axioms}
     By completeness of real numbers, every non-empty set $A\in \R$ which is bounded above has a least upper bound, i.e. $sup(A)$ exists.
  \end{theorem}

  \begin{defn}
    A closed and bounded set $A \in\R$, $inf(A)\in A \land sup(A)\in A$.
    \begin{proof}
      $ $\\
      Assume the set didn't contain its supremum (which exists since the set is bounded). Since the set is closed, an entire neighbourhood of the supremum lies outside the set. This neighbourhood contains upper bounds of the set that are lower than the supremum, which is a contradiction.
    \end{proof}
  \end{defn}
  \begin{note}
    While the maximum of a set must be contained in the set, the supremum does not necessarily have to be contained in the set
  \end{note}
\end{defn}


\begin{defn}
  \label{compactness}
  A set $S\subseteq \R^n$ is \textbf{compact} if it is both closed and bounded.

  \begin{theorem}
    A finite union of compact sets is compact because a finite union of closed and bounded sets is closed and bounded; however an infinite union of compact is not necessarily compact. \\
    For example, $S_n = [\frac{1}{n}, 1]$ and $U=\bigcup_i^{\infty}{S_i}$. $x = 0\subseteq U^c$, whose neighborhood meets $U$ and $U^c$. Since $x$ a boundary point of $U$ but not in $U$, then $U$ not closed.
  \end{theorem}
\end{defn}

\begin{theorem}
  \label{equivalent definition of compactness}
  The following are equivalent,

    \begin{enumerate}
      \item $S$ is compact
      \item \textbf{Bolzano-Weierstrass} Every sequence in $S$ has a convergent subsequence, that is, if $(X_n)_{n=1}^{\infty}\subseteq S$ (note $(X_n)$ may not be convergent), then there exists a subsequence $X_{n_k}$ and a point $x\in S$ such that $X_{n_k} \rightarrow x$
      \item \textbf{Heine-Borel} Every open cover of $S$ admits a finite subcover; that is, if $\{\cup_i\}_{i\in\I}$ is a collection of open sets such that $S\subseteq \{\cup_i\}_{i\in\I}{U_i}$, then there exists a finite subset $J\subseteq \I$ such that $S\subseteq \{\cup_i\}_{i\in J}{U_i}$
    \end{enumerate}

\end{theorem}


\begin{theorem}
  \label{continuous preimage of compact sets is compact}
  Let $f: \R^n \rightarrow \R^m$ be a continuous function. If $K\subseteq \R^n$, then $f(K)$ is also compact.
\end{theorem}


\begin{theorem}
  \label{extreme value theorem}
  \textbf{Extreme Value Theorem}
  If $f: \R^n \rightarrow \R$ is a continuous function and $K\subseteq \R^n$ is a compact set, then there exists $x_{min}, x_{max} \in K$ such that for every $x\in K$, $f(x_{min}) \leq f(x)\leq f(x_{max})$; that is, $f$ achieves both its extreme values on $K$.

  \begin{rem}
    $x_{min}, x_{max}$ need not be minimum and maximum in $K$.
  \end{rem}
\end{theorem}


\section*{Connectedness}


\begin{defn}
  \label{disconnected}
  A set $S\subseteq \R^n$ is said to be \textbf{disconnected} if there exist non-empty $S_1, S_2\subseteq S$ such that
  \begin{enumerate}
    \item $S_1\not= \emptyset \land S_2\not= \emptyset$
    \item $S = S_1 \cup S_2$
    \item $\overline{S_1}\cap S_2 = \emptyset$ and $S_1 \cap \overline{S_2} = \emptyset$
  \end{enumerate}
  \label{connectedness}
  Here $(S_1, S_2)$ is a disconnection of $S$. If $S$ admits no disconnection, we say that $S$ is \textbf{connected}.
\end{defn}

\begin{theorem}
  \label{intervals are connected in R}
  A set $S\subseteq \R$ is connected if and only if $S$ is an interval
\end{theorem}

\begin{defn}
  \label{path-connectedness}
  If $S\subseteq \R^n$ then a \textbf{path} in $S$ is any continuous map $\gamma: \R \rightarrow \R^m: [0,1]\rightarrow S$. We say that $S$ is \textbf{path-connected} if for every two points $a,b\in S$ there exists a path $\gamma: [0,1]\rightarrow S$ such that $\gamma(0) = a$ and $\gamma(1) = b$

  \begin{rem}
    path-connected is a weaker statement than connectedness.
  \end{rem}
\end{defn}

\begin{theorem}
  \label{continuous image of connected set is connected}
  \textbf{The continuous image of a (path) connected set is (path) connected.} More precisely, If $f: \R^n \rightarrow \R^m$ is continuous and $S\subseteq \R^n$ is (path) connected, then $f(S)$ is (path) connected.
\end{theorem}

\begin{theorem}
  \label{intemediate value theorem}
  \textbf{Intermediate Value Theorem} Let $V\subseteq \R^n$ be a (path) connected set and $f: \R^n \rightarrow \R$ be a continuous function. Let $a,b\in V$ and assume that $f(a) < f(b)$. Then for every $c\in \R$, such that $f(a) < c < f(b)$ there exists an $x\in V$ such that $f(x) = c$
\end{theorem}


\begin{theorem}
  \label{disconnection from a set to subset}
  \textbf{disconnection from a set to subset}\\
  Let $(S_1, S_2)$ be disconnection for $S$. Let $U\subseteq S$ and $U_1 = S_1\cap U$, $U_2 = S_2\cap U$. If $U$ meets both $S_1$ and $S_2$, i.e., $U_1 \not=\emptyset$ and $U_2 \not=\emptyset$, then $(U_1, U_2)$ is a disconnection for $U$

  \begin{corollary}
    If a set $S$ is disconnected $(S_1, S_2)$ and a subset $U\subseteq S$ is connected. Then $U$ must be contained in either $S_1$ or $S_2$, i.e. $U\subseteq S_1$ or $U\subseteq S_2$
  \end{corollary}

  \begin{corollary}
    Let $A$ be a connected set, suppose $B$ satisfies $A\subseteq B\subseteq \overline{B}$, the $B$ is connected.
  \end{corollary}
\end{theorem}

\begin{theorem}
  \label{Any set which is path connected is also connected}
  \textbf{Any path connected set is also connected}
  \begin{proof}
    $ $\\
    Proof by contradiction, assume $S$ is not connected. Then there exists a disconnection $(S_1, S_2)$.  Consider $a\in S_1, b\in S_2$. Since $S$ is path-connected, there exists $\gamma: [0,1]\rightarrow S$, such that $\gamma[0] = a$, $\gamma[1] = b$. Let $U= \gamma([0,1])\subseteq S$, then $U$ is path-connected since $\gamma$ is continuous and $[0,1]$ an interval in $\R$ is path-connected. Now introduce contradiction by showing $U$ is not connected. Let $U_1 = U\cap S_1$, $U_2 = U\cap S_2$.
    \[
      U_1\cup U_2 = (U\cap S_1)\cup (U\cap S_2) = U\cap (S_1\cup S_2) = U
    \]
    and
    \[
      \overline{U_1}\cap U_2 \subseteq (\overline{U}\cap \overline{S_1})\cap (U\cap S_2) = \overline{S_1}\cap S_2 = \emptyset
    \]
    Similarly $U_1\cap \overline{U_2}=\emptyset$. Then shown $(U_1, U_2)$ is a disconnection of $U$, which is a contradiction.
  \end{proof}
\end{theorem}


\begin{theorem}
  \label{connected set not necessarily path-connected}
  \textbf{If $S\subseteq \R^n$ is connected and open, then $S$ is path-connected.}
\end{theorem}



\begin{center}
  \begin{tabular}{ c c c }
                           & closed   & compact \\
     finite union          & yes      & yes  \\
     infinite union        & no       & no  \\
     finite intersection   & yes      & yes \\
     infinite intersection & yes      & yes \\
  \end{tabular}
\end{center}

\end{document}
