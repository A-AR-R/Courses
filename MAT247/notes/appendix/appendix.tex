\documentclass[11pt]{article}
\input{/Users/markwang/.preamble}


\begin{document}


% arg1=pdfurl arg2=pagenum arg3=sectiontitle
\newcommand{\linksection}[3][../../linear_algebra_friedberg_insel_spence_4ed.pdf]{
    \subsection*{\href[page=#2]{#1}{#3}}
}

\linksection{561}{Appendix A Sets}



\begin{defn*}
    Set definitions
    \begin{enumerate}
        \item \textbf{Set} is a collection of objects, called elements of the set.
        \item \textbf{Subset} $B\subseteq A$ if every element of $B$ is an element of $A$
        \item \textbf{Proper Subset} $B$ is a proper subset of $A$ if $B\subseteq A$ and $B\neq A$
        \item \textbf{Equality} Two sets are equal, $A=B$, if and only if $A\subseteq B$ and $B\subseteq A$
        \item \textbf{Empty Set} $\emptyset$ is a subset of every set. 
        \item \textbf{Union, Intersection} 
        \begin{align*}
            & A\cup B = \{x : x \in A \text{ or } x \in B \}
            \quad \quad  A\cap B = \{x : x \in A \text{ and } x \in B \} \\
            & \bigcup_{i=1}^n A_i = \{ x: x\in A_i \text{ for some } i = 1,2, \cdots, n \}
            \quad \quad  \bigcap_{i=1}^n A_i = \{ x: x\in A_i \text{ for all } i = 1,2, \cdots, n \} \\
            & \bigcup_{\alpha \in  \Lambda } = \{ x : x\in A_{\alpha} \text{ for some } \alpha \in \Lambda \}
            \quad \quad  \bigcap_{\alpha \in  \Lambda } = \{ x : x\in A_{\alpha} \text{ for all } \alpha \in \Lambda \} \\ 
        \end{align*}
        where $\Lambda$ is an index set and $\{A_{\alpha}: \alpha\in \Lambda \}$ is a collection of sets.
        \item \textbf{Disjoint} Two sets are disjoint if their intersection equals the empty set $A \cap B = \emptyset$
        \item \textbf{Relation} A relation on $A$ is a set $S$ of ordered pairs of elements of $A$ such that $(x,y)\in S$ if and only if $x$ stands in the given relationsihp to $y$. For example, is equal to, is less than, .. are relations. If $S$ is a relation on a set $A$, we write $x\sim y$ in place of $(x,y)\in S$
        \item \textbf{Equivalence Relation} A relation $S$ on a set $A$ is an equivalence relation on $A$ if the 3 condition holds 
        \begin{enumerate}
            \item For all $x\in A$, $x\sim x$ (reflexivity)
            \item If $x\sim y$, then $y\sim x$ (symmetry)
            \item If $x\sim y$ and $y\sim z$, then $x\sim z$ (transitivity)
        \end{enumerate}
        If we define $x\sim y$ to be $x-y$ divisible by a fixed integer $n$, then $\sim$ is an equivalence relation on the set of integers.
    \end{enumerate}
\end{defn*}




\linksection{563}{Appendix B Functions}

\begin{defn*}
    Functions 
    \begin{enumerate}
        \item \textbf{Function} $A$, $B$ are sets, a function $f$ from $A$ to $B$, $f: A\to B$ is a rule that associates each element $x\in A$ a unique element denoted by $f(x)$ in $B$. 
        \item \textbf{Image and Preimage} The element $f(x)$ is the image of $x$ under $f$; $x$ is the preimage of $f(x)$ under $f$. 
        \begin{enumerate}
            \item If $S\subseteq A$, then denote by $f(S)$ the set $\{ f(x): x \in S\}$ of all images of elements of $S$. \item Likewise, denote by $f^{-1}(T)$ the set $\{x\in A: f(x)\in T \}$ of all preimages of elements in $T$. 
            \item Preimage of an element in the range need not be unique
        \end{enumerate}
        \item \textbf{Domain and Codomain} If $f: A\to B$, then $A$ is called the domain of $f$ and $B$ is called the codomain of $f$. 
        \item \textbf{Range} The set $\{ f(x): x\in A\}$ is called the range of $f$. Note the range of $f$ is a subset of $B$
        \item \textbf{Function Equality} Two functions $f:A\to B$ and $g: A\to B$ are equal, $f=g$, if $f(x) = g(x)$ for all $x\in A$
        \item \textbf{One-to-one} Functions such that each element of the range has a unique preimage are one-to-one; that is $f:A\to B$ is one-to-one if $f(x) = f(y)$ implies $x=y$, or equivalently, if $x\neq y$ implies $f(x)\neq f(y)$
        \item \textbf{Onto} If $f: A\to B$ is a function with range $B$, that is if $f(A) = B$, then $f$ is called onto. In other words, $f$ is onto if and only if the range of $f$ equals codomain of $f$
        \item \textbf{Restriction} Let $f:A\to B$ be a function and $S\subeteq A$. Then a function $f_S: S\to B$, called restriction of $f$ to $S$, can be formed by defining $f_S(x) = f(x)$ for all $x\in S$. (Note codomain stay unchanged for restriction)
        \item \textbf{Composite} Let $A$, $B$, $C$, be sets and $f:A\to B$ and $g: B\to C$ be functions. then $g\circ f: A\to C$ is a composite of $g$ and $f$, i.e. $(g\circ f)(x) = g(f(x))$ for all $x\in A$. 
        \begin{enumerate}
            \item Usually composites are not associative, i.e. $g\circ f \neq f \circ g$
            \item associative this way, $h\circ (g \circ f) = (h\circ g) \circ f$
        \end{enumerate}
        \item \textbf{Invertible Function} A function $f: \R \to \R$ is invertible if there exists a function $g: B\to A$ such that $(f\circ g)(y) = y$ for all $y\in B$ and $(g\circ f)(x) = x$ for all $x\in A$. If such a function $g$ exists, then it is unique and is called inverse of $f$, denoted as $f^{-1}$. 
        \item \textbf{Invertible Function Properties}
        \begin{enumerate}
            \item $f$ is invertible if and only if $f$ is both one-to-one and onto
            \item If $f: A\to B$ is invertible, then $f^{-1}$ is invertible, $(f^{-1})^{-1} = f$
            \item If $f: A\to B$, $g: B\to C$ are invertible, then $g\circ f$ is invertible and $(g\circ f)^{-1} = f^{-1} \circ g^{-1}$
        \end{enumerate}
    \end{enumerate}
\end{defn*}


\linksection{565}{Appendix C Fields}

\begin{defn*} \textbf{Field} \\
A field $F$ is a set on which two operations $+$ and $\cdot$ (addition and multiplication) are defined so that, for each pair of elements $x,y$ in $F$, there are unique elements sum, $x+y$, and products ,$x\cdot y$, in $F$ for which the conditions hold for all elements $a,b,c\in F$
    \begin{enumerate}
        \item Commutativity of addition and multiplication 
        \[
            a + b = b + a \quad \text{ and } \quad 
            a\cdot b = b\cdot a    
        \]
        \item Associativity of addition and multiplication 
        \[
            (a + b) + c = a + (b + c) \quad \text{ and }\quad 
            (a\cdot b)\cdot c = a\cdot (b\cdot c) 
        \]
        \item Existence of identity elements for addition and multiplication, i.e. exists distinct identity elements zero, $0$, and one, 1, in $F$ such that 
        \[
            0 + a = a \quad \text{ and } \quad 
            1 \cdot a = a    
        \]
        \item Existence of inverses for addition and multiplication, i.e. for each $a\in F$ and each nonzero element $b\in F$, there exists $c,d\in F$ such that
        \[
            a + c = 0 \quad \text{ and }\quad 
            b\cdot d = 1
        \]
        where $c$ is the additive inverse for $a$ and $d$ is a multiplicative inverse for $b$. 
        \item Distributivity of multiplication over addition 
        \[
            a\cdot (b+c) = a\cdot b + a\cdot c    
        \]
    \end{enumerate}
    \begin{example}
        $ $\\
        \begin{enumerate}
            \item The set of real numbers $\R$ with usual definition of addition and multiplication is a field. 
            \item The set of integers with usual definition of addition and multiplication is a field, since no inverses exist for addition and multiplication. 
        \end{enumerate}
    \end{example}
\end{defn*}

\begin{theorem*}
    \textbf{Cancellation Laws} \\
    For arbitrary elements $a,b,c$ in a field, following statements are true, 
    \begin{enumerate}
        \item If $a+b = c+b$, then $a=c$
        \item If $a\cdot b = c\cdot b$ and $b\neq 0$, then $a=c$
    \end{enumerate}
    \begin{proof}
        Prove second part, If $b\neq 0$, then exists multiplicative inverse $d$ such that $b\cdot d = 0$. Now multiply both sides of equation to by $d$, by associativity of multiplication and identity of multiplication we have 
        \[
            (a\cdot b) \cdot d = (c\cdot b)\cdot d \quad\to\quad 
            a \cdot (b\cdot d) = c\cdot (b\cdot d) \quad\to\quad 
            a \cdot 1 = c\cdot 1 \quad\to\quad 
            a = c
        \]
    \end{proof}
\end{theorem*}

\begin{corollary*}
    \textbf{Each element in field has unique additive/multiplicative inverse} \\
    The elements 0 and 1 mentioned in condition 3 of definition for field and $c$ and $d$ mentioned in condition 4 are unique
    \begin{proof}
        Suppose eists another zero $0'\in F$ such that $0' + a = a$ for all $a\in F$. Since $0+a = a$ for all $a\in F$, we have $0' + a = 0 + a$ so $0 = 0'$
    \end{proof}
    Additive inverse and multiplicative inverse are denoted by $-b$ and $d^{-1}$. They are used to represent subtraction and division
    \[
        a - b = a + (-b) \quad \quad
        \frac{a}{b} = a\cdot b^{-1}    
    \]
\end{corollary*}


\begin{theorem*}
    Let $a$ and $b$ be arbitrary elements of a field. Then each of the following statements are true 
    \begin{enumerate}
        \item $a\cdot 0 = 0$
        \item $(-a)\cdot b = a\cdot(-b) = -(a\cdot b)$
        \item $(-a)\cdot (-b) = a\cdot b$
    \end{enumerate}
    \begin{proof}
        $ $\\
        \begin{enumerate}
            \item 
            \[
                0 + a\cdot 0 = a\cdot 0 = a\cdot (0 + 0) = a\cdot 0 + a\cdot 0
            \]
            so $0 = a\cdot 0$ by cancellation theorem
            \item Note $-(a\cdot b)$ is an unique element of $F$ with property $a\cdot b + (-(a\cdot b)) = 0$. To prove $(-a)\cdot b = -(a\cdot b)$, we show $a\cdot b + (-a)\cdot b  =  0$
            \[
                a\cdot b + (-a)\cdot b = (a + -(a))\cdot b = 0\cdot b = 0
            \]
            Similarly for proving $a\cdot(-b) = -(a\cdot b)$
            \item Applying 2nd point twice
            \[
                (-a)\cdot (-b) = -(a\cdot (-b)) = -(-(a\cdot b)) = a\cdot b
            \]
        \end{enumerate}
    \end{proof}
\end{theorem*}



\begin{corollary*}
    The additive identity of a field has no multiplicative inverse. 
\end{corollary*}

\begin{defn*}
    \textbf{Characteristic of Field} The smallest positive integer $p$ for which a sum of $p$ 1's equals 0 is called the characteristic of $F$. If no such $p$ exists, then $F$ is said to have characteristic zero. ($\R$ has characteristic zero)
\end{defn*}


\linksection{565}{Appendix D Complex Number}


\begin{defn*}
    \textbf{Complex Number} A complex number is an expression of the form $z = a + bi$ where $a$ and $b$ are real numbers called the \textbf{real part} and the \textbf{imaginary part} of $z$, respectively. The sum and product of 2 complex numbers $z = a+bi$ and $w =c+di$ are defined as 
    \[
        z + w = (a+bi) + (c+di) = (a+c) + (b+d)i
    \]
    \[
        zw = (a+bi)(c+di) = (ac-bd) + (bc+ad)i
    \]
    \begin{enumerate}
        \item Any real number $c\in \R$ can be regarded as a complex number with $c + 0i$. 
        \item Any complex number of form $bi = 0 + bi$, where $b$ is nonzero  real, is called an \textbf{imainary}. The product of 2 imaginary number is real 
        \[
            (bi)(di) = (0 + bi)(0 + di) = (0-bd)  + (b\cdot 0 + 0\cdot d)i = -bd   
        \]
        In particular, for $i = 0 + 1i$, $i\cdot i = -1$
        \item Real number $0$ is an additive identity for the complex numbers; Real number $1$ is a multiplicative identity element for the set of complex number  
        \item Each complex number $a + bi$ has an additive inverse. Each complex number except 0 has a multiplicative inverse, 
        \[
            -(a + bi) = (-a) + (-b)i
        \]
        \[  
            (a+bi)^{-1} = (\frac{a}{a^2 + b^2}) - (\frac{b}{a^2+b^2})i    
        \]
    \end{enumerate}
\end{defn*}


\begin{theorem*}
    The set of complex numbers with the operations of addition and multiplication previously defined is a field. (Just verify all the conditions...)
\end{theorem*}


\begin{defn*}
    \textbf{Complex Conjugate} The complex confugate of a complex number $a+bi$ is the complex number $a-bi$. Denote conjugate of a complex number $z$ by $\overline{z}$. As an example, 
    \[
        \overline{-3 + 2i} = -3 - 2i \quad \quad 
        \overline{6} = \overline{6+0i} = \overline{6-0i} = 6
    \]
\end{defn*}


\begin{theorem*} \textbf{Complex Conjugate Properties} \\
    Let $z$ and $w$ be complex numbers. Then the following statement is true 
    \begin{enumerate}
        \item $\overline{\overline{z}} = z$
        \item $\overline{(z+w)} = \overline{z} + \overline{w}$
        \item $\overline{zw} = \overline{z}\cdot \overline{w}$
        \item $\overline{(\frac{z}{w})} = \frac{\overline{z}}{\overline{w}}$
        \item $z$ is a real number if and only if $\overline{z}=z$
    \end{enumerate}
\end{theorem*}

\begin{defn*}
    \textbf{Abosolute Value} let $z = a+bi$, where $a,b\in \R$. The absolute value (modulus) of $z$ is the real number $\sqrt{a^2 + b^2}$. We denote the absolute value of $z$ by $|z|$. Note $z\overline{z} = |z|^2$, follows from 
    \[
        z\overline{z} = (a+bi)(a-bi) = a^2 + b^2    
    \]
    gives that product of a complex number with its conjugate is a real number provides an easy method for determining the quotient of 2 complex numbers, if $c+di \neq 0$, then 
    \[
        \frac{a+bi}{c+di} 
        = \frac{a+bi}{c+di} \frac{c-di}{c-di}
        = \frac{(ac+bd) + (bc-ad)i}{c^2 + d^2}
        = \frac{ac+bd}{c^2+d^2} + \frac{bc-ad}{c^2+d^2}i
    \]
    Also note, $|\overline{z}| = |z|$, and $|z| = |-z|$
\end{defn*}


\begin{theorem*}
    Let $z$ and $w$ denote any two complex numbers, then the following are true 
    \begin{enumerate}
        \item $|zw| = |z|\cdot |w|$
        \item $|\frac{z}{w}| = \frac{|z|}{|w|}$ if $w\neq 0$
        \item $|z+w| \leq |z| + |w|$  (triangular inequality)
        \item $|z| - |w| \leq |z+w|$
    \end{enumerate}
\end{theorem*}


\begin{defn*}
    \textbf{Geometric Interpretation} In $\R^2$, there are two axes, the real axis and the imaginary axis, the absolute value of $z$ gives the length of the vector $z$. By a special case of Euler's formula $e^{i\theta} = \cos{\theta} + i\sin{\theta}$, we use $e^{i\theta}$ to represent the unit vector that makes an angle $\theta$ with the positive real axis. Any nonzero complex number $z$ can be depicted as a multiple of a unit vector, i.e. $z = |z|e^{i\theta}$
\end{defn*}


\begin{theorem*}
    \textbf{The Foundamental Theorem of Algebra} Suppose that $p(z) = a_n  z^n + a_{n-1} z^{n-1} + \cdots + a_1 z + a_0$ is a polynomial in $P(C)$ of degree $n\geq 1$. Then $p(z)$ has a zero 
    \begin{rem}
        The theorem states that every non-constant single-variable polynomial with complex (or specifically real) coefficients has at least one complex root.
    \end{rem}
\end{theorem*}


\begin{corollary*}
    If $p(z) = a_n  z^n + a_{n-1} z^{n-1} + \cdots + a_1 z + a_0$ is a polynomial of degree $n\geq 1$ with complex coefficients, then there exists complex number $c_1,c_2, \cdots, c_n$ such that 
    \[
        p(z) = a_n(z-c_1)(z-c_2)\cdots (z-c_n)    
    \]
    In other words, all polynomials can be factored in this case.
\end{corollary*}



\begin{defn*}
    \textbf{Algebraically Closed} A field i called algebraically closed if it has the property that every polynomial of positive degree with coefficients from that field factors as a product of polynomial of degree 1. 
\end{defn*}



\end{document}
