\documentclass[11pt]{article}
\input{/Users/markwang/.preamble}
\begin{document}




\subsection*{Problem 2}
Consider a variant on the problem of Interval Scheduling where instead of wanting to schedule as many jobs as we can on one processor, we now want to schedule ALL of the jobs on as few processors as possible.\\
The input to the problem is $(s_1,f_1), (s_2,f_2), ···, (s_n,f_n)$, where $n \geq 1$ and all $s_i < f_i$ are nonnegative integers. The integers $s_i$ and $f_i$ represent the start and finish times, respectively, of job $i$.\\
A schedule specifies for each job $i$ a positive integer $P(i)$ (the “processor number” for job $i$). It must be the case that if $i \neq j$ and $P(i) = P(j)$, then jobs $i$ and $j$ do not overlap. We wish to find a schedule that uses as few processors as possible, i.e., such that $max\{P(1), P(2), \cdots , P(n)\}$ is minimal.\\


\begin{enumerate}
  \item Design an algorithm to solve the problem in time $O(n^2)$, i.e. strictly less than $O(n^2)$\\
  \begin{enumerate}
    \item Let $P$ be an empty array of size $n$ , representing $P(i)$ at index $i$
    \item Sort $(s_i, f_i)$ by start time such that for all $i\leq j$, $s_i \leq s_j$
    \item Starting from the start of the sorted job array $J$ and for each job $j_i$ do
    \begin{enumerate}
      \item Starting from processor number $k = 1$
      \item Define $J_k\subseteq J$ such that for all $j\in J_k$, $P(j) = k$. If $j_i$ is compatible with all $j\in J_k$, then assign $j_i$ processor number $k$, i.e. let $P(i) \leftarrow k$
      \item Otherwise, increment $k$ and try the previous step again until $j_i$ is assigned to either a previously used processor number or a new processor number not used before.
    \end{enumerate}
    \item Return $P$
  \end{enumerate}

  \item Prove that the above algorithm is guaranteed to compute a schedule that uses the minimum number of processors.\\

  We will prove that the greedy choice is always in some optimal solution to the problem. Then we prove that the problem exhibits optimal substructure. Here we define a \textit{compatible} processor number $k$ for job $j$ be an integer such that all jobs previously assigned $k$ are compatible with $j$. Let $J$ be the input jobs given. Let $J_t := \{ j_i \in J: s_i \geq s_t\}$ be subset of $J$ such that all jobs in $J_t$ starts after $j_t$ starts. Let $Max(P)$ be the maximum of processor numbers in $P$


  \begin{proposition*}
    Consider any subproblem $J_t$, let $j_i \in J_t$ be the job with earliest starting time, and let be $k$ be the lowest compatible processor number with $j_i$. Then assigning $k$ to $j_i$ is in some optimal ($max\{ P(1), \cdots, P(n) )\}$ minimized) solution to $J_t$
   \end{proposition*}
  \begin{proof}
    Assume $P'$ is an arbitrary optimal solution to $J_t$. Let $k' = P'(i)$. If $k = k'$, then we are done the proof since $k$ is assigned to $j_i$ by the greedy choice, which is in the optimal solution $P'$. Otherwise if $k \neq k'$, since $k$ is the lowest processor number possible (i.e. $k \leq k'$ ), then $k < k'$. Now we can construct a solution $P = P'$ where $P(i) = k$, thus $P(i) < P'(i)$. We arrive at a contradiction on the assumption that $P'$ is optimal. Hence we conclude that the greedy choice is always in some optimal solution to $J_t$
  \end{proof}

  \begin{proposition*}
    The scheduling problem exhibits optimal substructure.
  \end{proposition*}
  \begin{proof}
    Given arbitrary index $i$, we separate the problem into a greedy choice and a single subproblem, i.e. $\{ j_i\}$ and $J_{after} = J_i$. We make the choice assigning a processor number $k$ to $j_i$, Assume such assignment is in some optimal solution to the problem $P'$. Now we are left with assigning processor number to $J_{after}$ with $P_{after}$. Then the optimal solution follows
    \[
      Max(P') = Max\{k, Max(P_{after}) \}
    \]
    We claim that if $P'$ is optimal, then $P_{after}$ is also optimal, in a sense that if $Max(P_{after}) > k$, then $Max(P_{after})$ is minimized. If $Max(P_{after}) \leq k$, then solution is already optimal. Otherwise if $Max(P_{after})>k$, then suppose we can find a more optimal solution $P_{after}''$ such that $Max(P_{after}'') < Max(P_{after})$ then we can substitute $P_{after}''$ for $P_{after}$ and construct another solution set $P''$ with
    \[
      Max(P'') = Max\{k, Max(P_{after}'')\} < Max\{k, Max(P_{after})\} = Max(P')
    \]
    Hence contradicting the optimality assumption for $P'$, hence $P_{after}$ must be optimal in itself.
  \end{proof}

  We conclude by combining propositions proved earlier. By optimal substructure of the problem, given that at each step the greedy choice is optimum and we are left with finding optimal solution to a smaller subproblem, i.e. $J_{after}$, the solution to the original solution is optimal, specifically, the algorithm uses minimum number of processors.

  \item Briefly describe an efficient implementation of the algorithm, making it clear what data structures you are using. Express the running time of your implementation as a function of $n$ (the number of jobs), using appropriate asymptotic notation.\\

  We will use a min heap $H$ to store an array of finish time of currently scheduled jobs. $Q.size$ is size of the heap and assume is updated during insertion and deletion.

    \begin{algorithm}[H]
       \SetKwFunction{sa}{Schedule-All}
       \SetKwFunction{hmax}{Heap-Maximum}
       \SetKwFunction{extrmax}{Heap-Extract-Max}
       \SetKwFunction{hins}{Heap-Insert}
       \Fn{\sa$(s, f)$}{
          \KwIn{$s,f$ are arrays of size $n$, representing job $j_i = (s_i, f_i)$ at index $i$}
          \KwOut{$P$ is an array of size $n$ storing $P(i)$ at index $i$, where $max\{ P(1), \cdots, P(n) \}$ is minimized }

          $P \leftarrow$ Array of size $n$\\
          $H \leftarrow$ Min-Heap\\
          Sort $s, f$ by start time together such that $s_i \leq s_j$ for all $i\leq j$\\
          \For{$i = 1$ \KwTo $n$}{
            \While{\hmax$(H) < s_i$}{
              \extrmax$(H)$
            }
            \hins$(H, f_i)$\\
            $P(i) \leftarrow H.size$
          }
          \Return{P}
       }
     \end{algorithm}

     At line 6-7, we remove jobs' finish time from the heap $H$ such that the heap retains previously scheduled jobs that overlaps $j_i$. The size of the heap represent the smallest compatible processor number, which we record in solution $P(i)$ at each iteration afer inserting the finish time of $j_i$ to the heap. \\

     Now we analyze running time
     \begin{enumerate}
       \item Sorting takes $O(n\lg n)$
       \item By the time the procedure terminates, each jobs' finish time is inserted and removed from the heap, since  \textsc{Heap-Extract-Max} and \textsc{Heap-Insert} has worst case running time of $O(\lg n)$. Heap insertion and deletion has worst case running time of $O(2n\lg n) = O(n\lg n)$
       \item \textsc{Heap-Maximum} is called at least once per iteration of for loop for a total of $n$ iterations; and it is called at most $n$ number of times for each successful condition evaluation and subsequent deletion operation (since at most deleting a total of $n$ items). \textsc{Heap-Maximum} has worst case running time of $O(1)$ hence by the time procedure terminates, heap lookup operation has a worst case running time of $O(2n) = O(n)$
       \item Assigning $P$ at index $i$ takes $O(1)$ each iteration and since there are $n$ iterations, has a worst case running time of $O(n)$
       \item To conclude, the algorithm has a worst case running time of $O(n\lg n)$
     \end{enumerate}

\end{enumerate}


\subsection*{Problem 3}

Here is another variant on the problem of Interval Scheduling. Suppose we now have two processors, and we want to schedule as many jobs as we can. As before, the input is $(s_1,f_1), (s_2,f_2), ···, (s_n,f_n)$, where $n \geq 1$ and all $s_i < f_i$ are nonnegative integers. A \textit{schedule} is now defined as a pair of sets $(A_1,A_2)$, the intuition being that $A_i$ is the set of jobs scheduled on processor $i$. A schedule must satisfy the obvious constraints: $A_1 \subseteq \{1,2,\cdots,n\}$, $A_2\subseteq \{1, 2,\cdots , n\}$, $A_1 \cap A_2 = \emptyset$, and for all $i\neq j$ such that $i , j \in A_i$ or $i , j \in A_2$ , jobs $i$ and $j$ do not overlap.

\begin{enumerate}
  \item Design an algorithm (write a pseudocode) to solve the above problem in time $O(n^2)$, i.e., strictly less than $O(n^2)$.\\

  Let $J: \{ 1, 2, \cdots, n\}$ be the input set of jobs given. Here we define that a job $j\in J$, having start time $s$, is \textit{compatible} with $J_s\subseteq J$ if $j$ starts after every job in $J_s$ finishes, in other words,
  \[
    \forall j \in J_s: f_j \leq s
  \]
  Let subproblem $J_t$ be a set of jobs such that
  \[
    \forall j \in J_t: f_t \leq s_j
  \]
  in otherwords, $J_t$ is the set of jobs that starts after job $t$ ends. \\
  We define \textit{waste time} $W_i$ for a job $j$, having start time $s$, with respect to a compatible set of jobs $J_s$ as
  \[
    W_i = s - \underset{j\in J_s}{Max}\{ f_j \}
  \]
  in other words, the waste time is the time period between the finish time of the last finishing jobs in $J_s$ and the start time of the job $j$ in consideration

  \begin{enumerate}
    \item Let $(A_1, A_2) = (\O, \O)$
    \item Sort $(s_i, f_i)$ by finish time such that for all $i\leq j$, $f_i \leq f_j$
    \item Starting from the start of the sorted job array and for each job $j \in J$ do
    \begin{enumerate}
      \item Test if $j$ is compatible with $A_1$ and $A_2$.
      \item If $j$ is not compatible with either set, continue to next iteration
      \item If $j$ is compatible with only one of $A_1$ and $A_2$, then add $j$ to the compatible set
      \item If $j$ is compatible with both $A_1$ and $A_2$, then add $j$ to $A_i$ such that waste time for job $j$ with respect to $A_i$, $W_i$, is minimized.
    \end{enumerate}
    \item Return $(A_1, A_2)$
  \end{enumerate}


  \item Prove that the above algorithm is guaranteed to compute an optimal schedule.

  \begin{proposition*}
    Consider any subproblem $J_t$, let $j_e\in J_t$ be the job with earliest finish time with waste time $W_1$ and $W_2$. Then making the choice of assigning $j_e$ described above is in some optimal (i.e. $|A_1| + |A_2|$ maximized) solution to the problem.
  \end{proposition*}

  \begin{proof}
    Let $(O_1, O_2)$ be some optimal solution to the original problem $J$. Let $(A_1 \subseteq O_1, A_2\subseteq O_2)$ be the optimal solution to the subproblem $J_t$ with respect to $(O_1, O_2)$. and let $(B_1 \subseteq O_1, B_2\subseteq O_2)$ be the optimal solution to subproblem $J\setminus J_t$. Let $j_e\in J_t$ be job with earliest finish time. By definition of $J_t$, $j_e$ is compatible with at least one of $B_i$.
    \begin{enumerate}
      \item Suppose $j_e$ is compatible with exactly one of $B_i$, without loss of generality, suppose $B_1$ is the compatible set and $B_2$ is the in-compatible set. Let $j_a \in A_1$ be the first finishing job in $A_1$. Then we have,
      \begin{enumerate}
        \item If $j_e = j_a$, then the proposition holds
        \item If $j_e \neq j_a$. Since $j_e$ is not compatible with $B_2$, $j_e\not\in A_2$. Then consider a new solution set $A_1' = A_1 \cup \{ j_e \} \setminus \{ j_a\}$. Note $A_2$ is unchanged. jobs in $A_1'$ are disjoint because $A_1$ is disjoint, $j_a\in A_1$ is the first job to finish and $f_{j_e} \leq f_{j_a}$. Since $|A_1'| + |A_2| = |A_1| + |A_2|$, $(A_1', A_2)$ is an optimal solution to subproblem $J_t$ that contains $j_e$, hence the proposition holds.
      \end{enumerate}
      Similar argument holds if $B_2$ is the compatible set
      \item Now consider the case where $j_e$ is compatible with both $B_1$ and $B_2$. Without loss of generality, suppose $W_1 < W_2$, hence the greedy algorithm assigns $j_e$ to $B_1$. Let $j_a\in A_1$ be the earlist finishing job in $A_1$; let $j_b\in A_2$ be the earlist finishing job in $A_2$,
      \begin{enumerate}
        \item If $j_e = j_a$, then the proposition holds
        \item If $j_e \neq j_a$, Since $j_e$ is compatible with $B_2$ as well as $B_1$ there are two cases as to where $j_e$ might end up
        \begin{enumerate}
          \item If $j_e = j_b$, then consider a new solution set where $A_1' = A_2$ and $A_2' = A_1$, i.e. switching the set of jobs for processor 1 and 2. Note $A_1'$ and $A_2'$ are disjoint sets of jobs since $A_1$ and $A_2$ are disjoint sets. and $|A_1'| + |A_2'| = |A_1| + |A_2|$. Since $(A_1, A_2)$ optimal, then $(A_1', A_2')$ are optimal solutions and that $A_1'$ now contains $j_e$, because $j_e = j_b \in A_2 = A_1'$. The proposition hence holds.
          \item If $j_e$ is not in either $A_1$ or $A_2$, then consider a new solution $A_1' = A_1 \cup \{ j_e\} \setminus \{ j_a \}$. Proposition holds with same argument provided in $(a). ii.$
        \end{enumerate}
      \end{enumerate}
      Similar argument holds if $W_1 > W_2$.
    \end{enumerate}


    \begin{proposition*}
      This scheduling problem exhibits optimal substructure.
    \end{proposition*}
    \begin{proof}
      Given arbitrary index $t$, we separate the problem into a greedy choice and a single subproblem, i.e. $\{ j_e \}$ and $J_t$. Let $(A_1, A_2)$ be the optimal solution for $J_t$. We make the greedy choice of adding $j_e$ to $A_i$ or skipping $j_e$ (i.e. $j_e = \O$) to maximize time as a resource for subsequent jobs, and to minimize waste time as a resource if there is a choice to select one assuming both processors are available at the time. Assume such greedy choice is optimal, we are left with processing a smaller subproblem $J_{after} = J_t \setminus \{ j_e \}$. We claim that if $(A_1, A_2)$ is optimal, then solution to subproblem $J_{after}$, $(C_1, C_2)$ must also be optimal. Consider an alternative solution $(C_1', C_2')$ that is even more optimal, i.e. $|C_1'| + |C_2'| \geq |C_1| + |C_2|$. We can construct a new solution by replacing $(C_1, C_2)$ with $(C_1, C_2)$ and get an overall more optimal solution $(A_1' = \{ j_e \} \cup C_1', A_2' = \{ j_e \}\cup C_2')$ such that
      \[
        |A_1'| + |A_2'| < |A_1| + |A_2|
      \]
      Contradicts assumption that $(A_1, A_2)$ is optimal. Hence solution to subproblem $J_{after}$ must be optimal
    \end{proof}

    We conclude by combining propositions proved earlier. By optimal substructure of the problem, given that at each step the greedy choice is optimum and we are left with finding optimal solution to a smaller subproblem, i.e. $J_{after}$, the solution to the original solution is optimal, specifically, the algorithm schedules most jobs on two processors.
  \end{proof}

  \item Briefly describe an efficient implementation of the algorithm, making it clear what data structures you are using. Express the running time of your implementation as a function of $n$ (the number of jobs), using appropriate asymptotic notation.

  Let $A_1$ and $A_2$ be two linked list. Job $j$ may be appended to the tail of $A_1$ or $A_2$ in constant time $O(1)$. Assume that the last job added to $A_i$ can be efficiently looked up in $O(1)$ time with $A_1.tail$ operation.


  \begin{algorithm}[H]
     \SetKwFunction{st}{Schedule-On-Two-CPU}
     \SetKwFunction{ll}{Linked-List}
     \SetKwFunction{cp}{Compatible}

     \SetKwFunction{both}{Both}
     \SetKwFunction{none}{None}
     \SetKwFunction{one}{diff-one}
     \SetKwFunction{two}{diff-two}

     \SetKwFunction{test}{T}


     \Fn{\cp$(j, A_1, A_2, s, f)$}{
      \KwOut{Returns \none if $j$ not compatible with $A_1$ or $A_2$, Return \both if $j$ is compatible with both $A_1$ and $A_2$, and return the processor number, either 1 or 2, if $j$ is compatible with only $A_1$ or $A_2$. Each return statement also include the computed waste time $W_1$ and $W_2$}

      $\one = s[j] - f[A1.tail]$ \\
      $\two = s[j] - f[A2.tail]$ \\

      \If{$\one > 0$ and $\two > 0$}{
        \Return{$(\both, \one, \two)$}
      }\ElseIf{$\one \leq 0$ and $\two\leq 0$}{
        \Return{$(\none, \one, \two)$}
      }\Else{
        \If{$\one \geq 0$}{
          \Return{$(1, \one, \two)$}
        }\Else{
          \Return{$(2, \one, \two)$}
        }
      }


      \If{ and }
     }


     \Fn{\st$(s, f)$}{
        \KwIn{$s,f$ are arrays of size $n$, representing job $j_i = (s_i, f_i)$ at index $i$}
        \KwOut{$(A_1, A_2)$ is a set of solution to the problem given}

        $A_1,A_2 \leftarrow$ \ll\\
        $A_1.append(1)$ \quad \quad \quad \tcp{Add first finishing job arbitrarily to $A_1$}\\
        \For{$j=2$ \KwTo $n$}{
          $(\test, W_1, W_2) = \cp(j, A_1, A_2, s, f)$ \\
          \If{\test is \both}{
            \If{$W_1 < W_2$}{
              $A_1.append(j)$
            }\Else{
              $A_2.append(j)$
            }
          }
          \ElseIf{\test is 1}{
            $A_1.append(j)$
          }\ElseIf{\test is 2}{
            $A_2.append(j)$
          }
        }
        \Return{$(A_1, A_2)$}
     }
   \end{algorithm}

   Now we analyze running time. There are $O(n)$ iterations, and in each iteration, at most one $append$ operation and $tail$ operation, each $O(1)$, is required for Linked List operation. There is also some $O(1)$ array lookup in $s$ and $f$. Hence the algorithm has a worst case running time of $O(n)$

   

\end{enumerate}





\end{document}
