\documentclass[11pt]{article}
\input{/Users/markwang/.preamble}
\begin{document}



\section*{22 Elementary Graph Algorithms}


\subsection*{22.1 Representations of graphs}




\begin{defn*}
    Representations of graphs
    \begin{enumerate}
        \item \textbf{Adjacency List} 
        \begin{enumerate}
            \item An array of $|V|$ lists, one for each vertex in $V$. For each $u\in V$, $Adj[u]$ contains all the vertices $v$ such that $(u,v) \in E$ (i.e. all vertices adjacent to $u$ in $G$)
            \item compact for \textbf{sparse} graphs ($|E| << |V|^2$)
            \item For \textbf{directed graph}, the sum of lengths of all adjacency list is $|E|$, since edge of form $(u,v)$ is represented as having $v$ appearing in $Adj[u]$. (i.e. $u \to v$)
            \item For \textbf{undirected graph}, the sum of lengths of all the adjacency lists is $2|E|$, since if $(u,v)$ is an undirected edge, then $u$ appears in $v$'s adjacency list and vice versa 
            \item Memory: $\Theta(V + E)$
            \item Search: $\Theta(E)$ Have no quick way of determining if a given edge $(u,v)$ is present in the graph than to search for $v$ in the adjacency list $Adj[u]$ (have to go through the list)
        \end{enumerate}
        \item \textbf{Adjacency Matrix}
        \begin{enumerate}
            \item Assume vertices numbered $1, 2, \cdots, |V|$ arbitrarily. We have a $|V| \times |V|$ matrix $A = (a_{ij})$ such that 
            \[
                a_{ij} = 
                \begin{cases}
                    1 & \text{if } (i,j)\in E \\
                    0 & \text{otherwise}
                \end{cases}   
            \]
            \item good for \textbf{dense} graphs ($|E| ~= |V|^2$) or if need to tell if there is an edge between two vertices quickly
            \item Memory: $\Theta(V^2)$
            \item Search: $\Theta(1)$
        \end{enumerate}
    \end{enumerate}
\end{defn*}


\subsection*{BFS $\Theta(V+E)$}

\begin{lemma*}
    For Proof of correctness
    \begin{enumerate}
        \item Let $G = (V, E)$ be directed or undirected graph, let $s\in V$ be an arbitrary vertex, for any edge $(u,v) \in E$, 
        \[
            \delta(s, v) = \delta(s, u) + 1
        \]
        \begin{proof}
            The shortest path from $s$ to $v$ cannot be longer than shortest path from $s$ to $u$ followed by $(u,v)$, since otherwise we just take shortest path from $s$ to $v$ and $(u,v)$ which will be a shorter path. 
        \end{proof}
        \item Upon termination, for each vertex $v\in V$, the value $v.d$ computed by BFS satisfies $v.d \geq \delta(s,v)$
        \begin{proof}
            Induction on the number of enqueue operations. Inductive hypothesis is that $v.d \geq \delta(s, v)$ for all $v\in V$. Before $s$ enqueued, I.H. holds since $v.d = \infty \geq 0 = s.d = \delta(s,s)$. Now consider a white vertex $v$ that is just being discovered and we we serach $Adj[u]$. I.H. implies $u.d \geq \delta(s,u)$ By assignment of $v.d = u.d + 1$, and previous lemma (since $u \to v$)
            \[
                v.d = u.d + 1 \geq \delta(s,u) + 1 \geq \delta(s,v)
            \]
        \end{proof}
        \item suppose queue $Q$ contains vertices $\langle v_1, \cdots, v_r \rangle$, where $v_1$ is the head of $Q$ and $v_r$ is the tail. Then $v_r.d \leq v_1.d + 1$ and $v_i.d \leq v_{i+1}$ for $i=1,\cdots, r-1$
        \begin{proof}
            Proof by induction on number of queue operations. Initially, queue contains $s$ only, lemma holds. Now we prove in inductive step that lemma holds after both dequeuing and enqueuing a vertex. If head $v_1$ is dequeued, $v_2$ is the new head. By inductive hypothesis $v_1.d \leq v_2.d$. But then we have $v_r.d \leq v_1.d + 1 \leq v_2.d + 1$, the remaining inequalities remain unaffected, so lemma holds with after dequeu of $v_1$. During an enqueue, say $v$, it becomes $v_{r+1}$. At that time, we just moved $u$ from the queue. By inductive hypothesis, the new head $v_1$ satisfies $v_1.d \geq u.d$. We have 
            \[
                v_{r+1}.d = v.d = u.d + 1 \leq v_1.d + 1
            \]
            Now to prove inequalities holds, by I.H. $v_r.d \geq u.d + 1$ and so $v_r.d \leq u.d + 1 = v.d = v_{r+1}.d$ and the remaining inequalities remain unaffected. So lemma holds during enqueue
        \end{proof}
    \end{enumerate}
\end{lemma*}


\begin{theorem*}
    \textbf{Correctness of BFS} Let $G = (V,E)$ be directed or undirected graph, suppose BFS is run on $G$ given $s\in V$. during execution, BFS discovers every vertex $v\in V$ reachable from $s$ and upon termination, $v.d = \delta(s,v)$ for all $v\in V$. Moreover, for any vertex $v\neq s$ reachable from $s$, one of the shortest paths from $s$ to $v$ is a shortest path from $s$ to $v.\pi$ followed by $(v.\pi, v)$
\end{theorem*}


\begin{defn*}
    \textbf{Breadth-first Tree} For graph $G = (V,E)$ with source $s$, a predecessor subgraph of $G$, $G_{\pi} = (V_{\pi}, E_{\pi})$ where 
    \[
        V_{\pi} = \{ v\in V: v.\pi \neq NIL\} \quad \quad E_{\pi} = \{ (v.\pi, v): v\in V_{\pi} - \{s\}\}
    \]
    The predecessor graph $G_{\pi}$ is a Breadth first tree if $V_{\pi}$ consists of vertices raechable from $s$, and for all $v\in V_{\pi}$, the subgraph $G_{\pi}$ contains a unique simple path from $s$ to $v$ that is also the shortest path from $s$ to $v$ in $G$
\end{defn*}


\begin{lemma*}
    procedure BFS constructs $\pi$ such that predecessor graph $G_{\pi} = (V_{\pi}, E_{\pi})$ is a breadth-first tree
\end{lemma*}



\subsection*{DFS $\Theta(V+E)$} 

\begin{defn*}
    \textbf{Depth-first Tree} For graph $G = (V,E)$ with source $s$, a predecessor subgraph of $G$, $G_{\pi} = (V, E_{\pi})$ where 
    \[
        E_{\pi} = \{ (v.\pi, v): v\in V \text{ and } v.\pi \neq NIL\}
    \]
    The predecessor subgraph of a depth-first search forms a \textbf{Depth-first forest} comprisong several \textbf{Depth-first trees}. The edges in $E_{\pi}$ are \textbf{tree edges} (Note how we are not restricting $V$ since DFS will include vertices unreachable from source $s$)
\end{defn*}


\begin{proposition*}
    \textbf{Timestamping}
    \begin{enumerate}
        \item \textbf{Timestamp} $v.d$ records when $v$ first discovered ($WHITE\to GRAY$) and $v.f$ records when finishes examing $Adj[v]$ ($GRAY\to BLACK$)
        \item Vertex $u$ is $WHITE$ before time $u,d$, $GRAY$ between $u.d$ and $u.f$ ands $BLACK$ thereafter
        \item vertex $v$ is a descendent of $u$ in Depth-first forest if and only if $v$ is discovered during the time in which $u$ is gray
    \end{enumerate}
\end{proposition*}


\begin{theorem*}
    \textbf{Parenthesis theorem} In DFS of $G$, for any two vertices $u$ and $v$, exactly one of following holds 
    \begin{enumerate}
        \item $[u.d, u.f]$ and $[v.d, v.f]$ are entirely disjoint, then neither $u$ nor $v$ is a descendent of the other in depth-first forest 
        \item $[u.d, u.f]$ contained entirely within $[v.d, v.f]$, and $u$ is a descendent of $v$ in the depth-first tree 
        \item $[v.d, v.f]$ is containedf entirely within $[u.d, u.f]$ and $v$ is a descendent of $u$ in depth-first tree
    \end{enumerate}
\end{theorem*}


\begin{corollary*}
    \textbf{Nesting of descendents' interval} Vertex $v$ is a proper descendent of $u$ in depth-first forest for a graph $G$ if and only if 
    \[
        u.d < v.d < v.f < u.f
    \]
\end{corollary*}


\begin{theorem*}
    \textbf{White-Path Theorem} In depth-first forest of $G = (V,E)$, vertex $v$ is a descendent of $u$ if and only if at time $u.d$ that search discovers $u$, there is a path from $u$ to $v$ consisting entirely of white vertices
\end{theorem*}

\begin{proposition*}
    \textbf{classification of edges}
    \begin{enumerate}
        \item \textbf{Tree Edges} edges in depth-first forest $G_{\pi}$. $(u,v)$ is a tree edge if $v$ was first discovered by exploring edge $(u,v)$
        \item \textbf{Back Edges} are $(u,v)$ connecting $u$ to an ancestor $v$ in a depth-first tree. Self-loos (in directed graph) is also back edge 
        \item \textbf{Forward Edges} are edges $(u,v)$ connecting $u$ to a descendent $v$ in a depth-first tree 
        \item \textbf{Cross Edges} are all other edges. They go between vertice same vertices in the same depth-first tree, as long as one vertex is not an ancestor of the other, or they can go between vertices in different depth-first trees.
    \end{enumerate}
    When $(u,v)$ is first explored, the color of vertex $v$ tells us about its edge category
    \begin{enumerate}
        \item $WHITE$ indicate a tree edge 
        \item $GRAY$ indicate a back edge 
        \item $BLACK$ indicates a forward (if $u.d < v.d$) or cross edge (if $u.d > v.d$)
    \end{enumerate}
\end{proposition*}


\begin{theorem*}
    In DFS of an undirected graph $G$, every edge of $G$ is either a tree edge of a back edge 
\end{theorem*}


\section*{23 Minimum Spanning Tree}

\begin{defn*}
    \textbf{The MST Problem} Given a connected, undirected, weighted graph $G = (V,E)$, with weight function $w: E\to \R$. Find an acyclic subset $T\subseteq E$ that connects all of the vertices and whose total weight 
    \[
        w(T) = \sum_{(u,v)\in T} w(u,v)
    \]
    is minimized. Since $T$ is acyclic and connects all of the vertices, we call it a \textbf{spanning tree}. 
\end{defn*}
$ $\\
\textbf{Generic Greedy solution to MST}
Generic greedy method for MST involves maintaining a loop invariant on a set $A \subseteq E$, 
\[
    \text{Prior to each iteration, A is a subset of some MST}
\]
At each step, we determine an edge $(u,v)$ that we can add to $A$ without violating the invariant. Assume $A \subseteq E$ satisfies the loop invariant, a \textbf{safe edge} $(u,v)$ is an edge such that $A\cup \{(u,v) \}$ maintains the invariant


\begin{defn*}
    \textbf{Cut} 
    \begin{enumerate}
        \item A \textbf{cut} $(S, V-S)$ of an undirected graph $G = (V,E)$ is a partition of $V$. 
        \item An edge $(u,v) \in E$ \textbf{crosses} the cut if one of its endpoints is in $S$ and the other is in $V - S$
        \item A cut \textbf{respects} a set $A \subseteq E$ if no edges in $A$ crosses the cut 
        \item An edge is a \textbf{light edge} crossing a cut if its weight is the minimum of any edge crossing the cut (maybe $\geq 1$ light edges in case of tie)
        \item A \textbf{light edge satisifying a given property} if its weight is minimum of any edge satisfying the property 
    \end{enumerate}
\end{defn*}


\begin{proposition*}
    \begin{enumerate}
        \item \textbf{Possible Multiplicity} If there are $n$ vertices in the graph, then each spanning tree has $n − 1$ edges.
        \item \textbf{Cycle property} For any cycle $C$ in the graph, if the weight of an edge $e$ of $C$ is larger than the individual weights of all other edges of $C$, then this edge cannot belong to a MST.
    \end{enumerate}
\end{proposition*}


\begin{theorem*}
    \textbf{The greedy choice (light edge) is optimal (safe)} Let $G = (V,E)$ be connected, undirected graph with $w:E\to \R$. Let $A\subseteq E$ be included in some MST for $G$, let $(S,V-S)$ be any cut of $G$ that respects $A$, and let $(u,v)$ be a light edge crossing $C = (S,V-S)$. Then edge $(u,v)$ is safe for $A$, i.e. $A \cup \{(u,v) \}$ is also in a subset of some MST
    \begin{proof}
        Let $T$ be a MST such that $A\subseteq T$. Assume $T$ does not contain light edge $(u,v)$, since otherwise $A \cup \{(u,v) \} \subseteq T$, done. Otherwise, $(u,v) \not\in T$. Prove using cut-and-paste that $(u,v)$ is safe. In the context of MST, inclusion of $(u,v)$ in $T$ forms a cycle, $(u,v)$ along with path $p$, s.t. $u \overset{p}{\leadsto} v$. Since $T$ is by definition simple, $p$ is also simple. Since $u$ and $v$ are on opposite sides of the cut $C$, let $(x,y)$ be an edge that crosses $C$. Note $(x,y) \not\in A$ since $C$ respects $A$. Let $T' = T \cup \{(u,v) \} \setminus \{ (x,y) \}$. $T$ is connected since removal of $(x,y)$ breaks $T$ into 2 components, and inclusion of $(u,v)$ joins the components together. Now we show that $T'$ is a MST. Since $(u,v)$ is a light edge crossing $C$ and $(x,y)$ also crosses the cut, we have $w(u,v) \leq w(x,y)$, hence 
        \[
            w(T') = w(T) + w(u,v) - w(x,y) \leq w(T)
        \]
        Since $T$ already a MST, i.e. $w(T) \leq w(T')$, then $w(T) = w(T')$. $T'$ is also MST. Now we show $(u,v)$ is safe for $A$. since $A\subseteq T$, $A\subseteq T'$ since $(x,y) \not\in A$. hence $A \cup \{ (u,v)\} \subseteq T'$. Since $T'$ is MST, $(u,v)$ is safe for $A$.
    \end{proof}
\end{theorem*}



\begin{corollary*}
    \textbf{Above theorem holds for cuts in form of connected component} Let $G =(V,E)$ be a connected, undirected, weighted graph. Let $A\subseteq E$ such that $A$ is included in some MST of $G$. Let $C = (V_C, E_C)$ be a connected component in the forest $G_A = (V,A)$. If $(u,v)$ is a light edge connecting $C$ to some other component in $G_A$, then $(u,v)$ is safe for $A$
\end{corollary*}


\subsection*{23.2 Kruskal and Prim's algorithms $O(E\lg V)$}


\begin{defn*}
    \textbf{Kruskal's algorithm} Finds a safe edge to the growing forest by finding, of all edges that connect any two trees in the forest, an edge $(u,v)$ of least weight.
    \begin{enumerate}
        \item \textbf{Implementation} Needs a fast way to determine if an edge crosses connected components. Tracks trees in disjoint sets. Initializes vertices to disjoint sets with \textsc{Make-Set}. Sort edges by weight in nondecreasing order. Loop over all edges and include edge $(u,v)$ to $A\subseteq E$ if $u$ and $v$ are not in the same set (evaluate with \textsc{Find-Set}). Update disjoint sets with \textsc{UNION}
        \item \textbf{Complexity} Assume disjoint-set-forest impl with union-by-rank and path-compression. Sorting takes $O(E\lg E)$. $O(E)$ \textsc{Find-Set} and \textsc{UNION} and $O(V)$ \textsc{Make-Set} takes a total of $O((V+E)\alpha(V))$. Sicne $G$ connected, $|E| \geq |V| - 1$, so disjoint-set operation takes $O(E\alpha(V)) = O(E\lg V) = O(E\lg E)$. In total, algorihtm takes $O(E\lg E)$. Note since $|E| < |V|^2$, $\lg |E| = O(\lg V)$, so running time is $O(E\lg V)$
    \end{enumerate}
\end{defn*}



\begin{defn*}
    \textbf{Prim's algorithm} Tree ($A$) starts from an arbitrary root vertex $r$ and grows until the tree spans all vertices of $V$. Each step adds to the tree $A$ a light edge that connects $A$ to an isolated vertex, one on which no edge of $A$ is incident. (so that the cut respects $A$)
    \begin{enumerate}
        \item \textbf{Implementation} Needs a fast way to select a new edge to add to tree. vertices not in the tree reside in a min-priority queue $Q$ based on $key$ attributes, where $v.key$ is the minimum weight of any edge connecting $v$ to a vertex in the tree $A$. ($v.key = \infty$ if no such edge exists) 
        \item \textbf{Complexity} Assume $Q$ impl with binary min-heap. building heap requires $O(\lg V)$ time. $O(V)$ \textsc{Extract-Min} each taking $O(\lg V)$ amounts to $O(V\lg V)$. While loop iterates $O(E)$ times. The test for membership is $O(1)$ by keeping a bit in $G$ for each vertex and tells if its not in $Q$ and updating the bit once vertex is removed from $Q$. \textsc{Decrease-Key} taking $O(\lg V)$ each. Hence total time is $O(V\lg V + E\lg V) = O(E\lg V)$
    \end{enumerate}
\end{defn*}

\section*{24 Single-Source Shortest Path}


\begin{defn*}
    \textbf{The Single-Paths Problem} Given a weighted, directedf graph $G = (V,E)$, with $w: E\to\R$.

    \begin{enumerate}
        \item The \textbf{weight of path} $w(p)$ for $p = \langle v_0, \cdots, v_k \rangle$ is given by
        \[
            w(p) = \sum_{i=1}^k w(v_{i-1}, v_i)   
        \] 
        \item A \textbf{Shotest-path weight} $\delta(u,v)$ from $u$ to $v$ is given by 
        \[
            \delta(u,v) = 
            \begin{cases}
                min\{w(p): u\overset{p}{\leadsto} v \} & \text{if there is a path from $u$ to $v$}\\
                \infty & \text{otherwise}
            \end{cases}
        \]
        \item A \textbf{Shotest path} from $u$ to $v$ is defined as any path $p$ with weight 
        \[
            w(p) = \delta(u,v)
        \]
    \end{enumerate}

\end{defn*}


\begin{defn*}
    \textbf{Variants}
    \begin{enumerate}
        \item \textbf{Single-source shortest-path problem} Find the shortest path from a given \textbf{source} vertex $s\in V$ to each vertex $v\in V$
        \item \textbf{Single-destination shortest-path problem} Find a shortest path to a given \textbf{destination} vertex $t$ from each vertex $v\in V$. (By reversing direction of each edge, we can reduce this problem to a single-source problem)
        \item \textbf{Single-Pair shortest-path problem} Find a shortest path from $u$ to $v$ for given vertices $u$ and $v$. (If we solve single-source problem with source vertex $u$, we solve this problem also)
        \item \textbf{All-pairs shortet-path problem} Find a shortest path from $u$ to $v$ for every pair of vertices $u$ and $v$. (solving by single-source algo will be inefficient, there are better solutions)
    \end{enumerate}
\end{defn*}




\begin{proposition*}
    \textbf{Subpaths of shortest paths are shortest path (Optimal substructure)} Given a weighted, directed graph $G = (V,E)$ with weight function $w: E\to \R$, let $p = \langle v_0,\cdots, v_k \rangle$ be a shortest path from vertex $v_0$ to vertex $v_k$ and, for any $i$ and $j$ such that $0 \leq i \leq j \leq k$, let $p_{ij} = \langle v_i, v_{i+1}, \cdots, v_j \rangle$ be a subpath of $p$ from vertex $v_i$ to vertex $v_j$. Then $p_{ij}$ is a shortest path from $v_i$ to $v_j$
\end{proposition*}


\begin{proposition*}
    \textbf{Cycles} 
    \begin{enumerate}
        \item \textbf{Negative-weight cycle} 
        \begin{enumerate}
            \item The shortest path cannot contain negative-weight cycles.
            \item No path from $s$ to a vertex on a cycle can be a shortest path, since we can always find a path with lower weight by following the proposed shortest path and then traversing the negative-weight cycle. 
            \item Hence we define for all $v \in C$ for some negative-weight cycle, $\delta(s,v) = \infty$
        \end{enumerate}
        \item \textbf{Positive-weight cycle} 
        \begin{enumerate}
            \item The shortest path cannot contain positive-weight cycle, 
            \item since removing the cycle from the path produces a path with the same source and destination vertices and a lower path weight
            \item For 0-weight cycles, we can always remove the cycle and get a shortest path without a cycle. 
            \item Hence we assume shortest paths have no cycles (simple path). Since any acyclic path in $G$ has at most $|V|$ distinct vertices, it contains at most $|V| - 1$ edges, hence we try to find shortest path of at most $|V| - 1$ edges
        \end{enumerate}
    \end{enumerate}
\end{proposition*}


\begin{defn*}
    \textbf{Representing shortest path} Interested in predecessor subgraph $G_{\pi} = (V_{\pi}, E_{\pi})$ where
    \[
        V_{\pi} = \{ v\in V: v.\pi \neq NIL\} \quad \quad E_{\pi} = \{ (v.\pi, v): v\in V_{\pi} - \{s\}\}
    \]
    Specifically, let $G = (V,E)$ be a weighted,directed graph with weight function $w:E\to \R$ and assume $G$ contains no negative-weight cycles reachable from source vertex $s\in V$, so that shortest path are well-defined. A \textbf{Shortest-paths tree} rooted at $s$ is a directed subgraph $G' = (V', E')$, where $V' \subseteq V$ and $E' \subseteq E$ such that 
    \begin{enumerate}
        \item $V'$ is the set of vertices reachable from $s$ in $G$
        \item $G'$ forms a rooted tree with root $s$ and 
        \item for all $v\in V'$,the unique simple path from $s$ to $v$ in $G'$ is a shortest path from $s$ to $v$ in $G$
    \end{enumerate}
    Note shortest path or shortest path are not necessarily unique
\end{defn*}


\begin{proposition*}
    \textbf{Shortest-path estimate and Relaxation}
    \begin{enumerate}
        \item \textbf{Shortest-path estimate} For each $v\in V$, the shortest-path estimate $v.d$ is an upper bound on the weight of a shortest path from source $s$ to $v$. 
        \item \textbf{Relaxation} The process of relaxing an edge $(u,v)$ consists of testing whether we can improve the shortest path to $v$ found so far by going through $u$ and, if so, updating (improving) $v.d$ and $v.\pi$
        \begin{enumerate}
            \item Given $u.d$, $v.d$ and $w(u,v)$ for edge $u \to v$ 
            \item Update $v.d$ if $u.d + w(u,v) < v.d$. In essence, take path along $u$ instead of some other path
            \item Only way to change $v.d$ and $v.\pi$
        \end{enumerate}
        \item \textbf{Triangular Inequality (for weighted graphs)} For any edge $(u,v) \in E$ we have 
        \[
            \delta(s,v) \leq \delta(s,u) + w(u,v)
        \]
        \begin{proof}
            Suppose $p$ where $s\overset{p}{\leadsto} v$ is a shortest path, then claim holds by definition of shortest path. Otherwise, there is no shortest path from $s$ to $v$. This implies that there is no shortest path to from $s$ to $u$, since otherwise there exists shortest path $p'$ such that  $s \overset{p'}{\leadsto} u \to v$ which is a shortest path. Hence $\delta(s,v) = \delta(s,u)$ are either $\infty$ or $-\infty$. Hence the claim holds 
        \end{proof}
    \end{enumerate}
\end{proposition*}

\begin{proposition*}
    \textbf{Effect of relaxation on shortest-path estimates}
    \begin{enumerate}
        \item \textbf{Upper-bound property} Let $G = (V,E)$ be weighted, directed graph with $w$. Let $s\in V$ be source vertex and graph initialized by \textsc{Initialize-Single-Souce}$(G,s)$ then $v.d\geq \delta(s,v)$ for all $v\in V$ over any sequenece of relaxation steps on edges of $G$. Moreover, once $v.d$ achieves value $\delta(s,v)$, it never changes.
        \begin{proof}
            Prove by induction the claim $v.d \geq \delta(s,v)$ holds for all $v\in V$ on the number of relaxation steps. After initialization, $\infty = v.d \geq \delta(s,v)$ holds for all $v\in V \setminus \{ s \}$, and since $s.d = 0 \geq \delta(s,s)$. For inductive step, we have that $x.d \geq \delta(s,x)$ for all $x\in V$. Assume we relax an edge $(u,v)$, only $v.d$ will be changed 
            \[
                v.d = u.d + w(u,v) \geq \delta(s, u) + w(u, v) \geq \delta(s,v)
            \]
            by I.H. and triangular inequality. In addition $v.d$ never change once $v.d = \delta(s,v)$. This is because $v.d$ never decreases as $v.d \geq \delta(s,v)$ holds for all $v\in V$ just proved and no operation increases $v.d$
        \end{proof}
        \item \textbf{No-path property} Given a weighted, directed graph $G = (V,E)$ with $w:E\to\R$ and there is no path from $s$ to $v$. Then after graph is initialized by $\textsc{Initialize-Single-Source}(G,s)$, we have $v.d = \delta(s,v) = \infty$ and this equality is maintained as an invariant over any sequence of relaxation steps on edges of $G$
        \begin{proof}
            By definition of shortest path, $\delta(s,v) = \infty$ as there is no path from $s$ to $v$. By upper-bound property $v.d \geq \delta(s,v) = \infty$, hence $v.d = \delta(s,v) = \infty$
        \end{proof}
        \begin{lemma*}
            Let $(u,v) \in E$, then immediately after relaxing edge $(u,v)$ by executing $\textsc{Relax}(u, v, w)$, we have $v.d \leq u.d + w(u,v)$
            \begin{proof}
                If $v.d > u.d + w(u,v)$, then by $\textsc{Relax}$, $v.d = u.d + w(u,v)$. If $v.d\leq u.d + w(u,v)$, $v.d$ not updated. Hence $v.d \leq u.d + w(u,v)$ afterwards
            \end{proof}
        \end{lemma*}
        \item \textbf{Convergence property (Given $u.d = \delta(s,u)$, $v.d
        \xrightarrow{\textsc{Relax}(u,v)} \delta(s,v)$)} Let $G = (V,E)$ be weighted, directed graph with weight function $w: E\to\R$, let $s\in V$ be source vertex, and let $s \leadsto u \to v$ is a shortest path in $G$ for some $u,v \in V$. Suppose $G$ is initialized by $\textsc{Initialize-Single-Source}(G,s)$ and then execute a sequence of relaxation steps that includes the call $\textsc{Relax}(u,v,w)$ on edges of $G$. If $u.d = \delta(s,u)$ at any time prior to relaxing edge $(u,v)$, then $v.d = \delta(s,v)$ at all times afterwards
        \begin{proof}
            If $u.d =\delta(s,u)$ at any time prior to relaxing $(u,v)$, then by upper-bound property, $u.d = \delta(s,u)$ stays the same. After relaxation on $(u,v)$, by previous lemma
            \[
                v.d \leq u.d + w(u,v) = \delta(s,u) + w(u,v) = \delta(s,v)
            \]
            the last equality given by optimal substructure of shortest path. By upper-bound property, $v.d \geq \delta(s,v)$, hence $v.d = \delta(s,v)$ and this property is maintained afterwards
        \end{proof}
        \item \textbf{Path-relaxation property} Let $G = (V,E)$ be a weighted, directed graph with weight function $w:E\to\R$ and let $s\in V$ be source vertex. Consider any shortest path $p = \langle v_0, \cdots v_k \rangle$ from $s = v_0$ to $v_k$. If $G$ is initialized with $\textsc{Initialize-Single-Source}(G,s)$ and then a sequence of relaxation steps occurs that includes, in order, relaxing the edges $(v_0, v_1)$, $(v_1, v_2)$, $\cdots$, $(v_{k-1}, v_k)$, then $v_k.d = \delta(s, v_k)$ This property holds regardless of any other relaxation steps that occur, even if they are intermixed with relaxations of the edges of $p$
        \begin{proof}
            Proof by induction $v_i.d = \delta(s,v_i)$ holds on $i$-th vertex in $p$ relaxed. When $i = 0$, $v_0.d = s.d = 0 = \delta(s,s)$, by upper-bound property, the value never changes afterwards. In induction step, assume $v_{i-1}.d = \delta(s, v_{i-1})$. After relaxation of $(v_{i-1}, v_i)$, $v_i.d = \delta(s, v_i)$ by convergence property and the equality is maintained thereafter by upper-bound property
        \end{proof}
    \end{enumerate}
\end{proposition*}


\begin{proposition*}
    \textbf{Relaxation and Shortest-paths tree}
    \begin{enumerate}
        \item Let $G = (V,E)$ be a weighted, directed graph with $w:E\to\R$, let $s\in V$ be a source vertex, and assume $G$ contains no negative-weight cycles that are reachable from $s$. Then after the graph is initialized with $\textsc{Initialize-Single-Source}(G,s)$, the predecessor subgraph $G_{\pi}$ forms a \textbf{rooted tree} with root $s$, and any sequence of relaxation steps on edges of $G$ maintains this property as an invariant.
        \begin{proof}
            Proof consists of proving $G_{\pi}$ is an acyclic graph by contradiction (i.e. assume there is a cycle and prove the cycle is in fact a negative weight cycle, which contradicts assumption of the problem). Then proving the graph is rooted at $s$, i.e. proving there is unique simple path from $s$ to $v$ in $G_{\pi}$
        \end{proof}
        \item \textbf{Predecessor-subgraph property} Let $G = (V,E)$ be a weighted, directed graph with $w:E\to\R$, let $s\in V$ be a source vertex, and assume $G$ contains no negative-weight cycles that are reachable from $s$. Then after the graph is initialized with $\textsc{Initialize-Single-Source}(G,s)$ and execute any sequence of relaxation steps on edges of $G$ that produces $v.d = \delta(s,v)$ for all $v\in V$, then the predecessor subgraph $G_{\pi}$ is a shortest-path tree rooted at $s$.
        \begin{proof}
            Prove 3 properties of shortest-path trees given.
            \begin{enumerate}
                \item Prove $V_{\pi}$ is the set of vertices reachable from $s$. 
                Let $v\in V$ be not reachable from $s$, hence $\delta(s,v) = \infty$. Since $v.d$ and $v.{\pi}$ updated together in \textsc{Relax}, implying $v.{\pi} = NIL$ and hence $v\not\in V_{\pi}$ 
                \item Prove $G_{\pi}$ forms a rooted tree with root $s$, follows from previous proposition
                \item Prove for all $v\in V_{\pi}$, the unique simple path $s \overset{p}{\leadsto} v$ in $G_{\pi}$ is a shortest path from $s$ to $v$ in $G$. Let $p = \langle v_0, \cdots, v_k \rangle$ where $v_0 = s$ and $v_k = v$. For $ i = 1,\cdots, k$, we have $v_i.d = \delta(s,v_i)$ (Path-Relaxation property) and $v_i.d \geq v_{i-1}.d + w(v_{i-1}, v_i)$, hence $w(v_{i-1}, v_i) \leq \delta(s, v_i) - \delta(s, v_{i-1})$. Summing weights along $p$
                \[
                    w(p) = \sum_{i=1}^k w(v_{i-1}, v_1) = \sum_{i=1}^k (\delta(s, v_i) - \delta(s, v_{i-1})) = \delta(s,v_k) - \delta(s, v_0) = \delta(s, v_k)
                \]               
                hence $w(p) = \delta(s,v_k)$ and thus $p$ is a shortest path from $s$ to $v=v_k$
            \end{enumerate}
        \end{proof}
    \end{enumerate}
\end{proposition*}



\subsection*{24.2 Bellman-Ford algorithm $O(VE)$}


\begin{defn*}
    Bellman-Ford algorithm
    \begin{enumerate}
        \item \textbf{Goal} solves single-source shortest-paths problem in which edges may be negative 
        \begin{enumerate}
            \item returns a boolean indicating whether or not there is a negative-weight cycle that is reachable from source 
        \item and the shortest path and their weight is no such cycle exists 
        \end{enumerate}
        \item \textbf{Implementation} works by progressively decreasing estimate $v.d$ until it achieves $\delta(s,v)$ by making $|V| - 1$ passes, where in each pass, every $v\in V$ is relaxed once. 
        \item \textbf{Runtime} $O(VE)$, initialization $\Theta(V)$, each $|V| - 1$ passes takes $\Theta(E)$ times (since relax every $e\in E$ requires traversing the entire adjacency list).
    \end{enumerate}
\end{defn*}


\begin{lemma*}
    Let $G = (V,E)$ be weighted, directed graph with source $s$ and weight function $w:E\to\R$ asusme $G$ contains no negatie-weight cycles that are reachable from $s$. Then after $|V| - 1$ iterations of for loop in the algorithm, we have $v.d = \delta(s,v)$ for all vertices $v$ that are reachable from $s$
    \begin{proof}
        Let $v\in V$ be arbitrary vertices reachable from $s$, let $p = \langle v_0 = s,\cdots, v_k = v$ be any shortest path from $s$ to $v$. Since shortest path are simple, there are at most $|V| - 1$ edges. So $k \leq |V| - 1$. Since each of $|V| - 1$ iterations relax all $|E|$ edges, amongst them is the edge $(v_{i-1}, v_i)$. By path-relaxation property, $v.d = v_k.d = \delta(s,v_k) = \delta(s,v)$
    \end{proof}
\end{lemma*}


\begin{corollary*}
    Let $G = (V,E)$ be weighted, directed graph with soruce $s$ and weight function $w:E\to\R$ assume $G$ contains no negative-weight cycles that are \textbf{reachable from} $s$. For each $v\in V$, there is a path from $s$ to $v$ if and only if $\textsc{Bellman-Ford}$ terminates with $v.d < \infty$ when it is run on $G$
\end{corollary*}


\begin{theorem*}
    \textbf{Correctness of Bellman-Ford Algorithm} Let $\textsc{Bellman-Ford}$ be run on a weighted, directed graph $G = (V,E)$ with source $s$ and weight $w: E\to\R$ If $G$ contains no negative-weight cycles that are reachable from $S$, then the algorithm returns \textsc{True},we have $v.d = \delta(s,v)$ for all vertices $v\in V$,and the predecessor subgraph $G_{\pi}$ is a shortest-paths tree rooted at $s$. If $G$ does contain a negative-weight cycle reachable from $s$, then the algorithm returns \textsc{False} 
    \begin{proof}
        Suppose $G$ contains \textbf{no negative-weight cycles}. Prove $v.d = \delta(s,v)$ for all vertices $v\in V$. If $v$ is reachable from $s$, prevous lemma proves the claim. Otherwise $v$ not reachable from $s$, then claim follows from no-path property, i.e. $v.d = \delta(s,v) = \infty$. The predecessor subgraph property, along with the claim, implies $G_{\pi}$ is shortest path tree. Now prove the algorithm returns \textsc{True}. At termination, for all $v\in V$
        \[
            v.d = \delta(s,v) \leq \delta(s,u) + w(u,v) = u.d + w(u,v)
        \]
        so none of test for negative cycle in the algorithm returns \textsc{False} hence will return \textsc{True}. Suppose $G$ has negative cycles reachable from $s$, let $c = \langle v_0, \cdots, v_k \rangle$, where $v_0 = v_k$, then 
        \[
            \sum_{i=1}^k w(v_{i-1} , v_i) < 0
        \]
        Prove by contradiction, if the algorithm returns \textsc{True}, then we have 
        \[
            v_i.d \leq v_{i-1}.d + w(v_{i-1}, v_i)
        \]
        for $i = 1,\cdots, k$. Summing equalities around the cycle 
        \[
            \sum_{i=1}^k v_i.d \leq \sum_{i=1}^k (v_{i-1}.d + w(v_{i-1}, v_i)) = \sum_{i=1}^k v_{i-1}.d + \sum_{i=1}^k w(v_{i-1}, v_i)
        \]
        Note $\sum_{i=1}^k v_{i}.d = \sum_{i=1}^k v_{i-1}.d$ since cycles hold the same vertices despite difference in the way they are indexed. Note $v_i.d$ is finite, so 
        \[
             \sum_{i=1}^k w(v_{i-1}, v_i) \geq 0
        \]
        which contradicts the negative cycle assumption. 
    \end{proof}
\end{theorem*}




\subsection*{24.3 Single-Source shortest paths in directed acyclic graphs $O(V+E)$}


\begin{defn*}
    \textbf{DAG Shortest-Path} 
    \begin{enumerate}
        \item \textbf{Motivation} Increase runtime by relaxing edges according to a topological sort of its vertices (so that we can use path-relaxation property and only relax every edge once)
        \item \textbf{Implementation}
        \begin{enumerate}
            \item Topologically sort the dag, i.e. if there is $p$ such that $u \xrightarrow{p} v$, then $u$ precedes $v$
            \item Then make one pass over vertices in topologically sorted order. Relax each edge that leaves the vertex
        \end{enumerate}
        \item \textbf{Runtime} $O(V+E)$. Topological sort takes $\Theta(V+E)$ time, \textsc{Initialize-Single-Source} takes $\Theta(V)$. There is 1 pass where each edge is relaxed exactly once, each taking $O(1)$, hence amounts to $\Theta(V+E)$
    \end{enumerate}
\end{defn*}



\begin{theorem*}
    \textbf{Correctness of DAG Shortest-Path algorithm} If a weighted, directed graph $G = (V,E)$ has source vertex $s$ and no cycles, then at termination of \textsc{DAG-Shortest-Paths} procedure, $v.d = \delta(s,v)$ for all vertices $v\in V$, and the predecessor subgraph $G_{\pi}$ is a shortest-path tree

    \begin{proof}
        Show $v.d = \delta(s,v)$ for all $v\in V$ at termination. If $v$ not reachable from $s$, $v.d = \delta(s,v) = \infty$ by no-path property. If $v$ is reachable from $s$, then there is a shortest path $p = \langle v_0 =s,\cdots, v_k \rangle =v$. Because we process vertices in topological sorted order, the edges are relaxed in order. The path-relaxation property implies $v_i = \delta(s, v_i)$ at termination. The predecessor subgraph property implies $G_{\pi}$ is a shortest path tree
    \end{proof}
\end{theorem*}


\subsection*{24.4 Dijstra's Algorithm $O(V^2)$ or $O(E\lg V)$}

\begin{defn*}
    \textbf{Dijkstra's algorithm}
    \begin{enumerate}
        \item \textbf{Use case} Solves single-source shortest-paths problem on a weighted, directed graph $G = (V,E)$ for the case in which all edges weights are nonnegative, i.e. $w(u,v) \geq 0$ for all $(u,v)\in E$
        \item \textbf{Implementation} 
        \begin{enumerate}
            \item Maintains set $S$ of vertices whose final shortest-path weights from $s$ have already been determined.
            \item Repeated selects a vertex $u\in V\setminus S = Q$, implemented with min-priority queue, with minimum shortest-path estimate. 
            \item Adds $u$ to $S$
            \item Relax all edges leaving $u$, i.e. $Adj[u]$
        \end{enumerate}
        \item \textbf{Greedy} Since it chooses the lightest/closes vertex in $V\setminus S$ to add to set $S$
        \item \textbf{Analysis} \textbf{Min-priority queue} \textsc{Insert} \textsc{Extract-Min} called once per vertex, since each $u\in V$ added to $S$ exactly once. The loop iterates $|E|$, size of adjacency list, and \textsc{Decrease-Key} is called at most once per loop (in \textsc{Relax}). The runtime depends on how min-priority queue is implemented 
        \begin{enumerate}
            \item \textbf{Array} \textsc{Insert} and \textsc{Decrease-Key} $O(1)$, \textsc{Extract-Min} $O(V)$ (have to go through entire array) total time $O(V^2 + E) = O(V^2)$
            \item \textbf{binary min-heap} \textsc{Insert}, \textsc{Decrease-Key} and \textsc{Extract-Min} take $O(\lg n)$. Total runtime $O((V+E)\lg V)$, which is $O(E\lg V)$ if all vertices are reachable from source. Good if graph is sparce
            \item \textbf{Fibonacci heap} $O(V\lg V + E)$
        \end{enumerate}
        \item \textbf{Comparison} Both Dijkstra's and Prim's algorithm uses a min-priority queue and grow the tree from source $s$, while updating other vertices
    \end{enumerate}
\end{defn*}


\begin{theorem*}
    \textbf{Correctness of Dijkstra's algorithm} Dijkstra's algorithm run on a weighted, directed graph $G = (V,E)$ with nonnegative weight $w$ and source $s$, terminates with $u.d = \delta(s,u)$ for all vertices $u\in V$.
\end{theorem*}

\begin{proposition*}
    The Loop invariant
    \begin{center}
        At the start of each iteration, $v.d = \delta(s,v)$ for all vertex $v\in S$
    \end{center}
    Its enough to show for each vertex $u\in V$, $u.d =\delta(s,u)$ at time when $u$ is added to the set, The upper-bound guarantees $u.d = \delta(s,u)$ holds afterwards
    \begin{proof}
    Prove algo correct by proving invariant holds 
    \begin{enumerate}
        \item \textbf{Initialization} Initially, $S = \emptyset$, hence invariant trivially true.
        \item \textbf{Mainenance}  Now we prove $u.d = \delta(s,u)$ for $u$ added to $S$. Prove by contradiciton, let $u$ be first vertex added for which $u.d \neq \delta(s,u)$ when it was added to the set. Note $u \neq s$ since $s$ is first added with $s.d = \delta(s,s) = 0$, hence $S\neq \emptyset$. Also there must be some path connecting $s$ to $u$, otherwise $u.d \neq \delta(s,u) = \infty$ which violates assumption that $u.d \neq \delta(s,u)$. If there is a path, there is a shortest path, let $p$ be such path that connects $s\in S$ to $u\in V\setminus S$. Then at some point $p$ crosses the cut $(S, V\setminus S)$. Let $y \in V\setminus S$ be the first vertices and $x$ be $y$'s parent, i.e. $y.\pi = x$. Now we decompose $p$ 
        \[
            s \overset{p_1}{\leadsto} x \to y \overset{p_2}{\leadsto} u
        \]
        Now we claim $y.d = \delta(s,y)$ when $u$ is added to $S$. This is true because $u$ is the first vertex added to $S$ such that $u.d\neq \delta(s,u)$. Since $x\in S$, by I.H. we have $x.d = \delta(s,x)$ when $x$ was added to $S$. Then $(x,y)$ is relaxed at that time, hence the claim follows by convergence property. Now we obtain a contradiction, since $y$ comes before $u$ on a shortest path from $s$ to $u$ and all other edge in $p_2$ weights are non-negative, we have $\delta(s,y) \leq \delta(s,u)$, hence 
        \[
            y.d = \delta(s, y) \leq \delta(s,u) \leq u.d
        \]
        But since both $u$ and $y$ is in $V\setminus S$ when $u$ was chosen and we picked $u$ instead of $y$ hence $u.d \leq y.d$. The two inequalities yield a equality 
        \[
            y.d = \delta(s,y) = \delta(s,u) = u.d
        \]
        Hence $\delta(s, u) = u.d$ contradicts the choice of $u$. Hence $u.d = \delta(s,u)$ when it was first added to $S$. 
        \item \textbf{Termination} At termination $Q = V\setminus S = \emptyset$, hence $S = V$, hence by previous invariant, $u.d = \delta(s,u)$ for all $u\in V$
    \end{enumerate}
    \end{proof}
\end{proposition*}


\section*{25 All-Pairs Shortest Path} 


\begin{defn*}
    \textbf{All-Pairs shortest path}
    \begin{enumerate}
        \item \textbf{Goal} Given $G = (V,E)$ with weight $w:E\to\R$. Find, for every pair $u,v\in V$, a shortest(least weight) path from $u$ to $v$. Want to output in tabular form: each entry in $u$'s row and $v$'s column should be weight of a shortest path from $u$ to $v$
        \item \textbf{Naive solution} Run single-source shortest path algorithm $|V|$ times, once for each vertex as the source. Non-negative weight, use Dijkstra's algorithm, the min-heap impl of min-priority queue yields a runtime of $O(VE\lg V)$, fibonnaci heap impl yields runtime of $O(V^2 \lg V + VE) = O(V^3)$. If have non-negative weights have to use Bellman-Ford algorithm, with runtime of $O(V^2E)$, which is $O(V^4)$ if graph is dense.
        \item \textbf{Representation of Graph} Use matrix representation. Assume vertices numbered $1,2,\cdots, |V|$, an $n$-veretx directed $G =  (V,E)$ is represented as a $n\times n$ matrix $W = (w_{ij})$ representing edge weights. 
        \[
            w_{ij} = 
            \begin{cases}
                0 & \text{if } i = j\\
                \text{weight of directed edge (i,j)} & \text{if } i\neq j \land (i,j)\in E \\
                \infty & \text{if } i\neq j \land (i,j) \not\in E
            \end{cases}
        \]
        The all-pairs shortest-path algorithm outputs $n\times n$ matrix $D = (d_{ij})$, where $d_{ij} = \delta(i, j)$. A \textbf{Predecessor Matrix} $\Pi = (\pi_{ij})$, such that $\pi_{ij}$ is NIL if either $i = j$ or there is no path from $i$ to $j$, otherwise $\pi_{ij}$ is predecessor of $j$ on some shortest path from $i$. 
    \end{enumerate}
\end{defn*}


\subsection*{25.1 Shortest path and matrix multiplication with DP $O(V^3 \lg V)$ }


\begin{defn*}
    \textbf{Shortest path and matrix multiplication}
    \begin{enumerate}
        \item \textbf{Structure of shortest path} Given $W = (w_{ij})$, consider shortest path $p$ from $i$ to $j$, where $p$ has $m$ edges, assume no negative-weight cycles, and $m$ is finite. If $i=j$, $p$ has weight 0 and no edges. If $i\neq j$, then we can decompose $p$ into 
        \[
            i \overset{p'}{\leadsto} k \to j
        \]
        where path $p'$ now contains at most $m-1$ edges, By optimal substructure of shortest path, $p'$ is a shortest path from $i$ to $k$, and so $\delta(i,j) = \delta(i, k) + w_{kj}$
        \item \textbf{Recursive solution} Let $l_{ij}^{(m)}$ be the minimum weight of any path from vertex $i$ to vertex $j$ that contains at most $m$ edges. When $m = 0$, there is a shortest path from $i$ to $j$ with no edges if and only if $i = j$, hence 
        \[
            l_{ij}^{(0)} = 
            \begin{cases}
                0 & \text{if } i  = j\\
                \infty & \text{if } i\neq j \\
            \end{cases}
            \quad \quad 
            l_{ij}^{(m)} = Min\left\{ l_{ij}^{(m-1)}, \underset{1\leq k\leq n}{Min}\{ l_{ik}^{(m-1)} + w_{kj} \}  \right\} =  \underset{1\leq k\leq n}{Min}\{ l_{ik}^{(m-1)} + w_{kj} \} 
        \]
        The first term is the weight of a shortest path from $i$ to $j$ in potentially $m-1$ edges, the latter is the minimum weight of paths, where all possible predecessor $k$ of $j$ is explored. The latter simplification is because $w_{jj}= 0$. The actual shortest-path weights are given by
        \[
            \delta(i, j) = l_{ij}^{(n-1)} = l_{ij}^{(n)} = \cdots
        \]
        since a path from $i$ to $j$ with more than $n-1$ edges is not simple anymore and hence cannot have a lower weight than a shortest path from $i$ to $j$ in under $n-1$ edges
        \item \textbf{Bottom Up approach}
        The algorithm computes a series of matrices $W = L^{(1)}, L^{(2)}, \cdots, L^{(n-1)}$ for $m = 1,\cdots, n-1$ and $L^{(m)} = (l_{ij}^{(m)})$ and the final matrix $L^{(n-1)}$ contains the shortest path weights. Requires 3 nested for loop, hence runtime of $O(n^3)$. The procedure is very much similar to matrix multiplication, where 
        \[
            c_{ij} = \sum_{k} a_{ij} \cdot b_{kj}
        \]
        We have $L^{(m)} = L^{(m-1)}\cdot W$ where $\cdot$ represent taking mins instead... The procedure \textsc{Extend-Shortest-Paths} is run $n-1$ times to yield $L^{(n-1)}$ hence the total runtime amounts to $\Theta(n^4)$. 
        \item \textbf{Improvement in runtime} To improve the runtime, we notice that the matrix operation is associative and hence we can compute $L^{(n-1)}$ in $\lceil \lg (n-1) \rceil$ by computing $L^{(m)}$ such that $m$ is a power of 2. And once we loop to a point where $m \geq n-1$, we have $L^{(m)} = L^{(n-1)}$ as $\delta(i, j) = l_{ij}^{(n-1)} = l_{ij}^{(n)} = \cdots$. The total runtime is improved to $O(n^3 \lg n) = O(V^3 \lg V)$. The improvement lies in the fact that since there is no elaborate data structure, constant hidden in $\Theta$ is therefore small
    \end{enumerate}
\end{defn*}



\subsection*{The FLoyd-Warshall algorithm $\Theta(V^3)$}


\begin{defn*}
    \textbf{Structure of a shortest path} 
    \begin{enumerate}
        \item \textbf{Concepts} Consider \textbf{intermediate vertices} of a shortest path $p = \langle v_1, \cdots, v_l \rangle$ is the set $\{ v_2,\cdots, v_{l-1}\}$
        \item \textbf{Observation} Assume $V = \{ 1, 2, \cdots, n\}$ For some subset $\{ 1, 2, \cdots, k\}\subseteq V$. Let $i,j \in V$ and $p$ be a minimum-weight path from $i$ to $j$ with all intermediate vertices in $\{ 2, \cdots, k-1\}$. 
        \begin{enumerate}
            \item If $k$ is not an intermediate vertex of $p$, The shortest path $p$ with all intermediate vertices in $\{ 1, \cdots, k\}$ is also in $\{ 1, \cdots, k-1\}$
            \item If $k$ is an intermediate vertex of $p$, then decompose $p$
            \[
                i\overset{p_1}{\leadsto} k \overset{p_2}{\leadsto} j
            \]
            By optimal substructure of shortest path, $p_1$ is a shortest path from $i$ to $k$ with all intermediate vertices in $\{ 1, 2, \cdots, k \}$. Since $k$ is not an intermediate vertex, all intermediate vertices of $p_1$ are in $\{1,2,\cdots, k-1  \}$. Hence 
            \begin{center}
                $p_1$ is a shortest path from $i$ to $k$ with all intermediate vertices in the set $\{ 1,2,\cdots, k-1 \}$; Similarly, $p_2$ is a shortest path from $k$ to $j$ with all intermediate vertices in the set $\{ 1,2,\cdots, k-1 \}$
            \end{center}
        \end{enumerate}
        \item \textbf{Recursive solution} Let $d_{ij}^{(k)}$ be weight of a shortest path from $i$ to $j$ for which all intermediate vertices are in the set $\{1,2,\cdots, k \}$. Note when $k=1$, the set $\{ 1, 0 \}$ has no intermediate vertex and includes $i$ and $j$ respectively and has one edge $(i,j)$
        \[
            d_{ij}^{(k)} = 
            \begin{cases}
                w_{ij}  & \text{if } k = 0\\
                Min\left\{d_{ij}^{(k-1)}, d_{ik}^{(k-1)} + d_{kj}^{(k-1)}\right\} & \text{if } k\geq 1  
            \end{cases}
        \]
        Hence $D^{(n)} = (d_{ij}^{n})$ gives the right answer since all intermediate sets are in $\{1,\cdots,n \}$. So 
        \[
            d_{ij}^{(n)} = \delta(i,j) \quad \text{for all } i,j\in V
        \]
        \item \textbf{Bottom Up Approach} Runtime $O(V^3)$ because of the triple for loop, each taking $O(1)$ to look up previously computed values and calculate the minimum. Again, the code is tight, and so constat hidden in $\Theta$ notation is small
        \item \textbf{Constructing shortest path $\Pi$}
        \begin{enumerate}
            \item from $D$ of shortest path weights after computing $D$
            \item at the same time $D$ is calculated 
        \end{enumerate} 
    \end{enumerate}
    
\end{defn*}


\begin{defn*}
    \textbf{Transitive Closure of a directed graph} $G^* = (V, E^*)$ where 
    \[
        E^* = \{(i,j):\quad \text{ there is a path from vertex $i$ to $j$ in $G$} \}
    \]
    \textbf{Solutions}
    \begin{enumerate}
        \item We can compute transitive closure by assign weight of 1 to each edge in $E$ and run Floyd-Warshall algorithm. So if $d_{ij} < n$ there is a path from $i$ to $j$ otherwise $d_{ij} = \infty$
        \item To save time and space we substitute logical operations $land$ and $lor$ with arithmetic operation in Floyd-Warshall algorithm Define $t_{ij}^k$ be 1 if there is a path from $i$ to $j$ with all intermediate set in $G$ and 0 otherwise. 
        \[
            t_{ij}^{(0)} = 
            \begin{cases}
                0 & \text{ if $i\neq j$ and } (i,j)\not\in E \\
                1 & \text{ if $i = j$ or } (i,j)\in E\\
            \end{cases}
            \quad\quad 
            t_{ij} = t_{ij}^{(k-1)} \lor ( t_{ik}^{(k-1)} \land t_{kj}^{(k-1)})
        \]
        Then compute $T^{(k)} = (t_{ij}^{(k)})$ in order of increasing $k$ (bottom up). The runtime is $\Theta(n^3)$, same as previous algorithm. But is quite faster and memory efficient since operates on bits (logical) instead of on integer words (arithmetic)  
    \end{enumerate}
\end{defn*}


\section*{Maximum Flow}


\subsection*{26.1 Flow Networks}

\begin{defn*}
    \textbf{Flow networks}
    \begin{enumerate}
        \item \textbf{Flow network} A flow network $G = (V,E)$ is a directed graph in which 
        \begin{enumerate}
            \item each edge $(u,v) \in E$ has a nonnegative \textbf{capacity} $c(u,v) \geq 0$.
            \item if $(u,v) \in E$, then the edge in reverse direction $(v,u) \not\in E$
            \item if $(u,v) \not\in E$, then $c(u,v) = 0$
            \item No self-loops
            \item \textbf{source} $s$ and a \textbf{sink} $t$
            \item Assume each vertex lies on some path from $s$ to $t$, i.e. for all $v\in V$, we have $s \leadsto v \leadsto t$
            \item $|E| \geq |V| - 1$ since each vertex other than $s$ has at least one entering edge
        \end{enumerate}
        \item \textbf{Flow} Let $G =  (V,E)$ be flow network with capacity function $c$. A flow in $G$ is a real-valued funtion $f: V\times V \to \R$ satisfying
        \begin{enumerate}
            \item \textbf{Capacity Constraint} For all $u,v \in V$, we have $0 \leq f(u,v) \leq c(u,v)$
             \item \textbf{Flow Conservation} For all $u\in V \setminus \{ s,t\}$, we have 
             \[
                \sum_{v\in V} f(v,u) = \sum_{v\in V} f(u,v)
             \]
        \end{enumerate}
        If $(u,v) \not\in E$, then no flow from $u$ to $v$ and $f(u,v) = 0$. Denote $f(u,v)$ the flow from vertex $u$ to $v$. The \textbf{value of $|f|$ of a flow $f$} is defined as difference between total flow out of source and total flow into sink
        \[
            |f| = \sum_{v\in V} f(s,v) - \sum_{v\in V} f(v,s)
        \]
        as a typical flow network does not have edges into source $s$ we have 
        \[
            |f| =  \sum_{v\in V} f(s,v)
        \]
        \item \textbf{Maximum-Flow problem} Given flow network $G$ with source $s$ and sink $t$, find a flow $f$ of maximum value $|f|$
        \item \textbf{Transformation to flow network}
        \begin{enumerate}
            \item \textbf{Antiparallel edges} An antiparallel edge is the pair $(v_1, v_2)$ and $(v_2, v_1)$, which violates flow network. We can transform such graph into a flow network by taking one edge and decompose into 2 edges with an additional intermediate vertex, while set bot new edges' capacity constraint to the original edge. The two graphs are equivalent 
            \item \textbf{Multiple sources and sinks} Add a \textbf{supersource} $s$ and directed edge $(s, s_i)$ with capacity $c(c, c_i) = \infty$ for each $i = 1,\cdots n$ and likewise add a \textbf{supersink} $t$ with directed edge $(t_i, t$ with capacity $c(t_i, t) = \infty$. In other words, provided unlimited flow as desired for multiple sources $s_i$ and sinks $t_i$. The two graphs are equivalent 
        \end{enumerate}
    \end{enumerate}
\end{defn*}


\subsection*{26.2 The Ford-Fulkerson Method} 


\begin{defn*}
    \textbf{General Steps}
    \begin{enumerate}
        \item let $f(u,v) = 0$ for all $u,v\in V$
        \item At each step, increase flow value in $G$ by finding an \textbf{augmenting path} in an associated \textbf{residual network} $G_f$
        \item Repeat until the residue network has no more augmenting paths
    \end{enumerate}
\end{defn*}

\begin{defn*}
    \textbf{Residual Network}
    \begin{enumerate}
        \item \textbf{General Idea} $G_f$ consists of edges with capacities that represent how we can change the flow on edges of $G$. 
        \begin{enumerate}
            \item An edge $(u,v)$ of $G$ can admit $c(u,v) - f(u,v) = c_f(u,v)$ amount of additional flow (if edge has flow equal to capacity then $c_f(u,v) = 0$)
            \item An edge $(u,v)$ of $G$ can also reduce their flow by an amount up to $f(u,v) = c_f(v,u)$. The edge $(v,u)$ placed in $G_f$ is able to admit flow in opposite direction to $(u,v)$, at most cancelling out the flow on $(u,v)$
        \end{enumerate}
        \item \textbf{Residual Capacity} Given flow network $G$ and a flow $f$. Consider $u,v\in V$, the residual capacity $c_f(u,v)$ is defined by 
        \[
            c_f(u,v) = 
            \begin{cases}
                c(u,v) - f(u,v) & \text{if } (u,v) \in E    \\
                f(v,u) & \text{if } (v,u) \in E             \\
                0 & \text{otherwise}\\
            \end{cases}
        \]
        Note since flow network disallows antiparallel edges, exactly one of the cases applies
        \item \textbf{Residual Network} Given flow network $G$ and flow $f$, the residual network of $G$ induced by $f$ is $G_f = (V, E_f)$ where 
        \[
            E_f = \{ (u,v)\in V\times V : c_f(u,v) > 0 \}
        \]
        \item \textbf{Residual Edge} Edges in residual network is called residual edge $E_f$, which can be either edges in $E$, their reversal, or both 
        \[
            |E_f| \leq 2|E|
        \]
        \item \textbf{Augmentation} If $f$ is a flow in $G$ and $f'$ is a flow in corresponding residual network $G_f$, then $f \uparrow f'$, the augmentation of flow $f$ by $f'$, to be a function $(f \uparrow f'): V\times V\to \R$
        \[
            (f \uparrow f')(u,v) = 
            \begin{cases}
                f(u,v) + f'(u,v) - f'(v,u) & \text{if } (u,v) \in E\\
                0 & \text{otherwise}
            \end{cases}
        \]
        The idea is we increase flow on $(u,v) \in V$ by $f'(u,v)$ but decrease it bey $f'(v,u)$ because pushing flow on reverse edge in residual network signifies decreasing the flow in the original network, this is called \textbf{cancellation}
        \begin{lemma*}
            Let $G=(V,E)$ be flow network with source $s$ and sink $t$, let $f$ be a flow in $G$. Let $G_f$ be residual network of $G$ induced by $f$, and $f'$ be a flow in $G_f$. Then $f \uparrow f'$ is a flow in $G$ with value $|f \uparrow f'| = |f| + |f'|$
        \end{lemma*}
    \end{enumerate}
\end{defn*}




\begin{defn*}
    \textbf{Augmenting Paths (Improves value of flow)}
    \begin{enumerate}
        \item \textbf{Augmenting Path} Given flow network $G$ and a flow $f$, an augmenting path $p$ is a simple path from $s$ to $t$ in the residual network $G_f$. 
        \item \textbf{Residual Capacity of an Augmenting Path} The maximum amount by which we can increase the flow on each edge in an augmenting path $p$ the residual capacity of $p$ (such that capacity constraint is satisifed in $G$)
        \[
            c_f(p) = Min\{c_f(u,v): (u,v) \text{ is on } p \}
        \]
        \item \textbf{Flow of an Augmenting Path} We get a flow of an augmenting path $p$ by assigning the residual capacity of $p$, i.e. $c_f(p)$, to every edge on the path $p$.  Define function $f_p: V\times V \to \R$ by 
        \[
            f_p(u,v) = 
            \begin{cases}
                c_f(p) & \text{if $(u,v)$ is on $p$}    \\
                0 & \text{otherwise}                    \\
            \end{cases}
        \]
        \begin{lemma*}
            $f_p$ is a flow in $G_f$ with $|f_p| = c_f(p) > 0$
        \end{lemma*}
        \item If we augment $f$ by $f_p$, i.e. $f\uparrow f_p$, we get another flow in $G$ whose value is closer to the maximum 

        \begin{corollary*}
            Let $G = (V, E)$ be a flow network, let $f$ be a flow in $G$, and let $p$ be an augmenting path in $G_f$. Let $f_p$ be defined as previously, then the function $f\uparrow f_p$ is a flow in $G$ with value 
            \[
                |f \uparrow f_p | = |f| + |f_p| > |f|
            \]
            \begin{proof}
                Follows from $|f_p| = c_f(p) > 0$ and $|f \uparrow f'| = |f| + |f'|$
            \end{proof}
        \end{corollary*}
    \end{enumerate}
\end{defn*}


\begin{defn*}
    \textbf{Cut of Flow Networks (Determines when max flow is found)} 
    \begin{enumerate}
        \item \textbf{Cut} A cut $(S,T)$ of a flow network $G = (V,E)$ is a partition of $V$ into $S$ and $T = V \setminus S$ such that $s\in S$ and $t\in T$
        \item \textbf{Net Flow across a cut} If $f$ is a flow, then the net flow $f(S,T)$ across the cut $(S,T)$ is defined to be 
        \[
            f(S,T) = \sum_{u\in S} \sum_{v\in T} f(u,v) - \sum_{u\in S}\sum_{v\in T} f(v,u)
        \]
        Note \textbf{value of flow} $|f|$ is the net flow across cut $(\{s\}, V\setminus \{s\})$
        \item \textbf{Capacity of a cut} The capacity of cut $(S,T)$ is 
        \[
            c(S,T) = \sum_{u\in S} \sum_{v\in T} c(u,v)
        \]
        Note we only consider flow from $S$ to $T$, ignoring edges in the reverse direction (different from flow which considers both directions)
        \item \textbf{Minimum Cut} The minimum cut of a network is a cut whose capacity is minimum over all cuts of the network
    \end{enumerate}
\end{defn*}

\begin{lemma*}
    Let $f$ be a flow in a flow network $G$ with source $s$ and sink $t$, and let $(S,T)$ be any cut of $G$. Then the net flow across $(S,T)$ is $f(S,T) = |f|$
\end{lemma*}

\begin{corollary*}
    The value of any flow $f$ in a flow network $G$ is bounded from above by the capacity of any cut of $G$. (Implies that optimal $|f|$ is minimum capacity of all cuts in $G$)
    \begin{proof}
        Let $(S,T)$ be any cut of $G$ and $f$ be any flow. By previous lemma we have 
        \[
            |f| = f(S,T) = \sum_{u\in S} \sum_{v\in T} f(u,v) - \sum_{u\in S}\sum_{v\in T} f(v,u) \leq \sum_{u\in S} \sum_{v\in T} f(u,v) \leq \sum_{u\in S} \sum_{v\in T} c(u,v) = c(S,T)
        \]
    \end{proof}
\end{corollary*}

\begin{theorem*}
    \textbf{Max-flow Min-cut theorem} If $f$ is in a flow network $G = (V,E)$ with source $s$ and sink $t$, then the following conditions are equivalent 
    \begin{enumerate}
        \item $f$ is a maximum flow in $G$
        \item The residual network $G_f$ contains no augmenting paths 
        \item $|f| = c(S,T)$ for some cut $(S,T)$ of $G$
    \end{enumerate}

    \begin{proof}
        Prove 3 parts
        \begin{itemize}
            \item $(1) \to (2)$ Prove by contradiction. Assume $f$ is a maximum flow in $G$ but $G_f$ has an augmenting path $p$ with flow $f_p$. If we augment $f$ by $f_p$, we have $|f \uparrow f_p| = |f| + |f_p| > |f|$, implies there is a larger flow value, contradicting $f$ is the maximum flow
            \item $(2) \to (3)$ Idea is to identify cut $(S,T)$, infer value of $f(u,v)$ from the fact there exists no path from $s$ to $t$ in $G_f$, then calculate net flow $f(S,T)$ across an arbitrary cut, which is identical for any cut, including $|f|$. Assume $G_f$ has no augmenting path, that is there is no path from $s$ to $t$, Define 
            \[
                S = \{v\in V: \text{ there exists a path from $s$ to $v$ in $G_f$}  \}
            \]
            and $T = V \setminus S$. Consider vertices $u\in S$ and $v\in T$. $(u,v) \not\in E_f$ 
            \begin{enumerate}
                \item If $(u,v)\in E$, then $f(u,v) = c(u,v)$ since otherwise we have $c_f(u, v) = c(u,v) - f(u,v) > 0$, implying $(u,v)\in E_f$
                \item If $(v,u)\in E$, then $f(u,v) = 0$ since otherwise we have $c_f(u,v) = f(v,u) > 0$, implying $(u,v) \in E_f$
                \item If $(u,v)\not\in E$ or $(v,u)\not\in E$, hten $f(u,v) = f(v,u) = 0$
            \end{enumerate}
            Now we compute a net flow over the cut $(S,T)$ in $G$
            \[
                f(S,T) = \sum_{u\in S}\sum_{v\in T} f(u,v) - \sum_{v\in T}\sum_{u\in S} f(v,u) = \sum_{u\in S}\sum_{v\in T} c(u,v) - \sum_{v\in T}\sum_{u\in S} 0 = c(S,T)
            \]
            By previous corollary, net flow is same for all arbitrary cuts, we have 
            \[
                |f| = f(S,T) = c(S,T)
            \]
            \item $(3) \to (1)$ By previous corollary, the value of flow $|f|$ is bounded above by capacity of any cuts. $|f| \leq c(S,T)$. hence when $|f| = c(S,T)$ implies $f$ is a maximum flow
        \end{itemize}
    \end{proof}
\end{theorem*}


\begin{defn*}
    \textbf{Ford-Fulkerson algorithm $O(E|f^*|)$}  
    \begin{enumerate}
        \item \textbf{Steps}
        \begin{enumerate}
            \item Initialize $(u,v).f$ to 0
            \item Loop if there exists an augmenting path $p$ from $s$ to $t$ in residual network $G_f$ 
            \item Find residual capacity of the path $c_f(p) = Min\{c_f(u,v): (u,v) \text{ is in } p \}$ in $G_f$
            \item We replace $f$ with $f\uparrow f_p$ to obtain a new flow whose value is $|f| + |f_p|$
            \begin{enumerate}
                \item If $(u,v) \in E$, i.e. residual edge in $p$ is an edge in the original network, $(u,v)\in G_f$ specifies how much flow $(u,v)\in G$ can increase by, so add $c_f(p)$ amount of flow to $(u,v)\in G$
                \item If $(v, u) \in E$, i.e. residual edge in $p$ is a reverse edge in the original network, $(u,v)\in G_f$ specifies how much flow $(v,u) \in G$ can decrease by, so decrease $c_f(p)$ amount of flow to $(v,u)\in G$ 
            \end{enumerate}
        \end{enumerate}
        \item \textbf{Analysis} Runtime depends on finding the augmenting path $p$. 
        \begin{enumerate}
            \item \textbf{Initialization} $O(E)$
            \item \textbf{While loop} If capacity is rational, scale to integer. If $f^*$ is the max flow, \textsc{Ford-Fulkerson} executes while loop at most $|f^*|$ times,  since flow value increases by at least one unit in each iteration.
            \item \textbf{Finding path} Assume we have a data structure representing a directed graph $G' = (V,E')$ where $E' =\{(u,v): (u,v)\in E\lor (v,u)\in E \}$. The edges in $G_f$ consists off all edges $(u,v) \in E'$ such that $c_f(u,v) > 0$. If use DFS, or BFS, runtime $O(V + E') = O(E)$ (since $|E| \geq |V| -1$) for finding a path from $s$ to $t$. 
        \end{enumerate}
        In summary runtime of $O(E|f^*|)$. The algorithm is good if capacities are integral and the optimal flow value $|f^*|$ is small. 
    \end{enumerate}
\end{defn*}



\subsection*{26.3 Maximum Bipartite Matching $O(VE)$}

\begin{defn*} \textbf{Matching}
    \begin{enumerate}
        \item \textbf{Matching} Given undirected graph $G = (V,E)$, a matching is a subset of edges $M\subseteq E$ such that for all vertices $v\in V$, at most one edge of $M$ is incident on $v$.
        \item \textbf{Matched and Unmatched} a vertex $v\in V$ is matched by the matching $M$ if some edge in $M$ is incident on $v$; otherwise $v$ is unmatched
        \item \textbf{Maximum Matching} A maximum matching is a matching of maximum cardinality, that is, a matching $M$ such that for any matching $M'$, we have $|M | \geq |M'|$
        \item \textbf{Bipartite graphs} graphs in which $V$ can be partitioned into 2 disjoint sets $V = L\cup R$, $L\cap R = \emptyset$ and all edges in $E$ go between $L$ and $R$. Assume every vertex in $V$ has at least one incident edge
    \end{enumerate}
\end{defn*}


\begin{defn*}
    \textbf{Finding a Maximum Bipartite Matching}
    \begin{enumerate}
        \item \textbf{Corresponding Flow Network} $G' = (V', E')$ (directed) for a bipartite graph $G = (V,E)$ (undirected) with partition $V = L \cup R$ is defined as follows 
        \begin{enumerate}
            \item let source $s$ and sink $t$ be new vertices not in $V$
            \item let $V' = V \cup \{ s, t\}$
            \item let directed edge of $G'$ be edges of $E$, directed from $L$ to $R$, along with $|V|$ new directed edges connecting $s$ to $L$ and $R$ to $t$
            \[
                E' = \left\{ (s,u): u\in L \right\} \cup  \left\{ (u,v):(u,v)\in E \right\} \cup \left\{ (v,t): v\in R \right\} 
            \]
            Note $|E'| = \Theta(E)$
            \item assign unit capacity to each edge in $E'$ 
        \end{enumerate}
        
    \end{enumerate}
\end{defn*}


\end{document}

