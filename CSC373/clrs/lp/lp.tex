\documentclass[11pt]{article}
\input{/Users/markwang/.preamble}
\begin{document}

\section*{Linear Programming}


\begin{defn*}
    \textbf{General Linear Programs} 
    \begin{enumerate}
        \item \textbf{Linear Function} Given $a_1,\cdots, a_n\in \R$, and variables $x_1,\cdots,x_n$, define a linear function $f$ of those variables by 
        \[
            f(x_1,\cdots, x_n) = \sum_{j=1}^n a_j x_j
        \]
        \item \textbf{Linear equality \& inequalities} If $b\in \R$ and $f$ is a linear function, then 
        \[
            f(x_1, \cdots, x_n)  =b
        \]
        is a linear equality and the inequaliie 
        \[
            f(x_1,\cdots, x_n) \leq b \quad \quad f(x_1,\cdots, x_n) \geq b 
        \]
        are linear inequalities
        \item \textbf{Linear Constraints} are either linear equalities or linear inequalities 
        \item \textbf{Linear programming problem} Either minimizing or maximizing a linear fucntion subject to a finite set of linear constraints. If want to minimize, then linear program is a \textbf{minimization linear problem}, otherwise its called a \textbf{maximization linear problem}
        \item \textbf{Feasible solution} Any setting of variable $x_1,\cdots, x_n$ that satisifies all constraints a feasible solution to the linear program 
        \item \textbf{Feasible Region} a convex set of feasible solutions for which we which to maximize the objective function
        \item \textbf{Objective value} value of the objective function at a particular point in the feasible solution
        \item \textbf{Graphical solution} If 2 variables, then we can use the let $z$ be the objective. Such curve have the property that the intersection between the curve and the feasible solution is the set of feasible solutions with objective value $z$. A optimal solution to linear program occurs at a vertex of a feasible region, since the curve that intersect the feasible region for which maximum $z$ is obtained is on the boundary of the feasible region. This holds for higher dimension curves as well
        \item \textbf{Simplex} For $n$ variables, each constraint defines a half-space in $n$-dimensional space, the feasible region formed by the intersection of these half spaces is a simplex. The objective function is a hyperplane, and because of convexity, an optimal solution still occurs at a vertex of the simplex
        \item \textbf{Simplex alogrithm} takes as input a linear program and returns an optimal solution. It starts as some vertex of the simplex and performs a sequence of iterations. In each iteration, it moves along an edge of the simplex from a current vertex to a neighboring vertex whose objective value is no smaller than that of the current vertex. The algorithm terminates when it reaches a local minimum, i.e. all neighboring vertices have a smaller objective value. 
        \begin{lemma*}
            \textbf{Duality} Since a feasible region is convex and objective function is linear, a local optimum from a simplex algorithm is a global optimum
        \end{lemma*}
        \begin{enumerate}
            \item Write linear program in slack form
            \item \textbf{Pivot} Make one variable basic and another nonbasic
        \end{enumerate}
    \end{enumerate}
\end{defn*}

\begin{defn*}
    \textbf{Standard form}
    \begin{enumerate}
        \item \textbf{Specification} Given $n$ real number $c_1,\cdots, c_n \in \R$ and $m$ real number $b_1,\cdots, b_m\in\R$ and $mn$ real number $a_{ij}$ for $i=1,\cdots,m$ and $j=1,\cdots, n$. We wish to find $n$ real numbers $x_1,\cdots, x_n$ such that 
        \begin{align*}
            \text{Maximize  } &\sum_{k=1}^n c_j x_j \\
            \text{subject to  } &\sum_{j=1}^n a_{ij} x_j \leq b_i \text{ for } i= 1,\cdots, m \\
            & x_j \geq 0 \text{ for } j = 1,\cdots, n\\
        \end{align*}
        Note standard form requires the $n$ nonnegative constraints on $x_1,\cdots,x_n$. Alternatively, let $A = (a_{ij})$ be $m\times n$ matrix, $b = (b_i)$ a $m$-vector, $c = (c_j)$ a $n$-vector, and $x = (x_j)$ an $n$-vector. Then 
        \begin{align*}
            \text{Maximize  } & c^Tx \\
            \text{subject to  } & Ax \leq b \\
            & x\geq 0\\
        \end{align*}
        Therefore a standard form can be expressed with $(A,b,c)$
        \item \textbf{Re-definition}
        \begin{enumerate}
            \item \textbf{Feasible \& Infeasible solution} A setting of variable $\bar{x}$ satisfies all constraints a fesasible solution, whereas a setting of $\bar{x}$ that fails to satisfy at least one constraint if an infeasible solution . 
            \item \textbf{Objective Value} A solution $\bar{x}$ has objective value $c^T \bar{x}$ 
            \item \textbf{Optimal solution \& Optimal Objective Value} a feasible solution $\bar{x}$ whose objective value is maximum over all feasible solutions is an optimal solution, its objective value $c^T \bar{x}$ is the optimal objective vlaue 
            \item \textbf{Feasible \& Unfeasible LP} If a linear program has no feasible solution, then it is infeasible, otherwise it is feasible 
            \item \textbf{Unbounded LP} If a linear program has some feasible solution but does not have a finite optimal objective value, then LP is unbounded 
        \end{enumerate}
        \item \textbf{Converting linear program (4 types) to standard form}
        \begin{enumerate}
            \item \textbf{Equivalent LP} 
            \begin{enumerate}
                \item Two maximization linear programs $L$ and $L"$ are equivalent if for each feasible solution $\bar{x}$ to $L$ with objetive value $z$, there is a corresponding solution $\bar{x}'$ to $L'$ with objective value $z$, and vice versa
                \item A minimization linear program $L$ and a maximization linear program $L'$ are equivalent if for each feasible solution $\bar{x}$ to $L$ with objective value $z$, there is a corresponding feasible solution $\bar{x}$ to $L'$ with objective value $-z$, and vice versa
            \end{enumerate}
            \item \textbf{Objective function is a minimization rather than a maximization} \\
            \begin{center}
                Negate coefficients ($c' = -c$) in the objective function.
            \end{center}
            2 LP's are equivalent since we have the same feasible solution (constraints unchanged) and for each feasible solution, the objective value in $L$ is the negative of the objective value in $L'$ hence 2 linear programs are equivalent
            \item \textbf{There might be variables without nonnegativity constraints}
            \begin{center}
                Rerplace each occurrence of a variable variable $x_j$ without nonnegativity constraint by $x_j' - x_j''$, and add the nonnegativity constraint $x_j' > 0$ and $x_j'' > 0$
            \end{center}
            \item \textbf{There might be equality constraints}
            \begin{center}
                Replace equality constraints with a pair of inequality constraints 
                \[
                    f(x_1,\cdots, x_n) \leq b \quad \quad \quad f(x_1,\cdots, x_n) \geq b
                \]
            \end{center}
            \item \textbf{There might be $\geq$ inequality constraints}
            \begin{center}
                Multiple the greater than or equal to $\geq$ constraints to less than or equal to $\leq$ constraints by multiplying these constraints by -1
                \[
                    \sum_{j=1}^n a_{ij}x_j \geq b_i \quad \iff \quad  -\sum_{j=1}^n a_{ij}x_j \geq -b_i  
                \]
            \end{center}
        \end{enumerate}
    \end{enumerate}
    
\end{defn*}


\begin{defn*}
    \textbf{Slack form}
    \begin{enumerate}
        \item \textbf{Slack variable} Given inequality constraints $\sum_{j=1}^n a_{ij} x_j \leq b_i$, we have  
        \begin{align*}
            s &= b_i - \sum_{j=1}^n a_{ij} x_j \\
            s &\geq 0 \\
        \end{align*}
        where $s$ is a slack variable because it measures the slack, or difference, between left-hand and right-hand sides of equation. We can use this methods to convert from standard form to slack form, where the only inequality constraints are the nonnegativity constraints 
        \item \textbf{Conversion from standard to slack form} Use $x_{n+i}$ instead of $s$ to denote the slack variable associated with the $i$-th inequality. The $i$-th constriant is therefore 
        \[
            x_{n+i} = b_i - \sum_{j=1}^n a_{ij} x_j \quad \quad \quad \quad x_{n+i}\geq 0
        \]
        \item \textbf{Basic \& Nonbasic variables} Given a slack form with a set of equality constriants, one of variables on left-hand side of equality and all others on the right-hand side. The variables on the left-hand side of equalities are basic variables, and those on the right-hand side are nonbasic variables. Nonbasic variables are the only varaibles that constitutes the objective function
        \item \textbf{Slack Form} Let $z$ be the value of the objective function and linear inequalities be converted to a set of slack variables. Omit the nonnegativity constraints since it is assumed that all variables are nonnegative. Let $N$ be the set of indices of nonbasic variables, let $B$ be set of indices of the basic variables, we always have $|N| = n$ and $|B| = m$, where $N\cup B = \{ 1,\cdots, n+m\}$. 
        \begin{enumerate}
            \item equations are indexed by entries of $B$
            \item variables on RHS of equation are index by entries of $N$
        \end{enumerate}
        Let $A,b,c,$ be constants and coefficients. Let $v$ be the constant term in objective function. Therefore, we define a slack form by a tuple $(N, B, A, b, c, v)$ where 
        \begin{align*}
            z &= v+ \sum_{j\in N} c_j x_j \\
            x_i &= b_i - \sum_{j\in N} a_{ij} x_j \quad\quad \text{for } i\in B
        \end{align*}
        Note indices into $A,b,c$ are no necessarily sets of contiguous integers, they depend on the index sets $B$ and $N$
        
    \end{enumerate}
\end{defn*}

\subsection*{Formulating problems as Linear Programs}


\begin{defn*}
    \textbf{Shortest path} Given a weighte, directed graph $G=  (V,E)$ with weights $w:E\to\R$ and source $s$ and destination $t$. Wish to ompute the value $d_t$, i.e. the weight of a shortest path from $s$ to $t$. We can formulate it as LP as follows 
    \begin{align*}
        \text{Maximize  } \quad & d_t \\
        \text{Subject to  }\quad  & d_v \leq d_u + w(u,v) \quad \quad \text{for each } (u,v)\in E \\
        & d_s = 0
    \end{align*}
    The bellman-form algorithm sets source vertex distance $d_s = 0$ and never changes it. When the algorithm terminates, it has computed, for each $v$, a value $d_v$ such that for each edge $(u,v)\in E$, we have $d_v \leq d_u + w(u,v)$ \\
    Note we are \textbf{maximizing} $d_t$ for 2 reasons
    \begin{enumerate}
        \item setting $\bar{d}_v = 0$ for all $v\in V$ yields optimal solution without solving shortest-path problem 
        \item Maximize because an optimnal solution to shortest path problem sets each $\bar{d}_v$ to be $\underset{u: (u,v)\in E}{Min}\{d_u + w(u,v) \}$ (considers all incident edges to $v$) such that $d_v$ is the maximum value that is less than or equal to values in the set $\{ \bar{d}_u + w(u,v)\}$. We maximize $d_v$ for all vertex $v$ on a shortest path from $s$ to $t$ subject to constraints, and maximizing $d_t$ achieves this...
    \end{enumerate}
\end{defn*}



\begin{defn*}
    \textbf{Maximum flow}
    Given directed graph $G = (V,E)$ with nonnegative capacity $c:E\to\R^+$ and two vertices, a source $s$ and a sink $t$. A flow $f: V\times V\to \R$ satisfies capacity constraint and flow conservation. A maximum flow is a flow that satisfies these constraints and maximizes the flow value. Also we assume $c(u,v) = 0$ if $(u,v)\not\in E$ and no antiparallel edges
    \begin{align*}
        \text{Maximize  } \quad & \sum_{v\in V} f_{sv} - \sum_{v\in V} f_{vs} \text{ (Value of a flow)} \\
        \text{Subject to  }\quad  & f_{uv} \leq c(u,v) \quad \quad \text{for each } u,v\in V \text{ (capacity constraint)} \\
        & \sum_{v\in V} f_{vu} = \sum_{v\in V} f(u,v) \quad \text{for each } u\in V \setminus \{s,t \} \text{ (flow conservation)} \\
        & f_{uv} \geq 0 \\
    \end{align*}
\end{defn*}


\begin{defn*}
    \textbf{Min-Cost-Flow} Consider the that in addition to a capacity of each edge $(u,v)$ in a max flow problem, we are givne a real-valued cost $a(u,v)$. Assume $c(u,v) = 0$ if $(u,v)\not\in E$ and that there are no antiparallel edges. If we send $f_{uv}$ units of flow over edge $(u,v)$, we incur a cost of $a(u,v) f_{uv}$. Given a flow demand $d$. We wish to send $d$ units of flow from $s$ to $t$ while minimizing the total cost 
    \[
        \sum_{(u,v)\in E} a(u,v) f_{uv}
    \]
    incurred by he flow. The linear program is similar to max flow problem with the constraint that value of flow be exactly $d$ cost. 
    \begin{align*}
        \textbf{Minimize } \quad & \sum_{(u,v)\in E} a(u,v) f_{uv}\\
        \textbf{Subject to } \quad & f_{uv} \leq c(u,v) \quad \text{for each } u,v\in V\\
                                & \sum_{v\in V} f_{vu} - \sum_{v\in V} f_{uv} = 0 \quad \text{for each } u\in V \setminus \{s, t \}\\
                                & |f| = \sum_{v\in V} f_{sv} - \sum_{v\in V} f_{vs} = d \\
                                & f_{uv} \geq 0 \quad \text{for each } u,v\in V\\
    \end{align*}
\end{defn*}

\begin{defn*}
    \textbf{Multicommodity flow} Given a directed graph $G = (V,E)$ in which each edge $(u,v)\in E$ has nonnegative capacity $c(u,v) \geq 0$. Assume $c(u,v) = 0$ for $(u,v)]\neq E$ and graph has no antiparallel edges. In addition, given $k$ different commodities, $K_1, \cdots, K_k$ where we specify commodity $i$ by triple $K_i = (s_i, t_i, d_i)$. Here $s_i$ is source of $i$ and $t_i$ is sink of $i$ and $d_i$ is demand for $i$, which is the desired flow value for the commodity from $s_i$ to $t_i$. Define a flow for commodity $i$ by $f_i$ ($f_{iuv}$ is flow of commodity $i$ from $u$ to $v$) be real-valued function satisfying flow-conservation and capacity constraint. Define $f_{uv}$ be \textbf{aggregate flow}, to be the sum of the various commodity flows, i.e. 
    \[
        f_{uv} = \sum_{i=1}^k f_{iuv}
    \]
    where the aggregate flow $f_{uv}$ must be no more than capacity of edge $(u,v)$. Since we are tyring to determine if such flow exists,we write a linear program with a null objective function ... 
    \begin{align*}
        \textbf{Minimize} \quad & 0\\
        \textbf{Subject to} \quad & \sum_{i=1}^k f_{iuv} \leq c(u,v) \quad \text{for each } u,v\in V\\
                                & \sum_{v\in V} f_{iuv} - \sum_{v\in V} f_{ivu} = 0 \quad \text{for each } i = 1,\cdots, k \text{ and } u\in V - \{ s_i, t_i\}\\
                                &  \sum_{v\in V} f_{is_iv} - \sum_{v\in V} f_{ivs_i} = d_i \quad \text{for each } i = 1, \cdots, k \\
                                & f_{iuv}\geq 0 \quad \text{for each } u,v\in V \text{ and } i = 1,\cdots k\\
    \end{align*}
\end{defn*}





\subsection*{Simplex algorithm} 


\begin{defn*}
    \textbf{Simplex algorithm}
    \begin{enumerate}
        \item \textbf{Steps}
        \begin{enumerate}
            \item In each iteration, find basic solution from slack form of linear program
            \begin{enumerate}
                \item set each nonbasic variable $N$ to 0,
                \item compute the values of basic variables $B$ from the equality constraint
                \item compute objective value using objective function 
            \end{enumerate} 
            \item Convert one slack form into an equivalent slack form 
            \begin{enumerate}
                \item Pick a nonbasic \textbf{entering variable} $x_e \in  N$ such that if we were to increase that variable's value from 0, then the objective function would increase, too (i.e. positive coefficient in objective function)
                \item Raise the nonbasic variable $x_e$ as much as possible without violating any constraint (i.e. until some basic \textbf{leaving variable} $x_l$ becomes 0, slack becomes tight) 
                \item rewrite the slack form, exchanging the roles of $x_e$ and $x_l$, substitute expression of the now-basic $x_e$ to other constraint/objective
            \end{enumerate}
            \item stops when there is no applicable entering variable left (i.e. coefficient of nonbasic variable in objective function have all negative coefficient)
        \end{enumerate}
        \item \text{Definition} 
        \begin{enumerate}
            \item \textbf{Tight} An equality constraint is tight for a particular setting of its nonbasic variables if they cause the constraint's basic variable to become 0. A setting of nonbasic variable that makes a basic variable become negative violates that constraint, therefore slack 
            \begin{enumerate}
                \item maintains explicitly how far each constraint is from being tight 
                \item help to determine how much we can increase values of nonbasic variables without violating any constraints
            \end{enumerate}
            \item \textbf{Basic feasible solution} A basic solution that is also feasible. (As we run the simplex algorithm, the basic solution is almost always a basic feasible solution except for maybe the first several..)
        \end{enumerate}
        \item \textbf{Pivoting} Exchanging role of $x_l$ with $x_e$
        \begin{enumerate}
            \item takes a slack form given by tuple $(N,B, A, b,c ,v)$ as input, index $l$ of leaving variable $x_l$ and index $e$ of entering variable $x_e$
            \item Returns a tuple $(\hat{N}, \hat{B}, \hat{A}, \hat{b}, \hat{c}, \hat{v})$ describing the new slack form 
        \end{enumerate}
    \end{enumerate}
\end{defn*}


\begin{lemma*}
    Consider a call to $\textsc{Pivot}(N, B, A, b, c, v, l, e)$ in which $a_{le}\neq 0$. Let values returned be $(\hat{N}, \hat{B}, \hat{A}, \hat{b}, \hat{c}, \hat{v})$, and let $\overline{x}$ denote the basic solution after the call, then 
    \begin{enumerate}
        \item $\overline{x}_j = 0$ for each $j\in \hat{N}$
        \item $\overline{x}_e = \frac{b_l}{a_{le}}$ ( since all other nonbasic variable set to 0)
        \item $\overline{x} = b_i - a_{ie} \hat{b}_e$ for each $i\in \hat{B} \setminus \{ e\}$
    \end{enumerate}
\end{lemma*}


\begin{defn*}
    \textbf{Formal simplex algorithm}
    \begin{enumerate}
        \item $\textsc{Initialize-Simplex}(A,b,c)$ takes input of linear program in standard form, 
        \begin{enumerate}
            \item if problem infeasible, returns a message that program is infeasible 
            \item otherwise, return a slack form for which the initial basic solution is feasible
        \end{enumerate}
        \item Pick the next positive-coefficient nonbasic variable $e$ in the objective function such that $c_e > 0$
        \item Now populate $\triangle_i = \frac{b_i}{a_{ie}}$ for $i\in B$, i.e. the amount by which $e$ can increase by before the $i$-th basic variable becomes zero (i.e. satisfy nonnegativity constraint)
        \begin{enumerate}
            \item if $\triangle_i =  \infty$, then solution unbounded 
            \item otherwise pick an index $l\in B$ such that $\triangle_i$ is minimized. and call $\textsc{Pivot}$ on $e$ and $l$ which returns the next slack form 
        \end{enumerate}
        Loop until no variable has positive coefficient in objective function
        \item Based on the last slack form, compute and return the corresponding basic solution $\overline{x}$
        \begin{enumerate}
            \item if $i\in B$, then $\overline{x}_i = b_i$ 
            \item otherwise, $\overline{x}_i = 0$
        \end{enumerate}
    \end{enumerate}
\end{defn*}


\begin{lemma*}
    Given a linear program $(A,b,c)$, suppose call to \textsc{Initialize-Simplex} returns a slack form for which the basic solutin is feasible, then if \textsc{Simplex} returns a solution, that solution is a feasible solution to the linear program. If \textsc{Simplex} returns unbounded then the linear program is unbounded
\end{lemma*}


\begin{lemma*}
    Let $(A,b,c)$ be a linear program in standard form. Given a set $B$ of basic variables, the associated slack form is uniquely determined. 
\end{lemma*}

\begin{lemma*}
    If \textsc{Simplex} fials to terminate in at most $\binom{n+m}{m}$ iterations, the it cycles. 
    \begin{proof}
        By previous lemma, set $B$ uniquely determines. There are $n+m$ variables and $|B| = m$, and therefore, there are at most $\binom{n+m}{m}$ ways to choose $B$. Thus at most $\binom{n+m}{m}$ unique slack forms. Therefore if \textsc{Simplex} runs for more than $\binom{n+m}{m}$ iterations, it must cycle
    \end{proof}
    
\end{lemma*}

\begin{lemma*}
    \textsc{Bland's rule} To prevent cycling, we pick $e$ and $l$ such that we break ties by always choosing the variable with smallest index, then \textsc{Simplex} must terminate
\end{lemma*}

\begin{lemma*}
    Assume \textsc{Initialize-Simplex} returns a slack form for which the basic solutin is feasible, \textsc{Simplex} either reports a linear program is unbounded or it terminates with a feasible solution in at most $\binom{n+m}{m}$  iterations. 
\end{lemma*}



\subsection*{Duality}


\begin{defn*}
    
\end{defn*}


\end{document}
