

\documentclass[11pt]{article}
\input{"../preamble"}

% begin document
\begin{document}



\begin{enumerate}[label=\alph*]
  \item \\
  Since we are only interested in the sum, Let
  \[
    \Omega = \{ x\in \mathbb{Z}: 2 \leq x\leq 12\}
  \]
  \item \\
  Let $X$ be a discrete random variable associated with the sum of two dices in one toss. When $X=5$, they are 4 possible outcomes.
  \[
    \{ (1,4), (2,3), (3,2), (4,1)\}
  \]
  The total number of outcome for rolling two dices is $6\times 6 = 36$. Therefore $$P(X=5) = \frac{4}{36} = \frac{1}{9}$$
  \item \\
  When one of the dice is a 4, there are following possibilities of dice toss.
  \[
    (4,1), (4,2), (4,3), (4,4), (4,5), (4,6), (1,4), (2,4), (3,4), (4,4), (5,4), (6,4)
  \]
  We can see that only $(4,1)$ and $(1,4)$ has a sum of 5 and that one of the dice toss yield a 4. Therefore
  \begin{align*}
    P(X=5 | \text{one of the dice is a 4})  &= \frac{P(X=5 \cap \text{one of the dice is a 4})}{P(\text{one of the dice is a 4})} &= \frac{\frac{2}{36}}{\frac{12}{36}} = \frac{1}{6}
  \end{align*}
  \item \\
  \begin{align*}
    P(\text{one of the dice is a 4}| X=5) &= \frac{P(\text{one of the dice is a 4} \cap X=5)}{P(X=5)}
    &= \frac{\frac{2}{36}}{\frac{1}{9}}
    &=\frac{1}{2}
  \end{align*}
  \item \\
  To find $P(X\geq 5)$. We find $P(X\leq 4)$
  \begin{align*}
    P(X\geq 5) &= 1-P(X\leq 4) \\
    &= 1- (P(X=1) + P(X=2) + P(X=3) + P(X=4)) \\
    &= 1 - (0+\frac{1}{36} + \frac{2}{36} + \frac{3}{36}) \\
    &= \frac{5}{6}
  \end{align*}
  This is because the dice pairs that satisfy $X=1,2,3,4$ are
  \begin{align*}
    & \text{Not possible} \tag{X=1} \\
    & (1,1) \tag{X=2} \\
    & {1,2}, (2,1) \tag{X=3} \\
    & (1,3), (2,2), (3,1) \tag{X=4}
  \end{align*}
\end{enumerate}




\begin{enumerate}[label=\alph*]
  \item
  \begin{align*}
    P(C\cap A^c | B) &= \frac{P(C\cap A^c \cap B)}{P(B)} \\
    &= \frac{P(B\cap C) - P(A\cap B\cap C)}{P(B)} \\
    &= \frac{P(B\cap C)}{P(B)} -  \frac{P(A\cap B\cap C)}{P(B)} \\
    &= P(C|B) - P(A\cap C| B) \\
    &= 0.5 - 0.1 \\
    &= 0.4
  \end{align*}

  \item
  \begin{align*}
    P\bigg((A\cap C^c)&\cup (A^c\cap C)| B\bigg) \\
    &= \frac{P\bigg(\big((A\cap C^c)\cup (A^c\cap C) \big)\cap B\bigg)}{P(B)} \\
    &= \frac{P\bigg(  (A\cap B\cap C^c)\cup (A^c\cap B\cap C)  \bigg)}{P(B)} \\
    &= \frac{P(A\cap B\cap C^c) + P(A^c\cap B\cap C)}{P(B)} \tag{two sets are disjoint} \\
    &= \frac{P(A\cap B) - P(A\cap B\cap C)}{P(B)} + 0.4 \tag{$\frac{P(C\cap A^c \cap B)}{P(B)} = 0.4$ from $a$} \\
    &= P(A|B) - P(A\cap C|B) + 0.4 \\
    &= 0.25 - 0.1 + 0.4 \\
    &= 0.55
  \end{align*}

  \item
  \begin{align*}
    P(A\cup C|B) &= \frac{P((A\cup C)\cap B)}{P(B)} \\
    &= \frac{P((A\cap B)\cup (B\cap C))}{P(B)} \\
    &= \frac{P(A\cap B) + P(B\cap C) - P(A\cap B\cap C)}{P(B)} \\
    &= P(A|B) + P(C|B) - P(A\cap C|B) \\
    &= 0.25 + 0.5 - 0.1 \\
    & = 0.65
  \end{align*}
\end{enumerate}

\newpage


Event associated with randomly selecting 4 shoes from 16 shoes without replacement and order does not matter has $\binom{16}{4}$ possible outcomes. Event associated with having exactly 1 complete pair has $\binom{8}{1}\binom{14}{2}$ outcomes. Therefore probability of having exactly 1 pair by selecting randomly 4 shoes from 8 pairs of shoes is
\[
  P = \frac{\binom{8}{1}\binom{14}{2}}{\binom{16}{4}} = 0.4
\]


\end{document}
