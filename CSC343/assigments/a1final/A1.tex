\documentclass{article}
\usepackage{fullpage}
\usepackage[normalem]{ulem}
\usepackage{amstext}
\usepackage{amsmath}
\newcommand{\var}[1]{\mathit{#1}}
\setlength{\parskip}{6pt}

\title{CSC343, Fall 2017 A1}
\author{Peiqi Wang (1001132561), Xiaoye Yuan (1002342084)}

\begin{document}

\maketitle  

\noindent


%----------------------------------------------------------------------------------------------------------------------
\section*{Part 1: Queries}

\begin{enumerate}

\item   % ----------
Find all the users who have never liked or viewed a post or story 
of a user that they do {\it not} follow. 
Report their user id and ``about" information. 
Put the information into a relation with attributes ``username'' and ``description''. \\

{\large %This increase in font size makes the subscripts much more readable.

-- $A$ liked $B$'s post \\ 
$\var{AlikedB}(A,B) := $
\[
    \rho_{\var{AlikesB}(A,B)}
    \Pi_{liker, uid} 
    (\var{Likes} \bowtie \var{Post})
\]

-- $A$ saw $B$'s story \\ 
$\var{AsawB}(A,B) := $
\[
    \rho_{\var{AsawB}(A,B)}
    \Pi_{viewerid, uid} 
    (\var{Saw} \bowtie \var{Story})
\]

-- $A$ did not follow $B$ \\ 
$\var{ANotFollowB}(A,B) :=$
\begin{align*}
    &\left(\rho_{\var{UserCombination}(follower,followed)} \Pi_{U1.uid, U2.uid}( \rho_{U1}\var{User}\times\rho_{U2} \var{User})\right)  \\ 
    &- \Pi_{follower, followed} \var{Follows}\\ 
\end{align*}

-- User who have liked a post or saw a story of a user they do not follow \\ 
$\var{TheCuriousOnes}(uid) :=$
\begin{align*}
    &\Pi_{U.uid}
    \sigma_{
        \substack{
            U.uid=NF.A \,\land \\
            ((U.uid=L.A \,\land\, L.B=NF.B) \lor
            (U.uid=S.A \,\land\, S.B=NF.B))
        }
    }\\
    &(\rho_{U}\var{User} \times \rho_{NF} \var{ANotFollowB} \times \rho_{L} \var{ALikesB} \times \rho_{S} \var{ASawB})\\
\end{align*}

-- user who have not liked a post or saw a story of user they do not follow\\ 
$\var{TheUncuriousOnes}(uid) :=$
\[
    (\Pi_{uid}\var{User}) - \var{TheCuriousOnes}
\]

$\var{Solution}(username, description) :=$
\[
    \rho_{Solution(username, description)}  \Pi_{uid, about} 
    (\var{TheUncuriousOnes} \bowtie \var{User})
\]

} % End of font size increase.


\item   % ----------
Find every hashtag that has been mentioned in at least three post captions
on every day of 2017.
You may assume that there is at least one post on each day of a year. 

{\large 



-- Tags in 2017 \\
$\var{Post2017}(pid, uid, when, location, caption) :=$
\[
    \sigma_{when.year=2017} Post
\]

-- Combine tag and time info in same place \\ 
$\var{PostAndTag}(pid, tag, when) :=$
\[
    \Pi_{pid, tag, when} (\var{Post2017} \bowtie \var{Hashtag})
\]

-- Hashtags mentioned in at least 3 post captions \\ 
$\var{PossiblyPopularTags}(tag, when) :=$
\[
    \Pi_{tag, when}
    \sigma_{
        \substack{T1.when=T2.when=T3.when \\ 
        \land\, T1.tag=T2.tag=T3.tag  \\
        \land\, (T1.pid != T2.pid) \land (T1.pid!= T3.pid)\land (T2.pid!=T3.pid) }}
\]
\[
    (\rho_{T1} \var{PostAndTag} \times \rho_{T2} \var{PostAndTag} \times \rho_{T2} \var{PostAndTag})
\]

-- Expected tag-when pair \\
$\var{PopularTagsChecklist}(tag, when) :=$
\[
    \Pi_{tag} \var{PossiblyPopularTags} \times \Pi_{when} \var{Post}
\]

-- Tags that fails to be popular for $>=1$ days  \\ 
$\var{NotConsistentlyPopularTags}(tag, when) :=$
\[
    PopularTagsChecklist - PossiblyPopularTags
\]

-- Tags that is popular for all days in a year \\
$\var{Solution}(tag) :=$
\[
    \Pi_{tag} \var{PossiblyPopularTags} - \Pi_{tag} \var{NotConsistentlyPopularTags}
\]


}


\item   % ----------	
Let's say that a pair of users are ``reciprocal followers" if they follow each other. 
For each pair of reciprocal followers, 
find all of their ``uncommon followers": 
users who follow one of them but not the other. 
Report one row for each of the pair's uncommon follower.
In it, include
the identifiers of the reciprocal followers,
and the identifier, name and email of the uncommon follower. 

{\large


-- Create a table containing all reciprocal followers without duplication \\
$\var{ReciprocalFollowers}(follower,followed) :=$
\begin{align*}
    &\rho_{\var{ReciprocalFollowers}(follower,followed)}\\
	&\Pi_{F1.follower, F2.follower} 
	\left(
		\sigma_{
			\substack{
				F1.follower = F2.followed\\
				\land F2.follwer = F1.followed\\
				\land F1.follower < F1.followed
			} 
		}
		(\rho_{F1} \var{Follows} \times \rho_{F2} \var{Follows})
	\right)\\
\end{align*}


-- Create a table containing all followers who follows at least one of the reciprocal followers A and B \\
$\var{AllFollowers}(follower,followed) :=$
\begin{align*}
    &\rho_{\var{AllFollowers}(follower,followed)} \Pi_{Follows.follower,Follows.followed}\\
	&
	\left(
		\sigma_{
			\substack{
				Follows.followed = ReciprocalFollowers.follower\\
				\lor Follows.followed = ReciprocalFollowers.followed \\
			}
		}
		(\var{Follows} \times \var{ReciprocalFollowers})
	\right)\\ 
\end{align*}


-- Create a table containing all followers who follows both reciprocal followers \\
$\var{CommonFollowers}(follower):=$
\begin{align*}
    &\rho_{\var{CommonFollowers}(follower)}\\
	&\Pi_{f_3.follower}
	\left(
		\sigma_{
			\substack{
				f_3.followed = ReciprocalFollowers.follower \\
				\land f_4.followed = ReciprocalFollowers.followed \\
				\land f_3.follower = f_4.follower
			}
		}
		(\rho_{f_3} \var{Follows} \times \rho_{f_4} \var{Follows} \times \var{ReciprocalFollowers})
	\right) \\
\end{align*}


--Create a table to store the identifier for all uncommon followers \\
$\var{UncommonFollowersIndentifier}(follower) := $
\[
	\Pi_{follower} \var{AllFollowers} - \var{CommonFollowers}
\]

--Final result \\
$\var{UncommonFollower}(reciprocalfollower, uncommon follower, name, email):=$
\begin{align*}
    &\rho_{\var{UncommonFollower}(reciprocalfollower, uncommon follower, name, email)}\\
	&\Pi_{f_5.followed,f_5.follower,name,email}\\
		&\sigma_{
			\substack{
				f_5.follower = UncommonFollowersIndentifier.follower \\
				\land (f_5.followed = ReciprocalFollowers.follower \lor f_5.followed = ReciprocalFollowers.followed)\\
				\land f_5.follower = User.uid
			}
		}\\
		&(\rho_{f_5} \var{Follows} \times \var{ReciprocalFollowers} \times \var{UncommonFollowersIndentifier} \times \var{User})\\
\end{align*}

}

\item   % ---------- 
Find the user who has liked the most posts. 
Report the user's id, name and email, and the id of the posts they have liked. 
If there is a tie, report them all.

cannot be expressed

\item   % ----------
Let's say a pair of users are ``backscratchers" 
if they follow each other and like all of each others' posts. 
Report the user id of all users who follow some pair of backscratcher users.

{\large 

-- $A$ and $B$ follows each other, appear twice in table  \\ 
$\var{FollowEachOther}(A,B) :=$
\[
    \rho_{\var{FollowEachOther}(A,B)}
    \Pi_{F1.follower, F1.followed}
    \sigma_{
        \substack{
            F1.follower=F2.followed\\
            F1.followed=F2.follower
        }
    } \left( \rho_{F1} \var{Follows} \times \rho_{F2} \var{Follows} \right)
\]


-- Relation containing $B$'s posts for which the $A$ liked  \\ 
$\var{ALikedBPost}(A,B,pid) :=$
\[
    \rho_{ALikedBPost(A,B,pid)}
    \left(
        \Pi_{F.A, F.B, L.pid}
        \sigma_{
            \substack{  
                F.A=L.liker\\
                F.B=P.uid \\
                L.pid=P.pid
            }
        }
        (\rho_{F} \var{FollowEachOther} \times \rho_{L} \var{Likes}\times \rho_{P} \var{Post})
    \right)
\]

-- Expected relation where $A$ liked every post made by $B$ given that $A$ and $B$ follows each other \\ 
$\var{ALikedBEveryPost}(A,B,pid) :=$ 
\[
    \rho_{ALikedBEveryPost(A,B,pid)}
    \Pi_{F.A, F.B, P.pid} (\rho_{F} \var{FollowEachOther} \bowtie_{F.B=P.uid} \rho_{P} \var{Post})
\]


-- Relation where $A$ did not liked every post made by $B$ \\ 
$\var{DidNotLikedEveryPost}(A,B, pid) :=$
\[
    \var{ALikedBEveryPost} - \var{ALikedBPost}
\]

-- Observed relation where $A$ liked every post made by $B$\\
$\var{AIsBackScratcherOfB}(A,B) :=$
\[
    \Pi_{A,B} \var{ALikedBPost} - \Pi_{A,B} \var{DidNotLikedEveryPost}
\]

-- Observed relation where $A$ liked every post made by $B$. and vice versa\\
$\var{ABAreBackScratchers}(A,B) :=$
\[
    \rho_{\var{ABAreBackScratchers}(A,B)} \Pi_{T1.A, T1.B}
\]
\[
    \rho_{T1} \var{AIsBackScratcherOfB} \bowtie_{T1.A=T2.B \,\land\, T1.B=T2.A} \rho_{T2} \var{AIsBackScratcherOfB}
\]


-- Solution \\
$\var{Solution}(uid) :=$
\[
    \rho_{\var{Solution}(uid)}
    \Pi_{A} \var{ABAreBackScratchers}
\]


}
\item   % ----------
The ``most recent activity" of a user is his or her latest story or post. 
The ``most recently active user" is the user whose most recent activity
occurred most recently.

Report the name of every user, 
and for the \textbf{most recently active user they follow},
report their name and email, and the date of their most-recent activity.
If there is a tie for the most recently active user that a user follows,
report a row for each of them.


{\large  


-- User's not most recent posts \\
$\var{NotMostRecentPost}(uid, when) :=$
\begin{align*}
    \rho_{\var{NotMostRecentPost}(uid, when)}
    &\Pi_{P1.uid, P1.when}\\
    &\sigma_{P1.when < P2.when}
    \left( 
        \rho_{P1} \var{Post} \times \rho_{P2} \var{Post}
    \right)
\end{align*}

-- User's most recent posts \\ 
$\var{MostRecentPost}(uid, when) :=$
\[
    \Pi_{uid, when} \var{Post} - \var{NotMostRecentPost}
\]


-- User's not most recent stories \\ 
$\var{NotMostRecentStory}(uid, when) :=$
\begin{align*}
    \rho_{\var{NotMostRecentStory}(uid, when)}
    &\Pi_{S1.uid, S1.when}\\
    &\sigma_{S1.when < S2.when}
    \left( 
        \rho_{S1} \var{Story} \times \rho_{S2} \var{Story}
    \right)
\end{align*}

-- User's most recent stories \\ 
$\var{MostRecentStory}(uid, when) :=$
\[
    \Pi_{uid, when} \var{Story} - \var{NotMostRecentStory}
\]


-- User's most recent post and most recent story in one table  \\ 
$\var{MostRecentCombined}(uid, when) :=$
\[
    \var{MostRecentStory} \bigcup \var{MostRecentPost}  
\]

-- User's most recent activity, one for each user, unless there is a tie \\ 
$\var{MostRecentActivity}(uid, when) :=$
\begin{align*}
    \var{MostRecentCombined} \,\,- &\\ 
    &\rho_{\var{NotMostRecentActivity}(uid, when)} \Pi_{C1.uid, C1.when} \\
    &\sigma_{C1.when < C2.when} 
        (\rho_{C1}\var{MostRecentCombined} \times \rho_{C2}\var{MostRecentCombined})
\end{align*}


-- Relation containing users, and all users they follow with their respective most recent activity time\\
$\var{FollowActivity}(A, B, when) :=$
\begin{align*}
    \rho_{\var{FollowActivity}(A, B, when)}
    &\Pi_{F.follower, F.followed, M.when}\\ 
    &(\rho_{F} \var{Follows} \bowtie_{F.followed=M.uid} \rho_{M} \var{MostRecentActivity})
\end{align*}

-- Relation containing users, and only the most recently active users they follow\\
$\var{MostRecentlyActiveFollow}(A, B, when) :=$
\begin{align*}
    \var{FollowActivity}& \,\, -\\ 
    &\rho_{\var{NotRecentlyActiveFollow}(A,B,when)} \Pi_{F1.A, F1.B, F1.when} \\
    &\sigma_{F1.when < F2.when}
    (\rho_{F1} \var{FollowActivity} \bowtie_{F1.A=F2.A} \rho_{F2} \var{FollowActivity})
\end{align*}

-- Add name and email to previous relation \\
$\var{Solution}(name, followername, followeremail, followerwhen) :=$
\begin{align*}
    &\rho_{\var{Solution}(name, followername, followeremail, followerwhen)}\\ 
    &\quad \Pi_{U1.name, U2.name, U2.email, M.when} \\ 
    &\quad \quad (\rho_{M} \var{MostRecentlyActiveFollow} 
        \bowtie_{M.A=U1.uid} \rho_{U1} \var{User} 
        \bowtie_{M.B=U2.uid} \rho_{U2} \var{User})
\end{align*}


}

\item   % ----------
Find the users who have always liked posts
in the same order as the order in which they were posted,
that is,
users for whom the following is true:
if they liked $n$ different posts (posts of any users)
and
$$[post\_date\_1] < [post\_date\_2] < ... < [post\_date\_n]$$
where $post\_date\_i$ is the date on which a post $i$ was posted, 
then it holds that
$$[like\_date\_1] < [like\_date\_2] < ... < [like\_date\_n]$$ 
where $like\_date\_i$ is the date on which the post $i$ was liked 
by the user.  
Report the user's name and email.

{\large 


-- Relation that augment $\var{Likes}$ to include when post is posted  \\
$\var{LikeWhen}(uid, pid, postwhen, likewhen) :=$
\[
    \rho_{\var{LikeWhen}(uid, pid, postwhen, likewhen)}
    \Pi_{L.uid, L.pid, P.when, L.when}
    (\rho_{P} \var{Post} \bowtie \rho_{L} \var{Likes}) 
\]


-- Relation where user does not follow the ordering specified above \\
$\var{NotOrderedUser}(uid) :=$
\begin{align*}
    &\rho_{\var{NotOrderedUser}(uid)}
    \Pi_{L1.uid} \\
    &\sigma_{
        \substack{
            (L1.postwhen < L2.postwhen \,\land\, L1.likewhen \geq L2.likewhen)   \\
            \,\lor (L1.postwhen > L2.postwhen \,\land\, L1.likewhen \leq L1.likewhen)
        }
    }
    (\rho_{L1} \var{LikeWhen} \bowtie_{L1.uid=L2.uid} \rho_{L2} \var{LikeWhen})
\end{align*}
}

-- Relation where user does follow the ordering specified above \\
$\var{OrderedUser}(uid) :=$
\[
    (\rho_{all(uid)} \Pi_{liker} \var{Likes}) - \var{NotOrderedUser}
\]


-- Solution \\
$\var{Solution}(name, email) := $
\[
    \sigma_{name, email} (\var{OrderedUser} \bowtie \var{User})
\]

\item   % ----------
Report the name and email of the user
who has gained the greatest number of new followers in 2017. 
If there is a tie, report them all.

cannot be expressed

\item   % ----------
For each user who has ever viewed any story, 
    report their id and the id of the first and of the last story they have seen.
If there is a tie for the first story seen, report both;
if there is a tie for the last story seen, report both.
This means that a user could have up to 4 rows in the resulting relation.


\large{
    
    -- Create a table to record viewers and stories they viewed that weren't the last stories they viewed \\
    $\var{NotLast}(viewerid,sid):=$
    \[
        \rho_{\var{NotLast}(viewerid,sid)}
        \Pi_{s_1.viewerid,s_1.sid} 
        \sigma_{
            \substack{
                s_1.viewerid = s_2.viewerid \\
                s_1.sid \ne s_2.sid \\
                s_1.when < s_2.when \\
            }
        }\\
        (\rho_{s_1} \var{Saw} \times \rho_{s_2} \var{Saw}) 
    \]
    
    -- Create a table to record viewers and stories they viewed that were the last stories they viewed \\
    $\var{Last}(viewerid,sid):=$
    \[
        \Pi_{viewerid,sid} \var{Saw} - NotLast
    \]
    
    -- Create a table to record viewers and stories they viewed that weren't the first stories they viewed \\
    $\var{NotFirst}(viewerid,sid):=$
    \[
        \rho_{\var{NotFirst}(viewerid,sid)}
        \Pi_{s_1.viewerid,s_1.sid} 
        \sigma_{
            \substack{
                s_1.viewerid = s_2.viewerid \\
                s_1.sid \ne s_2.sid \\
                s_1.when > s_2.when \\
            }
        } 
        (\rho_{s_1} \var{Saw} \times \rho_{s_2} \var{Saw}) 
    \]
    
    -- Create a table to record viewers and stories they viewed that were the first stories they viewed \\
    $\var{First}(viewerid,sid):=$
    \[
        \Pi_{viewerid,sid} \var{Saw} - NotFirst
    \]
    
    -- Create a table to record each viewer's first and last viewed stories \\
    $\var{FirstAndLast}(viewerid,firstsid,lastsid) :=$
    \[
        \rho_{\var{FirstAndLast}(viewerid,firstsid,lastsid)}
        \Pi_{First.viewerid, First.sid, Last.sid}
    \]
    \[
        \sigma_{
                First.viewerid = Last.viewerid
        }
        (\var{First} \times \var{Last})
    \] 
    
    
    }





\item   % ----------\
A comment is said to have either positive or negative sentiment
based on the presence of words such as ``like,'' ``love,'' ``dislike,'' and ``hate.'' 
A ``sentiment shift" in the comments on a post occurs at moment $m$ iff
all comments on that post before $m$ have positive sentiment, 
while all comments on that post after $m$ have negative sentiment ---
or the other way around, with comments shifting from negative to positive sentiment.

Find posts that have at least three comments and for which there has been a sentiment shift over time. 
For each post, report the user who owns it and,
for each comment on the post,
the commenter's id, 
the date of their comment and its sentiment.

You may assume there is a function, called {\it sentiment}
that can be applied to a comment's text and 
returns the sentiment of the comment as a string with the value ``positive" or ``negative".
For example,
you may refer to $sentiment(text)$ in the condition of a select operator.



{\large 


-- Posts with at least 3 different comment. Comments for a single post are uniquely identified by commenter and when. So a post has at least three unique comments if pairwise inequality operation of 3 cross product of the $\var{Comment}$ yields true conjunctively\\ 
$\var{HotPostPid}(pid) :=$ 
\begin{align*}
    & \rho_{\var{HotPost}(pid)} \Pi_{C1.pid} \\ 
    &
    \left(
        \sigma_{
            \substack{
                (C1.pid=C2.pid=C3.pid) \\ 
                \land (C1.commenter\neq C2.commenter \,\lor\, C1.when!=C2.when)\\
                \land (C1.commenter\neq C3.commenter \,\lor\, C1.when!=C3.when)\\
                \land (C2.commenter\neq C3.commenter \,\lor\, C2.when!=C3.when)
            }
        }
        (\rho_{C1} \var{Comment} \times \rho_{C2} \var{Comment} \times \rho_{C3} \var{Comment})
    \right)
\end{align*}


-- As per explained by prof Horton, comments of a post that only has positive or only has negative sentiment does not count as sentiment shift. So here we filter $\var{HotPostPid}$ to exclude such cases \\
$\var{HostPostPidFiltered}(pid) :=$
\begin{align*}
    &\rho_{\var{HostPostPidFiltered}(pid)}\Pi_{pid}  \\
    &\sigma_{
        sentiment(H1.text) \neq sentiment(H2.text)
    }
    (\rho_{H1} (\var{HotPostPid}\bowtie \var{Comment}) \times \rho_{H2} (\var{HotPostPid} \bowtie \var{Comment}))
\end{align*}

-- Filters $\var{Comment}$ such that only comments belonging to a post with at least 3 comments exist and there should be a sentiment change\\
$\var{HotPostComments}(pid, commenter, when, text) :=$
\[
    \var{HostPostPidFiltered} \bowtie \var{Comment}
\]

-- Posts with at least 3 different comments and for which there is not a sentiment change over time. A three tuple of $\var{HotPostComments}$ does not have a sentiment change iff there is a sequence of ``positive''-``negative''-``positive'' or ``negative''-``positive''-``negative'' of comments when ordered by time \\
$\var{NotSentimentHotPost}(pid) :=$
\begin{align*}
    &\rho_{\var{NotSentimentHotPost}(pid)} 
    \Pi_{C1.pid} \\ 
    &\sigma_{
        \substack{
            (C1.pid=C2.pid=C3.pid) \\
            \land (C1.when < C2.when < C3.when) \\
            \land ((sentiment(C1.text)=``positive'' \,\land\, sentiment(C2.text)=``negative'' \,\land\, sentiment(C3.text)=``positive'') \lor \\
            (sentiment(C1.text)=``negative'' \,\land\, sentiment(C2.text)=``positive'' \,\land\, sentiment(C3.text)=``negative'')) \\ 
        }
    }\\ 
    &(\rho_{C1} \var{HotPostComments} \times \rho_{C2} \var{HotPostComments} \times \rho_{C3} \var{HotPostComments})
\end{align*}


-- Posts with at least 3 different comments and there is a sentiment shift \\
$\var{SentimentHotPost}(pid) :=$
\[
    \var{HostPostPidFiltered} - \var{NotSentimentHotPost}
\]


-- Adds owner info to above relation \\
$\var{Solution}(ownerid, commenter, commentwhen, commentsentiment) :=$
\begin{align*}
    &\rho_{\var{Solution}(ownerid, commenter, commentwhen, commentsentiment)}\\ 
    &\Pi_{P.uid, C.uid, C.when, sentiment(C.text)}
    (\rho_{T} \var{SentimentHotPost} \bowtie \rho_{P} \var{Post} \bowtie \rho_{C} \var{Comment})\\
\end{align*}



}

\end{enumerate}



%----------------------------------------------------------------------------------------------------------------------
\section*{Part 2: Additional Integrity Constraints}


Express the following integrity constraints
with the notation $R = \emptyset$, where $R$ is an expression of relational algebra. 
You are welcome to define intermediate results with assignment
and then use them in an integrity constraint.

\begin{enumerate}

\item   % ----------
A comment on a post must occur after the date-time of the post itself.
(Remember that you can compare two date-time attributes with simple $<$,
$>=$ etc.)

{\large  

\[
    \left( 
        \sigma_{P.when > C.when}
        (\rho_{C}\var{Comment} \bowtie \rho_{P} \var{Post})    
    \right)
    =\emptyset
\]

}

\item %---------
Each user can have at most one current story.


{\large 

\[
    \left(
    \sigma_{
        \substack{
            S1.current=``yes''\,\land\, S2.current=``yes'' \\ 
            S1.sid < S2.sid
        }
    }
    (\rho_{S1}\var{Story} \bowtie_{S1.uid=S2.uid} \rho_{S1} \var{Story})
    \right) = \emptyset
\]
}

\item %----------
Every post must include at least one picture or one video and so must every story.

{\large 

\[
    \left(
        \rho_{\var{PostWithNoIncludes}(id)}
        (\Pi_{pid} \var{Post} - \Pi_{pid} \var{PIncludes})
    \right)
\]
\[
    \quad \quad \quad \bigcup
    \left(
        \rho_{\var{StoryWithNoIncludes}(id)}
        (\Pi_{sid} \var{Story} - \Pi_{sid} \var{SIncludes})
    \right)
    = \emptyset
\]
}

\end{enumerate}

\end{document}

    